%%%%%%%%%%%%%%%%%%%%%%%%%%%%%%%%%%%%%%%%%%%%%%%%%%%%%%%%%%%%%%%%%%%%%%%%%%%%%%
%
%%%%%%%%%%%%%%%%%%%%%%%%%%%%%%%%%%%%%%%%%%%%%%%%%%%%%%%%%%%%%%%%%%%%%%%%%%%%%%
\chapter{xymon server programs}

%%%%%%%%%%%%%%%%%%%%%%%%%%%%%%%%%%%%%%%%%%%%%%%%%%%%%%%%%%%%%%%%%%%%%%%%%%%%%%
%
%%%%%%%%%%%%%%%%%%%%%%%%%%%%%%%%%%%%%%%%%%%%%%%%%%%%%%%%%%%%%%%%%%%%%%%%%%%%%%

\section{HOBBITD\_CHANNEL}

 xymond\_channel - Feed a hobbitd channel to a worker module 

\subsection{SYNOPSIS}
\textbf{xymond\_channel --channel=CHANNEL [options] workerprogram [worker-options]}


 
\subsection{DESCRIPTION}
 xymond\_channel hooks into one of the \emph{hobbitd(8)} channels
 that provide information about events occurring in the xymon
 system. It retrieves messages from the xymond daemon, and passes
 them on to the \textbf{workerprogram} on the STDIN (file descripter
 1) of the worker program. Worker programs can then handle messages as
 they like. 


  A number of worker programs are shipped with xymond,
  e.g. \emph{xymond\_filestore(8)}

 \emph{xymond\_history(8)}
 \emph{xymond\_alert(8)}
 \emph{xymond\_rrd(8)}
 \emph{xymond\_client(8)}



  If you want to write your own worker module, a sample worker module
  is provided as part of the xymond distribution in the
  xymond\_sample.c file. This illustrates how to easily fetch and
  parse messages. 



 
\subsection{OPTIONS}
 xymond\_channel accepts a few options. 

 \begin{description}
\item[--channel=CHANNELNAME] Specifies the channel to receive messages
  from, only one channel can be used. This option is required. The
  following channels are available:  

 ``status'' receives all xymon status- and summary-messages  
 ``stachg'' receives information about status changes  
 ``page'' receives information about statuses triggering alerts  
 ``data'' receives all xymon ``data'' messages  
 ``notes'' receives all xymon ``notes'' messages  
 ``enadis'' receives information about hosts being disabled or enabled.  
 ``client'' receives all data sent from xymon client systems 

 

\item[--net=PEERSERVER:PEERPORT] Instead of launching a worker module
  as a local task, the messages in the channel are forwarded over a
  TCP/IP connection to another host where the worker module is
  running. This is typically used for sharing the load of the heavier
  worker modules across multiple systems, e.g. the xymond\_client and
  xymond\_rrd workers may receive data this way. With this option,
  the \textbf{workerprogram} parameter is ignored and may be omitted.  

 On the remote server, the worker modules are usually launched via inetd. 

 

\item[--daemon] xymond\_channel is normally started by
  \emph{xymonlaunch(8)} as a task defined in the
  \emph{xymonlaunch.cfg(5)} file. If you are not using hobbitlaunch,
  then starting xymond\_channel with this option causes it to run as
  a stand-alone background task. 


 

\item[--pidfile=FILENAME] If running as a stand-alone daemon,
  xymond\_channel will save the proces-ID of the daemon in
  FILENAME. This is useful for automated startup- and shutdown-
  scripts. 


 

\item[--env=FILENAME] Loads the environment variables defined in
  FILENAME before starting xymond\_channel. This is normally used
  only when running as a stand-alone daemon; if xymond\_channel is
  started by xymonlaunch, then the environment is controlled by the
  task definition in the \emph{xymonlaunch.cfg(5)} file. 

 

\item[--log=FILENAME] Redirect output to this log-file. 

 

\item[--debug] Enable debugging output. 

 


\end{description}

\subsection{FILES}
 This program does not use any configuration files. 

 
\subsection{SEE ALSO}
xymond(8), hobbit(7) 

 
%%%%%%%%%%%%%%%%%%%%%%%%%%%%%%%%%%%%%%%%%%%%%%%%%%%%%%%%%%%%%%%%%%%%%%%%%%%%%%
%
%%%%%%%%%%%%%%%%%%%%%%%%%%%%%%%%%%%%%%%%%%%%%%%%%%%%%%%%%%%%%%%%%%%%%%%%%%%%%%
\newpage
\section{BB-HOSTSVC.CGI}

hgcmd{xymonsvc.cgi} - CGI program to view xymon status logs

\subsection{SYNOPSIS}
\textbf{xymonsvc.cgi [--hobbitd|--historical] [--history={top|bottom}]}



\subsection{DESCRIPTION}
\textbf{xymonsvc.cgi}
 is a CGI program to present a xymon status log in HTML form (ie, as a web page). It can be used both for the logs showing the current status, and for historical logs from the ``histlogs'' directory. It is normally invoked as a CGI program, and therefore receives most of the input parameters via the CGI QUERY\_STRING environment variable. 

  Unless the ``--historical'' option is present, the current status log is used. This assumes a QUERY\_STRING environment variable of the form  
 
HOSTSVC=hostname.servicename  
 where ``hostname'' is the name of the host with commas instead of dots, and ``servicename'' is the name of the service (the column name in xymon). Such links are automatically generated by the \emph{bbgen(1)}
 tool when the environment contains ``BBLOGSTATUS=dynamic''. 


  With the ``--historical'' option present, a historical logfile is used. This assumes a QUERY\_STRING environment variable of the form  
 
HOST=hostname\&SERVICE=servicename\&TIMEBUF=timestamp  
 where ``hostname'' is the name of the host with commas instead of dots, ``servicename'' is the name of the service, and ``timestamp'' is the time of the log. This is automatically generated by the \emph{bb-hist.cgi(1)}
 tool. 



\subsection{OPTIONS}
\begin{description}
\item[--xymond] Retrieve the current status log from \emph{hobbitd(1)}
 rather than from the logfile. This is for use with the xymon daemon from the Xymon monitor version 4. 

 

\item[--historical] Use a historical logfile instead of the current logfile. 

 

\item[--history={top|bottom|none}] When showing the current logfile, provide a ``HISTORY'' button at the top or the bottom of the webpage, or not at all. The default is to put the HISTORY button at the bottom of the page. 

 

\item[--env=FILENAME] Load the environment from FILENAME before executing the CGI. 

 

\item[--templates=DIRECTORY] Where to look for the HTML header- and footer-templates used when generating the webpages. Default: \$BBHOME/web/ 

 

\item[--no-svcid] Do not include the HTML tags to identify the hostname/service on the generated web page. Useful is this already happens in the hostsvc\_header template file, for instance. 

 

\item[--multigraphs=TEST1[,TEST2]] This causes xymonsvc.cgi to generate links to service graphs that are split up into multiple images, with at most 5 graphs per image. This option only works in xymon mode. If not specified, only the ``disk'' status is split up this way. 

 

\item[--no-disable] By default, the info-column page includes a form allowing users to disable and re-enable tests. If your setup uses the default separation of administration tools into a separate, password- protected area, then use of the disable- and enable-functions requires access to the administration tools. If you prefer to do this only via the dedicated administration page, this option will remove the disable-function from the info page. 

 

\item[--no-jsvalidation] The disable-function on the info-column page by default uses JavaScript to validate the form before submitting the input to the xymon server. However, some browsers cannot handle the Javascript code correctly so the form does not work. This option disables the use of Javascript for form-validation, allowing these browsers to use the disable-function. 

 


\end{description}
\subsection{FILES}
\begin{description}
\item[\$BBHOME/web/hostsvc\_header] HTML template header 

 

\item[\$BBHOME/web/hostsvc\_footer] HTML template footer 

 


\end{description}
\subsection{ENVIRONMENT}
\begin{description}
\item[NONHISTS=info,trends,graphs] A comma-separated list of services that does not have meaningful history, e.g. the ``info'' and ``trends'' columns. Services listed here do not get a ``History'' button. 

 

\item[TEST2RRD=test,test] A comma-separated list of the tests that have an RRD graph. 

 


\end{description}
\subsection{SEE ALSO}
xymon(7), hobbitd(1) 

%%%%%%%%%%%%%%%%%%%%%%%%%%%%%%%%%%%%%%%%%%%%%%%%%%%%%%%%%%%%%%%%%%%%%%%%%%%%%%
%
%%%%%%%%%%%%%%%%%%%%%%%%%%%%%%%%%%%%%%%%%%%%%%%%%%%%%%%%%%%%%%%%%%%%%%%%%%%%%%
%



\chapter{BBPROXY}

\section{BBPROXY}
 Section: Maintenance Commands (8) 
Updated: Version Exp: 11 Jan 2008 
 
\section{NAME}
 bbproxy - Hobbit message proxy \section{SYNOPSIS}
\textbf{bbproxy [options] --servers=IP}

 
\section{DESCRIPTION}
\emph{bbproxy(8)}
 is a proxy for forwarding Hobbit messages from one server to another. It will typically be needed if you have clients behind a firewall, so they cannot send status messages to the Hobbit server directly. 

 ~\cite{web:patchutils}hgcmd{bbproxy} serves three purposes. First, it acts as a regular proxy server, allowing clients that cannot connect directly to the Hobbit server to send messages to the Hobbit servers. Although bbproxy is optimized for handling status messages, it will forward all types of messages.  



  Second, it acts as a buffer, smoothing out peak loads if many clients try to send status messages simultaneously. bbproxy can absorb messages very quickly, but will queue them up internally and forward them to the Hobbit server at a reasonable pace. This helps even out the load on your Hobbit server.  



  Third, bbproxy merges small ``status'' messages into larger ``combo'' messages. This can dramatically decrease the number of connections that need to go from bbproxy to the Hobbit server, and is a slightly more efficient way of transmitting data to the Hobbit server. The merging of messages causes ``status'' messages to be delayed for up to 0.25 seconds before being sent off to the Hobbit server. 


 
\section{OPTIONS}
\begin{description}
\item[--servers=SERVERIP[:PORT][,SERVER2IP[:PORT]]] Specifies the IP-address and optional portnumber where incoming messages are forwarded to. The default portnumber is 1984, the standard Hobbit port number. Up to 3 servers can be specified; incoming messages are sent to all of them (except ``config'', ``query'' and ``download'' messages, which go to the LAST server only). If you have Hobbit clients sending their data via this proxy, note that the clients will receive their configuration data from the LAST of the servers listed here. This option is required. 

 

\item[--bbdisplay=SERVERIP[:PORT][,SERVER2IP[:PORT]]] Obsolete. Use ``--servers'' instead. 

 

\item[--listen=LOCALIP[:PORT]] Specifies the IP-adress where bbproxy listens for incoming connections. By default, bbproxy listens on all IP-adresses assigned to the host. If no portnumber is given, port 1984 will be used. 

 

\item[--timeout=N] Specifies the number of seconds after which a connection is aborted due to a timeout. Default: 10 seconds. 

 

\item[--report=[PROXYHOSTNAME.]PROXYSERVICE] If given, this option causes bbproxy to send a status report every 5 minutes to the Hobbit server about itself. If you have set the standard Hobbit environment, you can use ``--report=bbproxy'' to have bbproxy report its status to a ``bbproxy'' column in Hobbit. The default for PROXYHOSTNAME is the \$MACHINE environment variable, i.e. the hostname of the server running bbproxy. See REPORT OUTPUT below for an explanation of the report contents. 

 

\item[--lqueue=N] Size of the listen-queue where incoming connections can queue up before being processed. This should be large to accomodate bursts of activity from clients. Default: 512. 

 

\item[--daemon] Run in daemon mode, i.e. detach and run as a background proces. This is the default. 

 

\item[--no-daemon] Runs bbproxy as a foreground proces. 

 

\item[--pidfile=FILENAME] Specifies the location of a file containing the proces-ID of the bbproxy daemon proces. Default: /var/run/bbproxy.pid. 

 

\item[--logfile=FILENAME] Sends all logging output to the specified file instead of stderr. 

 

\item[--log-details] Log details (IP-address, message type and hostname) to the logfile. This can also be enabled and disabled at run-time by sending the bbproxy proces a SIGUSR1 signal. 

 

\item[--debug] Enable debugging output. 

 


\end{description}
\section{REPORT OUTPUT}
 If enabled via the ``--report'' option, bbproxy will send a status message about itself to the Hobbit server once every 5 minutes. 

  The status message includes the following information: 


 \begin{description}
\item[Incoming messages] The total number of connections accepted from clients since the proxy started. The ``(N msgs/second)'' is the average number of messages per second over the past 5 minutes. 

 

\item[Outbound messages] The total number of messages sent to the Hobbit servers. Note that this is probably smaller than the number of incoming messages, since bbproxy merges messages before sending them. 

 

\item[Incoming - Combo messages] The number of ``combo'' messages received from a client. 

 

\item[Incoming - Status messages] The number of ``status'' messages received from a client. bbproxy attempts to merge these into ``combo'' messages. The ``Messages merged'' is the number of ``status'' messages that were merged into a combo message, the ``Resulting combos'' is the number of ``combo'' messages that resulted from the merging. 

 

\item[Incoming - Page messages] The number of ``page'' messages received from a client. These are discarded, they are generated by the old Big Brother clients, but have no meaning in Hobbit. 

 

\item[Incoming - Other messages] The number of other messages (data, notes, ack, query, ...) messages received from a client. 

 

\item[Proxy ressources - Connection table size] This is the number of connection table slots in the proxy. This measures the number of simultaneously active requests that the proxy has handled, and so gives an idea about the peak number of clients that the proxy has handled simultaneously. 

 

\item[Proxy ressources - Buffer space] This is the number of KB memory allocated for network buffers. 

 

\item[Timeout details - reading from client] The number of messages dropped because reading the message from the client timed out. 

 

\item[Timeout details - connecting to server] The number of messages dropped, because a connection to the Hobbit server could not be established. 

 

\item[Timeout details - sending to server] The number of messages dropped because the communication to the Hobbit server timed out after a connection was established. 

 

\item[Timeout details - recovered] When a timeout happens while sending the status message to the server, bbproxy will attempt to recover the message and retry sending it to the server after waiting a few seconds. This number is the number of messages that were recovered, and so were not lost. 

 

\item[Timeout details - reading from server] The number of response messages that timed out while attempting to read them from the server. Note that this applies to the ``config'' and ``query'' messages only, since all other message types do not get any response from the servers. 

 

\item[Timeout details - sending to client] The number of response messages that timed out while attempting to send them to the client. Note that this applies to the ``config'' and ``query'' messages only, since all other message types do not get any response from the servers. 

 

\item[Average queue time] The average time it took the proxy to process a message, calculated from the messages that have passed through the proxy during the past 5 minutes. This number is computed from the messages that actually end up establishing a connection to the Hobbit server, i.e. status messages that were combined into combo-messages do not go into the calculation - if they did, it would reduce the average time, since it is faster to merge messages than send them out over the network. 

 


\end{description}

\section{}
 If you think the numbers do not add up, here is how they relate. 

  The ``Incoming messages'' should be equal to the sum of the ``Incoming Combo/Status/Page/Other messages'', or slightly more because messages in transit are not included in the per-type message counts. 


  The ``Outbound messages'' should be equal to sum of the ``Incoming Combo/Page/Other messages'', plus the ``Resulting combos'' count, plus ``Incoming Status messages'' minus ``Messages merged'' (this latter number is the number of status messages that were NOT merged into combos, but sent directly). The ``Outbound messages'' may be slightly lower than that, because messages in transit are not included in the ``Outbound messages'' count until they have been fully sent. 


 
\section{SIGNALS}
\begin{description}
\item[SIGHUP] Re-opens the logfile, e.g. after it has been rotated. 

 

\item[SIGTERM] Shut down the proxy. 

 

\item[SIGUSR1] Toggles logging of individual messages. 

 


\end{description}
\section{SEE ALSO}
bb(1), hobbitd(8), hobbit(7) 


\section{Index}
\begin{description}
\item[NAME]
\item[SYNOPSIS]
\item[DESCRIPTION]
\item[OPTIONS]
\item[REPORT OUTPUT]
\item[]
\item[SIGNALS]
\item[SEE ALSO]

\end{description}




\newpage
\section{BBPROXY}

 bbproxy - xymon message proxy 
\subsection{SYNOPSIS}
\textbf{bbproxy [options] --servers=IP}

 
\subsection{DESCRIPTION}
\emph{bbproxy(8)}
 is a proxy for forwarding xymon messages from one server to another. It will typically be needed if you have clients behind a firewall, so they cannot send status messages to the Xymon server directly. 

 ~\cite{web:patchutils}hgcmd{bbproxy} serves three purposes. First, it acts as a regular proxy server, allowing clients that cannot connect directly to the xymon server to send messages to the Xymon servers. Although bbproxy is optimized for handling status messages, it will forward all types of messages.  



  Second, it acts as a buffer, smoothing out peak loads if many clients try to send status messages simultaneously. bbproxy can absorb messages very quickly, but will queue them up internally and forward them to the xymon server at a reasonable pace. This helps even out the load on your Xymon server.  



  Third, bbproxy merges small ``status'' messages into larger ``combo'' messages. This can dramatically decrease the number of connections that need to go from bbproxy to the xymon server, and is a slightly more efficient way of transmitting data to the Xymon server. The merging of messages causes ``status'' messages to be delayed for up to 0.25 seconds before being sent off to the Hobbit server. 


 
\subsection{OPTIONS}
\begin{description}
\item[--servers=SERVERIP[:PORT][,SERVER2IP[:PORT]]] Specifies the IP-address and optional portnumber where incoming messages are forwarded to. The default portnumber is 1984, the standard xymon port number. Up to 3 servers can be specified; incoming messages are sent to all of them (except ``config'', ``query'' and ``download'' messages, which go to the LAST server only). If you have Xymon clients sending their data via this proxy, note that the clients will receive their configuration data from the LAST of the servers listed here. This option is required. 

 

\item[--bbdisplay=SERVERIP[:PORT][,SERVER2IP[:PORT]]] Obsolete. Use ``--servers'' instead. 

 

\item[--listen=LOCALIP[:PORT]] Specifies the IP-adress where bbproxy listens for incoming connections. By default, bbproxy listens on all IP-adresses assigned to the host. If no portnumber is given, port 1984 will be used. 

 

\item[--timeout=N] Specifies the number of seconds after which a connection is aborted due to a timeout. Default: 10 seconds. 

 

\item[--report=[PROXYHOSTNAME.]PROXYSERVICE] If given, this option causes bbproxy to send a status report every 5 minutes to the xymon server about itself. If you have set the standard Xymon environment, you can use ``--report=bbproxy'' to have bbproxy report its status to a ``bbproxy'' column in Hobbit. The default for PROXYHOSTNAME is the \$MACHINE environment variable, i.e. the hostname of the server running bbproxy. See REPORT OUTPUT below for an explanation of the report contents. 

 

\item[--lqueue=N] Size of the listen-queue where incoming connections can queue up before being processed. This should be large to accomodate bursts of activity from clients. Default: 512. 

 

\item[--daemon] Run in daemon mode, i.e. detach and run as a background proces. This is the default. 

 

\item[--no-daemon] Runs bbproxy as a foreground proces. 

 

\item[--pidfile=FILENAME] Specifies the location of a file containing the proces-ID of the bbproxy daemon proces. Default: /var/run/bbproxy.pid. 

 

\item[--logfile=FILENAME] Sends all logging output to the specified file instead of stderr. 

 

\item[--log-details] Log details (IP-address, message type and hostname) to the logfile. This can also be enabled and disabled at run-time by sending the bbproxy proces a SIGUSR1 signal. 

 

\item[--debug] Enable debugging output. 

 


\end{description}
\subsection{REPORT OUTPUT}
 If enabled via the ``--report'' option, bbproxy will send a status message about itself to the xymon server once every 5 minutes. 

  The status message includes the following information: 


 \begin{description}
\item[Incoming messages] The total number of connections accepted from clients since the proxy started. The ``(N msgs/second)'' is the average number of messages per second over the past 5 minutes. 

 

\item[Outbound messages] The total number of messages sent to the xymon servers. Note that this is probably smaller than the number of incoming messages, since bbproxy merges messages before sending them. 

 

\item[Incoming - Combo messages] The number of ``combo'' messages received from a client. 

 

\item[Incoming - Status messages] The number of ``status'' messages received from a client. bbproxy attempts to merge these into ``combo'' messages. The ``Messages merged'' is the number of ``status'' messages that were merged into a combo message, the ``Resulting combos'' is the number of ``combo'' messages that resulted from the merging. 

 

\item[Incoming - Page messages] The number of ``page'' messages received from a client. These are discarded, they are generated by the old Big Brother clients, but have no meaning in xymon. 

 

\item[Incoming - Other messages] The number of other messages (data, notes, ack, query, ...) messages received from a client. 

 

\item[Proxy ressources - Connection table size] This is the number of connection table slots in the proxy. This measures the number of simultaneously active requests that the proxy has handled, and so gives an idea about the peak number of clients that the proxy has handled simultaneously. 

 

\item[Proxy ressources - Buffer space] This is the number of KB memory allocated for network buffers. 

 

\item[Timeout details - reading from client] The number of messages dropped because reading the message from the client timed out. 

 

\item[Timeout details - connecting to server] The number of messages dropped, because a connection to the xymon server could not be established. 

 

\item[Timeout details - sending to server] The number of messages dropped because the communication to the xymon server timed out after a connection was established. 

 

\item[Timeout details - recovered] When a timeout happens while sending the status message to the server, bbproxy will attempt to recover the message and retry sending it to the server after waiting a few seconds. This number is the number of messages that were recovered, and so were not lost. 

 

\item[Timeout details - reading from server] The number of response messages that timed out while attempting to read them from the server. Note that this applies to the ``config'' and ``query'' messages only, since all other message types do not get any response from the servers. 

 

\item[Timeout details - sending to client] The number of response messages that timed out while attempting to send them to the client. Note that this applies to the ``config'' and ``query'' messages only, since all other message types do not get any response from the servers. 

 

\item[Average queue time] The average time it took the proxy to process a message, calculated from the messages that have passed through the proxy during the past 5 minutes. This number is computed from the messages that actually end up establishing a connection to the xymon server, i.e. status messages that were combined into combo-messages do not go into the calculation - if they did, it would reduce the average time, since it is faster to merge messages than send them out over the network. 

 


\end{description}

\subsection{}
 If you think the numbers do not add up, here is how they relate. 

  The ``Incoming messages'' should be equal to the sum of the ``Incoming Combo/Status/Page/Other messages'', or slightly more because messages in transit are not included in the per-type message counts. 


  The ``Outbound messages'' should be equal to sum of the ``Incoming Combo/Page/Other messages'', plus the ``Resulting combos'' count, plus ``Incoming Status messages'' minus ``Messages merged'' (this latter number is the number of status messages that were NOT merged into combos, but sent directly). The ``Outbound messages'' may be slightly lower than that, because messages in transit are not included in the ``Outbound messages'' count until they have been fully sent. 


 
\subsection{SIGNALS}
\begin{description}
\item[SIGHUP] Re-opens the logfile, e.g. after it has been rotated. 

 

\item[SIGTERM] Shut down the proxy. 

 

\item[SIGUSR1] Toggles logging of individual messages. 

 


\end{description}
\subsection{SEE ALSO}
bb(1), xymond(8), hobbit(7) 



%%%%%%%%%%%%%%%%%%%%%%%%%%%%%%%%%%%%%%%%%%%%%%%%%%%%%%%%%%%%%%%%%%%%%%%%%%%%%%
%
%%%%%%%%%%%%%%%%%%%%%%%%%%%%%%%%%%%%%%%%%%%%%%%%%%%%%%%%%%%%%%%%%%%%%%%%%%%%%%
\newpage
\section{HOBBITD\_CLIENT}

 xymond\_client - hobbitd worker module for client data 
\subsection{SYNOPSIS}
\textbf{xymond\_channel --channel=client hobbitd\_client [options]}


 
\subsection{DESCRIPTION}
 xymond\_client is a worker module for hobbitd, and as such it is normally run via the \emph{hobbitd\_channel(8)}
 program. It receives xymond client messages sent from systems that have the the xymon client installed, and use the client data to generate the Xymon status messages for the cpu-, disk-, memory- and procs-columns. It also feeds Hobbit data messages with the netstat- and vmstat-data collected by the client. 

  When generating these status messages from the client data, xymond\_client will use the configuration rules defined in the \emph{hobbit-clients.cfg(5)}
 file to determine the color of each status message. 


 
\subsection{OPTIONS}
\begin{description}
\item[--clear-color=COLOR] Define the color used when sending ``msgs'', ``files'' or ``ports'' reports and there are no rules to check for these statuses. The default is to show a ``clear'' status, but some people prefer to have it ``green''. If you would rather prefer not to see these status columns at all, then you can use the ``--no-clear-msgs'', ``--no-clear-files'' and ``--no-clear-ports'' options instead. 

 

\item[--no-clear-msgs] If there are no logfile checks, the ``msgs'' column will show a ``clear'' status. If you would rather avoid having a ``msgs'' column, this option causes xymond\_client to not send in a clear ``msgs'' status. 

 

\item[--no-clear-files] If there are no file checks, the ``files'' column will show a ``clear'' status. If you would rather avoid having a ``files'' column, this option causes xymond\_client to not send in a clear ``files'' status. 

 

\item[--no-clear-ports] If there are no port checks, the ``ports'' column will show a ``clear'' status. If you would rather avoid having a ``ports'' column, this option causes xymond\_client to not send in a clear ``ports'' status. 

 

\item[--no-ps-listing] Normally the ``procs'' status message includes the full process-listing received from the client. If you prefer to just have the monitored processes shown, this option will turn off the full ps-listing. 

 

\item[--no-port-listing] Normally the ``ports'' status message includes the full netstat-listing received from the client. If you prefer to just have the monitored ports shown, this option will turn off the full netstat-listing. 

 

\item[--config=FILENAME] Sets the filename for the \emph{xymon-clients.cfg}
 file. The default value is ``etc/xymon-clients.cfg'' below the xymon server directory. 

 

\item[--dump-config] Dumps the configuration after parsing it. May be useful to track down problems with configuration file errors. 

 

\item[--test] Starts an interactive session where you can test the xymon-clients.cfg configuration. 

 

\item[--debug] Enable debugging output. 

 


\end{description}
\subsection{FILES}
\begin{description}
\item[~xymon/server/etc/hobbit-clients.cfg]

 


\end{description}
\subsection{SEE ALSO}
xymon-clients.cfg(5), hobbitd(8), hobbitd\_channel(8), hobbit(7) 

 
%%%%%%%%%%%%%%%%%%%%%%%%%%%%%%%%%%%%%%%%%%%%%%%%%%%%%%%%%%%%%%%%%%%%%%%%%%%%%%
%
%%%%%%%%%%%%%%%%%%%%%%%%%%%%%%%%%%%%%%%%%%%%%%%%%%%%%%%%%%%%%%%%%%%%%%%%%%%%%%
%\include{xymonweb.5}
\newpage
\section{HOBBITWEB}

 xymon web page headers, footers and forms. 

 
\subsection{DESCRIPTION}
 The xymon webpages are somewhat customizable, by modifying the header- and footer-templates found in the ~xymon/server/web/ directory. There are usually two or more files for a webpage: A \textbf{template\_header}
 file which is the header for this webpage, and a \textbf{template\_footer}
 file which is the footer. Webpages where entry forms are used have a \textbf{template\_form}
 file which is the data-entry form. 

  With the exception of the \textbf{bulletin}
 files, the header files are inserted into the HTML code at the very beginning and the footer files are inserted at the bottom. 


  The following templates are available: 


 \begin{description}
\item[bulletin] A \textbf{bulletin\_header}
 and \textbf{bulletin\_footer}
 is not shipped with xymon, but if they exist then the content of these files will be inserted in all HTML documents generated by Xymon. The ``bulletin\_header'' contents will appear after the normal header for the webpage, and the ``bulletin\_footer'' will appear just before the normal footer for the webpage. These files can be used to post important information about the Hobbit system, e.g. to notify users of current operational or monitoring problems. 

 

\item[acknowledge] Header, footer and form template for the xymon \textbf{acknowledge alert}
 webpage generated by \emph{bb-ack.cgi(1)}


 

\item[bb] Header and footer for the xymon \textbf{Main view}
 webpages, generated by \emph{bbgen(1)}


 

\item[bb2] Header and footer for the xymon \textbf{All non-green view}
 webpage, generated by \emph{bbgen(1)}


 

\item[bbnk] Header and footer for the now deprecated \textbf{BBNK}
 webpage, generated by bbgen. You should use the newer \emph{xymon-nkview.cgi(1)}
 utility instead, which uses the \textbf{xymonnk}
 templates. 

 

\item[bbrep] Header and footer for the xymon \textbf{Main view}
 availability report webpages, generated by \emph{bbgen(1)}
 when running in availability report mode. 

 

\item[bbsnap] Header and footer for the xymon \textbf{Main view}
 snapshot webpages, generated by \emph{bbgen(1)}
 when running in snapshot report mode. 

 

\item[bbsnap2] Header and footer for the xymon \textbf{All non-green view}
 snapshot webpage, generated by \emph{bbgen(1)}
 when running in snapshot report mode. 

 

\item[columndoc] Header and footer for the xymon \textbf{Column documentation}
 webpages, generated by the \emph{bb-csvinfo.cgi(1)}
 utility in the default xymon configuration. 

 

\item[confreport] Header and footer for the xymon \textbf{Configuration report}
 webpage, generated by the \emph{xymon-confreport.cgi(1)}
 utility. Note that there are also ``confreport\_front'' and ``confreport\_back'' templates, these are inserted into the generated report before the hostlist, and before the column documentation, respectively. 

 

\item[event] Header, footer and form for the xymon \textbf{Eventlog report}
, generated by \emph{xymon-eventlog.cgi(1)}


 

\item[findhost] Header, footer and form for the xymon \textbf{Find host}
 webpage, generated by \emph{bb-findhost.cgi(1)}


 

\item[graphs] Header and footer for the xymon \textbf{Graph details}
 webpages, generated by \emph{xymongraph.cgi(1)}


 

\item[hist] Header and footer for the xymon \textbf{History}
 webpage, generated by \emph{bb-hist.cgi(1)}


 

\item[histlog] Header and footer for the xymon \textbf{Historical status-log}
 webpage, generated by \emph{xymonsvc.cgi(1)}
 utility when used to show a historical (non-current) status log. 

 

\item[xymonnk] Header and footer for the xymon \textbf{Critical Systems view}
 webpage, generated by \emph{xymon-nkview.cgi(1)}


 

\item[hostsvc] Header and footer for the xymon \textbf{Status-log}
 webpage, generated by \emph{xymonsvc.cgi(1)}
 utility when used to show a current status log. 

 

\item[info] Header and footer for the xymon \textbf{Info column}
 webpage, generated by \emph{xymonsvc.cgi(1)}
 utility when used to show the host configuration page. 

 

\item[maintact] Header and footer for the xymon \textbf{}
 webpage, generated by \emph{xymon-enadis.cgi(1)}
 utility when using the Enable/Disable ``preview'' mode. 

 

\item[maint] Header, footer and form for the xymon \textbf{Enable/disable}
 webpage, generated by \emph{xymon-enadis.cgi(1)}


 

\item[nkack] Form show on the \textbf{status-log}
 webpage when viewed from the ``Critical Systems'' overview. This form is used to acknowledge a critical status by the operators monitoring the Critical Systems view. 

 

\item[nkedit] Header, footer and form for the \textbf{Critical Systems Editor}
, the \emph{xymon-nkedit.cgi(1)}
 utility. 

 

\item[replog] Header and footer for the xymon \textbf{Report status-log}
 webpage, generated by \emph{xymonsvc.cgi(1)}
 utility when used to show a status log for an availability report. 

 

\item[report] Header, footer and forms for selecting a pre-generated \textbf{Availability Report}
. Handled by the \emph{bb-datepage.cgi(1)}
 utility. 

 

\item[snapshot] Header and footer for the xymon \textbf{Snapshot report}
 selection webpage, generated by \emph{bb-snapshot.cgi(1)}


 


\end{description}

\subsection{SEE ALSO}
bbgen(1), xymonsvc.cgi(1), hobbit(7) 
