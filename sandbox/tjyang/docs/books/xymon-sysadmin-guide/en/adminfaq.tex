\chapter{Administration FAQ}
\label{chap:adminfaq}

Mercurial offers a powerful mechanism to let you perform automated
actions in response to events that occur in a repository.  In some
cases, you can even control Mercurial's response to those events.

The name Mercurial uses for one of these actions is a \emph{hook}.
Hooks are called ``triggers'' in some revision control systems, but
the two names refer to the same idea.

\section{An overview of hooks in Mercurial}

Here is a brief list of the hooks that Mercurial supports.  We will
revisit each of these hooks in more detail later, in
section~\ref{sec:hook:ref}.

\begin{itemize}
\item[\small\hook{changegroup}] This is run after a group of
  changesets has been brought into the repository from elsewhere.
\item[\small\hook{commit}] This is run after a new changeset has been
  created in the local repository.
\item[\small\hook{incoming}] This is run once for each new changeset
  that is brought into the repository from elsewhere.  Notice the
  difference from \hook{changegroup}, which is run once per
  \emph{group} of changesets brought in.
\item[\small\hook{outgoing}] This is run after a group of changesets
  has been transmitted from this repository.
\item[\small\hook{prechangegroup}] This is run before starting to
  bring a group of changesets into the repository.
\item[\small\hook{precommit}] Controlling. This is run before starting
  a commit.
\item[\small\hook{preoutgoing}] Controlling. This is run before
  starting to transmit a group of changesets from this repository.
\item[\small\hook{pretag}] Controlling. This is run before creating a tag.
\item[\small\hook{pretxnchangegroup}] Controlling. This is run after a
  group of changesets has been brought into the local repository from
  another, but before the transaction completes that will make the
  changes permanent in the repository.
\item[\small\hook{pretxncommit}] Controlling. This is run after a new
  changeset has been created in the local repository, but before the
  transaction completes that will make it permanent.
\item[\small\hook{preupdate}] Controlling. This is run before starting
  an update or merge of the working directory.
\item[\small\hook{tag}] This is run after a tag is created.
\item[\small\hook{update}] This is run after an update or merge of the
  working directory has finished.
\end{itemize}
Each of the hooks whose description begins with the word
``Controlling'' has the ability to determine whether an activity can
proceed.  If the hook succeeds, the activity may proceed; if it fails,
the activity is either not permitted or undone, depending on the hook.

\section{Hooks and security}

\section{Hooks are run with your privileges}

When you run a Mercurial command in a repository, and the command
causes a hook to run, that hook runs on \emph{your} system, under
\emph{your} user account, with \emph{your} privilege level.  Since
hooks are arbitrary pieces of executable code, you should treat them
with an appropriate level of suspicion.  Do not install a hook unless
you are confident that you know who created it and what it does.

In some cases, you may be exposed to hooks that you did not install
yourself.  If you work with Mercurial on an unfamiliar system,
Mercurial will run hooks defined in that system's global \hgrc\ file.

If you are working with a repository owned by another user, Mercurial
can run hooks defined in that user's repository, but it will still run
them as ``you''.  For example, if you \hgcmd{pull} from that
repository, and its \sfilename{.hg/hgrc} defines a local
\hook{outgoing} hook, that hook will run under your user account, even
though you don't own that repository.

\begin{note}
  This only applies if you are pulling from a repository on a local or
  network filesystem.  If you're pulling over http or ssh, any
  \hook{outgoing} hook will run under whatever account is executing
  the server process, on the server.
\end{note}

XXX To see what hooks are defined in a repository, use the
\hgcmdargs{config}{hooks} command.  If you are working in one
repository, but talking to another that you do not own (e.g.~using
\hgcmd{pull} or \hgcmd{incoming}), remember that it is the other
repository's hooks you should be checking, not your own.

\section{Hooks do not propagate}

In Mercurial, hooks are not revision controlled, and do not propagate
when you clone, or pull from, a repository.  The reason for this is
simple: a hook is a completely arbitrary piece of executable code.  It
runs under your user identity, with your privilege level, on your
machine.

It would be extremely reckless for any distributed revision control
system to implement revision-controlled hooks, as this would offer an
easily exploitable way to subvert the accounts of users of the
revision control system.

Since Mercurial does not propagate hooks, if you are collaborating
with other people on a common project, you should not assume that they
are using the same Mercurial hooks as you are, or that theirs are
correctly configured.  You should document the hooks you expect people
to use.

In a corporate intranet, this is somewhat easier to control, as you
can for example provide a ``standard'' installation of Mercurial on an
NFS filesystem, and use a site-wide \hgrc\ file to define hooks that
all users will see.  However, this too has its limits; see below.

\section{Hooks can be overridden}

Mercurial allows you to override a hook definition by redefining the
hook.  You can disable it by setting its value to the empty string, or
change its behaviour as you wish.

If you deploy a system-~or site-wide \hgrc\ file that defines some
hooks, you should thus understand that your users can disable or
override those hooks.

\section{Ensuring that critical hooks are run}

Sometimes you may want to enforce a policy that you do not want others
to be able to work around.  For example, you may have a requirement
that every changeset must pass a rigorous set of tests.  Defining this
requirement via a hook in a site-wide \hgrc\ won't work for remote
users on laptops, and of course local users can subvert it at will by
overriding the hook.

Instead, you can set up your policies for use of Mercurial so that
people are expected to propagate changes through a well-known
``canonical'' server that you have locked down and configured
appropriately.

One way to do this is via a combination of social engineering and
technology.  Set up a restricted-access account; users can push
changes over the network to repositories managed by this account, but
they cannot log into the account and run normal shell commands.  In
this scenario, a user can commit a changeset that contains any old
garbage they want.

When someone pushes a changeset to the server that everyone pulls
from, the server will test the changeset before it accepts it as
permanent, and reject it if it fails to pass the test suite.  If
people only pull changes from this filtering server, it will serve to
ensure that all changes that people pull have been automatically
vetted.

\section{Care with \texttt{pretxn} hooks in a shared-access repository}

If you want to use hooks to do some automated work in a repository
that a number of people have shared access to, you need to be careful
in how you do this.

Mercurial only locks a repository when it is writing to the
repository, and only the parts of Mercurial that write to the
repository pay attention to locks.  Write locks are necessary to
prevent multiple simultaneous writers from scribbling on each other's
work, corrupting the repository.

Because Mercurial is careful with the order in which it reads and
writes data, it does not need to acquire a lock when it wants to read
data from the repository.  The parts of Mercurial that read from the
repository never pay attention to locks.  This lockless reading scheme
greatly increases performance and concurrency.

With great performance comes a trade-off, though, one which has the
potential to cause you trouble unless you're aware of it.  To describe
this requires a little detail about how Mercurial adds changesets to a
repository and reads those changes.

When Mercurial \emph{writes} metadata, it writes it straight into the
destination file.  It writes file data first, then manifest data
(which contains pointers to the new file data), then changelog data
(which contains pointers to the new manifest data).  Before the first
write to each file, it stores a record of where the end of the file
was in its transaction log.  If the transaction must be rolled back,
Mercurial simply truncates each file back to the size it was before the
transaction began.

When Mercurial \emph{reads} metadata, it reads the changelog first,
then everything else.  Since a reader will only access parts of the
manifest or file metadata that it can see in the changelog, it can
never see partially written data.

Some controlling hooks (\hook{pretxncommit} and
\hook{pretxnchangegroup}) run when a transaction is almost complete.
All of the metadata has been written, but Mercurial can still roll the
transaction back and cause the newly-written data to disappear.

If one of these hooks runs for long, it opens a window of time during
which a reader can see the metadata for changesets that are not yet
permanent, and should not be thought of as ``really there''.  The
longer the hook runs, the longer that window is open.

\section{The problem illustrated}

In principle, a good use for the \hook{pretxnchangegroup} hook would
be to automatically build and test incoming changes before they are
accepted into a central repository.  This could let you guarantee that
nobody can push changes to this repository that ``break the build''.
But if a client can pull changes while they're being tested, the
usefulness of the test is zero; an unsuspecting someone can pull
untested changes, potentially breaking their build.

The safest technological answer to this challenge is to set up such a
``gatekeeper'' repository as \emph{unidirectional}.  Let it take
changes pushed in from the outside, but do not allow anyone to pull
changes from it (use the \hook{preoutgoing} hook to lock it down).
Configure a \hook{changegroup} hook so that if a build or test
succeeds, the hook will push the new changes out to another repository
that people \emph{can} pull from.

In practice, putting a centralised bottleneck like this in place is
not often a good idea, and transaction visibility has nothing to do
with the problem.  As the size of a project---and the time it takes to
build and test---grows, you rapidly run into a wall with this ``try
before you buy'' approach, where you have more changesets to test than
time in which to deal with them.  The inevitable result is frustration
on the part of all involved.

An approach that scales better is to get people to build and test
before they push, then run automated builds and tests centrally
\emph{after} a push, to be sure all is well.  The advantage of this
approach is that it does not impose a limit on the rate at which the
repository can accept changes.

\section{A short tutorial on using hooks}
\label{sec:hook:simple}

It is easy to write a Mercurial hook.  Let's start with a hook that
runs when you finish a \hgcmd{commit}, and simply prints the hash of
the changeset you just created.  The hook is called \hook{commit}.

\begin{figure}[ht]
  \interaction{hook.simple.init}
  \caption{A simple hook that runs when a changeset is committed}
  \label{ex:hook:init}
\end{figure}

All hooks follow the pattern in example~\ref{ex:hook:init}.  You add
an entry to the \rcsection{hooks} section of your \hgrc\.  On the left
is the name of the event to trigger on; on the right is the action to
take.  As you can see, you can run an arbitrary shell command in a
hook.  Mercurial passes extra information to the hook using
environment variables (look for \envar{HG\_NODE} in the example).

\section{Performing multiple actions per event}

Quite often, you will want to define more than one hook for a
particular kind of event, as shown in example~\ref{ex:hook:ext}.
Mercurial lets you do this by adding an \emph{extension} to the end of
a hook's name.  You extend a hook's name by giving the name of the
hook, followed by a full stop (the ``\texttt{.}'' character), followed
by some more text of your choosing.  For example, Mercurial will run
both \texttt{commit.foo} and \texttt{commit.bar} when the
\texttt{commit} event occurs.

\begin{figure}[ht]
  \interaction{hook.simple.ext}
  \caption{Defining a second \hook{commit} hook}
  \label{ex:hook:ext}
\end{figure}

To give a well-defined order of execution when there are multiple
hooks defined for an event, Mercurial sorts hooks by extension, and
executes the hook commands in this sorted order.  In the above
example, it will execute \texttt{commit.bar} before
\texttt{commit.foo}, and \texttt{commit} before both.

It is a good idea to use a somewhat descriptive extension when you
define a new hook.  This will help you to remember what the hook was
for.  If the hook fails, you'll get an error message that contains the
hook name and extension, so using a descriptive extension could give
you an immediate hint as to why the hook failed (see
section~\ref{sec:hook:perm} for an example).

\section{Controlling whether an activity can proceed}
\label{sec:hook:perm}

In our earlier examples, we used the \hook{commit} hook, which is
run after a commit has completed.  This is one of several Mercurial
hooks that run after an activity finishes.  Such hooks have no way of
influencing the activity itself.

Mercurial defines a number of events that occur before an activity
starts; or after it starts, but before it finishes.  Hooks that
trigger on these events have the added ability to choose whether the
activity can continue, or will abort.  

The \hook{pretxncommit} hook runs after a commit has all but
completed.  In other words, the metadata representing the changeset
has been written out to disk, but the transaction has not yet been
allowed to complete.  The \hook{pretxncommit} hook has the ability to
decide whether the transaction can complete, or must be rolled back.

If the \hook{pretxncommit} hook exits with a status code of zero, the
transaction is allowed to complete; the commit finishes; and the
\hook{commit} hook is run.  If the \hook{pretxncommit} hook exits with
a non-zero status code, the transaction is rolled back; the metadata
representing the changeset is erased; and the \hook{commit} hook is
not run.

\begin{figure}[ht]
  \interaction{hook.simple.pretxncommit}
  \caption{Using the \hook{pretxncommit} hook to control commits}
  \label{ex:hook:pretxncommit}
\end{figure}

The hook in example~\ref{ex:hook:pretxncommit} checks that a commit
comment contains a bug ID.  If it does, the commit can complete.  If
not, the commit is rolled back.

\section{Writing your own hooks}

When you are writing a hook, you might find it useful to run Mercurial
either with the \hggopt{-v} option, or the \rcitem{ui}{verbose} config
item set to ``true''.  When you do so, Mercurial will print a message
before it calls each hook.

\section{Choosing how your hook should run}
\label{sec:hook:lang}

You can write a hook either as a normal program---typically a shell
script---or as a Python function that is executed within the Mercurial
process.

Writing a hook as an external program has the advantage that it
requires no knowledge of Mercurial's internals.  You can call normal
Mercurial commands to get any added information you need.  The
trade-off is that external hooks are slower than in-process hooks.

An in-process Python hook has complete access to the Mercurial API,
and does not ``shell out'' to another process, so it is inherently
faster than an external hook.  It is also easier to obtain much of the
information that a hook requires by using the Mercurial API than by
running Mercurial commands.

If you are comfortable with Python, or require high performance,
writing your hooks in Python may be a good choice.  However, when you
have a straightforward hook to write and you don't need to care about
performance (probably the majority of hooks), a shell script is
perfectly fine.

\section{Hook parameters}
\label{sec:hook:param}

Mercurial calls each hook with a set of well-defined parameters.  In
Python, a parameter is passed as a keyword argument to your hook
function.  For an external program, a parameter is passed as an
environment variable.

Whether your hook is written in Python or as a shell script, the
hook-specific parameter names and values will be the same.  A boolean
parameter will be represented as a boolean value in Python, but as the
number 1 (for ``true'') or 0 (for ``false'') as an environment
variable for an external hook.  If a hook parameter is named
\texttt{foo}, the keyword argument for a Python hook will also be
named \texttt{foo}, while the environment variable for an external
hook will be named \texttt{HG\_FOO}.

\section{Hook return values and activity control}

A hook that executes successfully must exit with a status of zero if
external, or return boolean ``false'' if in-process.  Failure is
indicated with a non-zero exit status from an external hook, or an
in-process hook returning boolean ``true''.  If an in-process hook
raises an exception, the hook is considered to have failed.

For a hook that controls whether an activity can proceed, zero/false
means ``allow'', while non-zero/true/exception means ``deny''.

\section{Writing an external hook}

When you define an external hook in your \hgrc\ and the hook is run,
its value is passed to your shell, which interprets it.  This means
that you can use normal shell constructs in the body of the hook.

An executable hook is always run with its current directory set to a
repository's root directory.

Each hook parameter is passed in as an environment variable; the name
is upper-cased, and prefixed with the string ``\texttt{HG\_}''.

With the exception of hook parameters, Mercurial does not set or
modify any environment variables when running a hook.  This is useful
to remember if you are writing a site-wide hook that may be run by a
number of different users with differing environment variables set.
In multi-user situations, you should not rely on environment variables
being set to the values you have in your environment when testing the
hook.

\section{Telling Mercurial to use an in-process hook}

The \hgrc\ syntax for defining an in-process hook is slightly
different than for an executable hook.  The value of the hook must
start with the text ``\texttt{python:}'', and continue with the
fully-qualified name of a callable object to use as the hook's value.

The module in which a hook lives is automatically imported when a hook
is run.  So long as you have the module name and \envar{PYTHONPATH}
right, it should ``just work''.

The following \hgrc\ example snippet illustrates the syntax and
meaning of the notions we just described.
\begin{codesample2}
  [hooks]
  commit.example = python:mymodule.submodule.myhook
\end{codesample2}
When Mercurial runs the \texttt{commit.example} hook, it imports
\texttt{mymodule.submodule}, looks for the callable object named
\texttt{myhook}, and calls it.

\section{Writing an in-process hook}

The simplest in-process hook does nothing, but illustrates the basic
shape of the hook API:
\begin{codesample2}
  def myhook(ui, repo, **kwargs):
      pass
\end{codesample2}
The first argument to a Python hook is always a
\pymodclass{mercurial.ui}{ui} object.  The second is a repository object;
at the moment, it is always an instance of
\pymodclass{mercurial.localrepo}{localrepository}.  Following these two
arguments are other keyword arguments.  Which ones are passed in
depends on the hook being called, but a hook can ignore arguments it
doesn't care about by dropping them into a keyword argument dict, as
with \texttt{**kwargs} above.

\section{Some hook examples}

\section{Writing meaningful commit messages}

It's hard to imagine a useful commit message being very short.  The
simple \hook{pretxncommit} hook of figure~\ref{ex:hook:msglen.go}
will prevent you from committing a changeset with a message that is
less than ten bytes long.

\begin{figure}[ht]
  \interaction{hook.msglen.go}
  \caption{A hook that forbids overly short commit messages}
  \label{ex:hook:msglen.go}
\end{figure}

\section{Checking for trailing whitespace}

An interesting use of a commit-related hook is to help you to write
cleaner code.  A simple example of ``cleaner code'' is the dictum that
a change should not add any new lines of text that contain ``trailing
whitespace''.  Trailing whitespace is a series of space and tab
characters at the end of a line of text.  In most cases, trailing
whitespace is unnecessary, invisible noise, but it is occasionally
problematic, and people often prefer to get rid of it.

You can use either the \hook{precommit} or \hook{pretxncommit} hook to
tell whether you have a trailing whitespace problem.  If you use the
\hook{precommit} hook, the hook will not know which files you are
committing, so it will have to check every modified file in the
repository for trailing white space.  If you want to commit a change
to just the file \filename{foo}, but the file \filename{bar} contains
trailing whitespace, doing a check in the \hook{precommit} hook will
prevent you from committing \filename{foo} due to the problem with
\filename{bar}.  This doesn't seem right.

Should you choose the \hook{pretxncommit} hook, the check won't occur
until just before the transaction for the commit completes.  This will
allow you to check for problems only the exact files that are being
committed.  However, if you entered the commit message interactively
and the hook fails, the transaction will roll back; you'll have to
re-enter the commit message after you fix the trailing whitespace and
run \hgcmd{commit} again.

\begin{figure}[ht]
  \interaction{hook.ws.simple}
  \caption{A simple hook that checks for trailing whitespace}
  \label{ex:hook:ws.simple}
\end{figure}

Figure~\ref{ex:hook:ws.simple} introduces a simple \hook{pretxncommit}
hook that checks for trailing whitespace.  This hook is short, but not
very helpful.  It exits with an error status if a change adds a line
with trailing whitespace to any file, but does not print any
information that might help us to identify the offending file or
line.  It also has the nice property of not paying attention to
unmodified lines; only lines that introduce new trailing whitespace
cause problems.

\begin{figure}[ht]
  \interaction{hook.ws.better}
  \caption{A better trailing whitespace hook}
  \label{ex:hook:ws.better}
\end{figure}

The example of figure~\ref{ex:hook:ws.better} is much more complex,
but also more useful.  It parses a unified diff to see if any lines
add trailing whitespace, and prints the name of the file and the line
number of each such occurrence.  Even better, if the change adds
trailing whitespace, this hook saves the commit comment and prints the
name of the save file before exiting and telling Mercurial to roll the
transaction back, so you can use
\hgcmdargs{commit}{\hgopt{commit}{-l}~\emph{filename}} to reuse the
saved commit message once you've corrected the problem.

As a final aside, note in figure~\ref{ex:hook:ws.better} the use of
\command{perl}'s in-place editing feature to get rid of trailing
whitespace from a file.  This is concise and useful enough that I will
reproduce it here.
\begin{codesample2}
  perl -pi -e 's,s+\$,,' filename
\end{codesample2}

\section{Bundled hooks}

Mercurial ships with several bundled hooks.  You can find them in the
\dirname{hgext} directory of a Mercurial source tree.  If you are
using a Mercurial binary package, the hooks will be located in the
\dirname{hgext} directory of wherever your package installer put
Mercurial.

\section{\hgext{acl}---access control for parts of a repository}

The \hgext{acl} extension lets you control which remote users are
allowed to push changesets to a networked server.  You can protect any
portion of a repository (including the entire repo), so that a
specific remote user can push changes that do not affect the protected
portion.

This extension implements access control based on the identity of the
user performing a push, \emph{not} on who committed the changesets
they're pushing.  It makes sense to use this hook only if you have a
locked-down server environment that authenticates remote users, and
you want to be sure that only specific users are allowed to push
changes to that server.

\subsubsection{Configuring the \hook{acl} hook}

In order to manage incoming changesets, the \hgext{acl} hook must be
used as a \hook{pretxnchangegroup} hook.  This lets it see which files
are modified by each incoming changeset, and roll back a group of
changesets if they modify ``forbidden'' files.  Example:
\begin{codesample2}
  [hooks]
  pretxnchangegroup.acl = python:hgext.acl.hook
\end{codesample2}

The \hgext{acl} extension is configured using three sections.  

The \rcsection{acl} section has only one entry, \rcitem{acl}{sources},
which lists the sources of incoming changesets that the hook should
pay attention to.  You don't normally need to configure this section.
\begin{itemize}
\item[\rcitem{acl}{serve}] Control incoming changesets that are arriving
  from a remote repository over http or ssh.  This is the default
  value of \rcitem{acl}{sources}, and usually the only setting you'll
  need for this configuration item.
\item[\rcitem{acl}{pull}] Control incoming changesets that are
  arriving via a pull from a local repository.
\item[\rcitem{acl}{push}] Control incoming changesets that are
  arriving via a push from a local repository.
\item[\rcitem{acl}{bundle}] Control incoming changesets that are
  arriving from another repository via a bundle.
\end{itemize}

The \rcsection{acl.allow} section controls the users that are allowed to
add changesets to the repository.  If this section is not present, all
users that are not explicitly denied are allowed.  If this section is
present, all users that are not explicitly allowed are denied (so an
empty section means that all users are denied).

The \rcsection{acl.deny} section determines which users are denied
from adding changesets to the repository.  If this section is not
present or is empty, no users are denied.

The syntaxes for the \rcsection{acl.allow} and \rcsection{acl.deny}
sections are identical.  On the left of each entry is a glob pattern
that matches files or directories, relative to the root of the
repository; on the right, a user name.

In the following example, the user \texttt{docwriter} can only push
changes to the \dirname{docs} subtree of the repository, while
\texttt{intern} can push changes to any file or directory except
\dirname{source/sensitive}.
\begin{codesample2}
  [acl.allow]
  docs/** = docwriter

  [acl.deny]
  source/sensitive/** = intern
\end{codesample2}

\subsubsection{Testing and troubleshooting}

If you want to test the \hgext{acl} hook, run it with Mercurial's
debugging output enabled.  Since you'll probably be running it on a
server where it's not convenient (or sometimes possible) to pass in
the \hggopt{--debug} option, don't forget that you can enable
debugging output in your \hgrc:
\begin{codesample2}
  [ui]
  debug = true
\end{codesample2}
With this enabled, the \hgext{acl} hook will print enough information
to let you figure out why it is allowing or forbidding pushes from
specific users.

\section{\hgext{bugzilla}---integration with Bugzilla}

The \hgext{bugzilla} extension adds a comment to a Bugzilla bug
whenever it finds a reference to that bug ID in a commit comment.  You
can install this hook on a shared server, so that any time a remote
user pushes changes to this server, the hook gets run.  

It adds a comment to the bug that looks like this (you can configure
the contents of the comment---see below):
\begin{codesample2}
  Changeset aad8b264143a, made by Joe User <joe.user@domain.com> in
  the frobnitz repository, refers to this bug.

  For complete details, see
  http://hg.domain.com/frobnitz?cmd=changeset;node=aad8b264143a

  Changeset description:
        Fix bug 10483 by guarding against some NULL pointers
\end{codesample2}
The value of this hook is that it automates the process of updating a
bug any time a changeset refers to it.  If you configure the hook
properly, it makes it easy for people to browse straight from a
Bugzilla bug to a changeset that refers to that bug.

You can use the code in this hook as a starting point for some more
exotic Bugzilla integration recipes.  Here are a few possibilities:
\begin{itemize}
\item Require that every changeset pushed to the server have a valid
  bug~ID in its commit comment.  In this case, you'd want to configure
  the hook as a \hook{pretxncommit} hook.  This would allow the hook
  to reject changes that didn't contain bug IDs.
\item Allow incoming changesets to automatically modify the
  \emph{state} of a bug, as well as simply adding a comment.  For
  example, the hook could recognise the string ``fixed bug 31337'' as
  indicating that it should update the state of bug 31337 to
  ``requires testing''.
\end{itemize}

\subsubsection{Configuring the \hook{bugzilla} hook}
\label{sec:hook:bugzilla:config}

You should configure this hook in your server's \hgrc\ as an
\hook{incoming} hook, for example as follows:
\begin{codesample2}
  [hooks]
  incoming.bugzilla = python:hgext.bugzilla.hook
\end{codesample2}

Because of the specialised nature of this hook, and because Bugzilla
was not written with this kind of integration in mind, configuring
this hook is a somewhat involved process.

Before you begin, you must install the MySQL bindings for Python on
the host(s) where you'll be running the hook.  If this is not
available as a binary package for your system, you can download it
from~\cite{web:wikibook-xymon}.

Configuration information for this hook lives in the
\rcsection{bugzilla} section of your \hgrc.
\begin{itemize}
\item[\rcitem{bugzilla}{version}] The version of Bugzilla installed on
  the server.  The database schema that Bugzilla uses changes
  occasionally, so this hook has to know exactly which schema to use.
  At the moment, the only version supported is \texttt{2.16}.
\item[\rcitem{bugzilla}{host}] The hostname of the MySQL server that
  stores your Bugzilla data.  The database must be configured to allow
  connections from whatever host you are running the \hook{bugzilla}
  hook on.
\item[\rcitem{bugzilla}{user}] The username with which to connect to
  the MySQL server.  The database must be configured to allow this
  user to connect from whatever host you are running the
  \hook{bugzilla} hook on.  This user must be able to access and
  modify Bugzilla tables.  The default value of this item is
  \texttt{bugs}, which is the standard name of the Bugzilla user in a
  MySQL database.
\item[\rcitem{bugzilla}{password}] The MySQL password for the user you
  configured above.  This is stored as plain text, so you should make
  sure that unauthorised users cannot read the \hgrc\ file where you
  store this information.
\item[\rcitem{bugzilla}{db}] The name of the Bugzilla database on the
  MySQL server.  The default value of this item is \texttt{bugs},
  which is the standard name of the MySQL database where Bugzilla
  stores its data.
\item[\rcitem{bugzilla}{notify}] If you want Bugzilla to send out a
  notification email to subscribers after this hook has added a
  comment to a bug, you will need this hook to run a command whenever
  it updates the database.  The command to run depends on where you
  have installed Bugzilla, but it will typically look something like
  this, if you have Bugzilla installed in
  \dirname{/var/www/html/bugzilla}:
  \begin{codesample4}
    cd /var/www/html/bugzilla && ./processmail %s nobody@nowhere.com
  \end{codesample4}
  The Bugzilla \texttt{processmail} program expects to be given a
  bug~ID (the hook replaces ``\texttt{\%s}'' with the bug~ID) and an
  email address.  It also expects to be able to write to some files in
  the directory that it runs in.  If Bugzilla and this hook are not
  installed on the same machine, you will need to find a way to run
  \texttt{processmail} on the server where Bugzilla is installed.
\end{itemize}

\subsubsection{Mapping committer names to Bugzilla user names}

By default, the \hgext{bugzilla} hook tries to use the email address
of a changeset's committer as the Bugzilla user name with which to
update a bug.  If this does not suit your needs, you can map committer
email addresses to Bugzilla user names using a \rcsection{usermap}
section.

Each item in the \rcsection{usermap} section contains an email address
on the left, and a Bugzilla user name on the right.
\begin{codesample2}
  [usermap]
  jane.user@example.com = jane
\end{codesample2}
You can either keep the \rcsection{usermap} data in a normal \hgrc, or
tell the \hgext{bugzilla} hook to read the information from an
external \filename{usermap} file.  In the latter case, you can store
\filename{usermap} data by itself in (for example) a user-modifiable
repository.  This makes it possible to let your users maintain their
own \rcitem{bugzilla}{usermap} entries.  The main \hgrc\ file might
look like this:
\begin{codesample2}
  # regular hgrc file refers to external usermap file
  [bugzilla]
  usermap = /home/hg/repos/userdata/bugzilla-usermap.conf
\end{codesample2}
While the \filename{usermap} file that it refers to might look like
this:
\begin{codesample2}
  # bugzilla-usermap.conf - inside a hg repository
  [usermap]
  stephanie@example.com = steph
\end{codesample2}

\subsubsection{Configuring the text that gets added to a bug}

You can configure the text that this hook adds as a comment; you
specify it in the form of a Mercurial template.  Several \hgrc\
entries (still in the \rcsection{bugzilla} section) control this
behaviour.
\begin{itemize}
\item[\texttt{strip}] The number of leading path elements to strip
  from a repository's path name to construct a partial path for a URL.
  For example, if the repositories on your server live under
  \dirname{/home/hg/repos}, and you have a repository whose path is
  \dirname{/home/hg/repos/app/tests}, then setting \texttt{strip} to
  \texttt{4} will give a partial path of \dirname{app/tests}.  The
  hook will make this partial path available when expanding a
  template, as \texttt{webroot}.
\item[\texttt{template}] The text of the template to use.  In addition
  to the usual changeset-related variables, this template can use
  \texttt{hgweb} (the value of the \texttt{hgweb} configuration item
  above) and \texttt{webroot} (the path constructed using
  \texttt{strip} above).
\end{itemize}

In addition, you can add a \rcitem{web}{baseurl} item to the
\rcsection{web} section of your \hgrc.  The \hgext{bugzilla} hook will
make this available when expanding a template, as the base string to
use when constructing a URL that will let users browse from a Bugzilla
comment to view a changeset.  Example:
\begin{codesample2}
  [web]
  baseurl = http://hg.domain.com/
\end{codesample2}

Here is an example set of \hgext{bugzilla} hook config information.
\begin{codesample2}
  [bugzilla]
  host = bugzilla.example.com
  password = mypassword
  version = 2.16
  # server-side repos live in /home/hg/repos, so strip 4 leading
  # separators
  strip = 4
  hgweb = http://hg.example.com/
  usermap = /home/hg/repos/notify/bugzilla.conf
  template = Changeset \{node|short\}, made by \{author\} in the \{webroot\}
    repo, refers to this bug.nFor complete details, see 
    \{hgweb\}\{webroot\}?cmd=changeset;node=\{node|short\}nChangeset
    description:n\\t\{desc|tabindent\}
\end{codesample2}

\subsubsection{Testing and troubleshooting}

The most common problems with configuring the \hgext{bugzilla} hook
relate to running Bugzilla's \filename{processmail} script and mapping
committer names to user names.

Recall from section~\ref{sec:hook:bugzilla:config} above that the user
that runs the Mercurial process on the server is also the one that
will run the \filename{processmail} script.  The
\filename{processmail} script sometimes causes Bugzilla to write to
files in its configuration directory, and Bugzilla's configuration
files are usually owned by the user that your web server runs under.

You can cause \filename{processmail} to be run with the suitable
user's identity using the \command{sudo} command.  Here is an example
entry for a \filename{sudoers} file.
\begin{codesample2}
  hg_user = (httpd_user) NOPASSWD: /var/www/html/bugzilla/processmail-wrapper %s
\end{codesample2}
This allows the \texttt{hg\_user} user to run a
\filename{processmail-wrapper} program under the identity of
\texttt{httpd\_user}.

This indirection through a wrapper script is necessary, because
\filename{processmail} expects to be run with its current directory
set to wherever you installed Bugzilla; you can't specify that kind of
constraint in a \filename{sudoers} file.  The contents of the wrapper
script are simple:
\begin{codesample2}
  #!/bin/sh
  cd `dirname $0` && ./processmail "$1" nobody@example.com
\end{codesample2}
It doesn't seem to matter what email address you pass to
\filename{processmail}.

If your \rcsection{usermap} is not set up correctly, users will see an
error message from the \hgext{bugzilla} hook when they push changes
to the server.  The error message will look like this:
\begin{codesample2}
  cannot find bugzilla user id for john.q.public@example.com
\end{codesample2}
What this means is that the committer's address,
\texttt{john.q.public@example.com}, is not a valid Bugzilla user name,
nor does it have an entry in your \rcsection{usermap} that maps it to
a valid Bugzilla user name.

\section{\hgext{notify}---send email notifications}

Although Mercurial's built-in web server provides RSS feeds of changes
in every repository, many people prefer to receive change
notifications via email.  The \hgext{notify} hook lets you send out
notifications to a set of email addresses whenever changesets arrive
that those subscribers are interested in.

As with the \hgext{bugzilla} hook, the \hgext{notify} hook is
template-driven, so you can customise the contents of the notification
messages that it sends.

By default, the \hgext{notify} hook includes a diff of every changeset
that it sends out; you can limit the size of the diff, or turn this
feature off entirely.  It is useful for letting subscribers review
changes immediately, rather than clicking to follow a URL.

\subsubsection{Configuring the \hgext{notify} hook}

You can set up the \hgext{notify} hook to send one email message per
incoming changeset, or one per incoming group of changesets (all those
that arrived in a single pull or push).
\begin{codesample2}
  [hooks]
  # send one email per group of changes
  changegroup.notify = python:hgext.notify.hook
  # send one email per change
  incoming.notify = python:hgext.notify.hook
\end{codesample2}

Configuration information for this hook lives in the
\rcsection{notify} section of a \hgrc\ file.
\begin{itemize}
\item[\rcitem{notify}{test}] By default, this hook does not send out
  email at all; instead, it prints the message that it \emph{would}
  send.  Set this item to \texttt{false} to allow email to be sent.
  The reason that sending of email is turned off by default is that it
  takes several tries to configure this extension exactly as you would
  like, and it would be bad form to spam subscribers with a number of
  ``broken'' notifications while you debug your configuration.
\item[\rcitem{notify}{config}] The path to a configuration file that
  contains subscription information.  This is kept separate from the
  main \hgrc\ so that you can maintain it in a repository of its own.
  People can then clone that repository, update their subscriptions,
  and push the changes back to your server.
\item[\rcitem{notify}{strip}] The number of leading path separator
  characters to strip from a repository's path, when deciding whether
  a repository has subscribers.  For example, if the repositories on
  your server live in \dirname{/home/hg/repos}, and \hgext{notify} is
  considering a repository named \dirname{/home/hg/repos/shared/test},
  setting \rcitem{notify}{strip} to \texttt{4} will cause
  \hgext{notify} to trim the path it considers down to
  \dirname{shared/test}, and it will match subscribers against that.
\item[\rcitem{notify}{template}] The template text to use when sending
  messages.  This specifies both the contents of the message header
  and its body.
\item[\rcitem{notify}{maxdiff}] The maximum number of lines of diff
  data to append to the end of a message.  If a diff is longer than
  this, it is truncated.  By default, this is set to 300.  Set this to
  \texttt{0} to omit diffs from notification emails.
\item[\rcitem{notify}{sources}] A list of sources of changesets to
  consider.  This lets you limit \hgext{notify} to only sending out
  email about changes that remote users pushed into this repository
  via a server, for example.  See section~\ref{sec:hook:sources} for
  the sources you can specify here.
\end{itemize}

If you set the \rcitem{web}{baseurl} item in the \rcsection{web}
section, you can use it in a template; it will be available as
\texttt{webroot}.

Here is an example set of \hgext{notify} configuration information.
\begin{codesample2}
  [notify]
  # really send email
  test = false
  # subscriber data lives in the notify repo
  config = /home/hg/repos/notify/notify.conf
  # repos live in /home/hg/repos on server, so strip 4 "/" chars
  strip = 4
  template = X-Hg-Repo: \{webroot\}
    Subject: \{webroot\}: \{desc|firstline|strip\}
    From: \{author\}

    changeset \{node|short\} in \{root\}
    details: \{baseurl\}\{webroot\}?cmd=changeset;node=\{node|short\}
    description:
      \{desc|tabindent|strip\}

  [web]
  baseurl = http://hg.example.com/
\end{codesample2}

This will produce a message that looks like the following:
\begin{codesample2}
  X-Hg-Repo: tests/slave
  Subject: tests/slave: Handle error case when slave has no buffers
  Date: Wed,  2 Aug 2006 15:25:46 -0700 (PDT)

  changeset 3cba9bfe74b5 in /home/hg/repos/tests/slave
  details: http://hg.example.com/tests/slave?cmd=changeset;node=3cba9bfe74b5
  description:
          Handle error case when slave has no buffers
  diffs (54 lines):

  diff -r 9d95df7cf2ad -r 3cba9bfe74b5 include/tests.h
  --- a/include/tests.h      Wed Aug 02 15:19:52 2006 -0700
  +++ b/include/tests.h      Wed Aug 02 15:25:26 2006 -0700
  @@ -212,6 +212,15 @@ static __inline__ void test_headers(void *h)
  [...snip...]
\end{codesample2}

\subsubsection{Testing and troubleshooting}

Do not forget that by default, the \hgext{notify} extension \emph{will
  not send any mail} until you explicitly configure it to do so, by
setting \rcitem{notify}{test} to \texttt{false}.  Until you do that,
it simply prints the message it \emph{would} send.

\section{Information for writers of hooks}
\label{sec:hook:ref}

\section{In-process hook execution}

An in-process hook is called with arguments of the following form:
\begin{codesample2}
  def myhook(ui, repo, **kwargs):
      pass
\end{codesample2}
The \texttt{ui} parameter is a \pymodclass{mercurial.ui}{ui} object.
The \texttt{repo} parameter is a
\pymodclass{mercurial.localrepo}{localrepository} object.  The
names and values of the \texttt{**kwargs} parameters depend on the
hook being invoked, with the following common features:
\begin{itemize}
\item If a parameter is named \texttt{node} or
  \texttt{parent\emph{N}}, it will contain a hexadecimal changeset ID.
  The empty string is used to represent ``null changeset ID'' instead
  of a string of zeroes.
\item If a parameter is named \texttt{url}, it will contain the URL of
  a remote repository, if that can be determined.
\item Boolean-valued parameters are represented as Python
  \texttt{bool} objects.
\end{itemize}

An in-process hook is called without a change to the process's working
directory (unlike external hooks, which are run in the root of the
repository).  It must not change the process's working directory, or
it will cause any calls it makes into the Mercurial API to fail.

If a hook returns a boolean ``false'' value, it is considered to have
succeeded.  If it returns a boolean ``true'' value or raises an
exception, it is considered to have failed.  A useful way to think of
the calling convention is ``tell me if you fail''.

Note that changeset IDs are passed into Python hooks as hexadecimal
strings, not the binary hashes that Mercurial's APIs normally use.  To
convert a hash from hex to binary, use the
\pymodfunc{mercurial.node}{bin} function.

\section{External hook execution}

An external hook is passed to the shell of the user running Mercurial.
Features of that shell, such as variable substitution and command
redirection, are available.  The hook is run in the root directory of
the repository (unlike in-process hooks, which are run in the same
directory that Mercurial was run in).

Hook parameters are passed to the hook as environment variables.  Each
environment variable's name is converted in upper case and prefixed
with the string ``\texttt{HG\_}''.  For example, if the name of a
parameter is ``\texttt{node}'', the name of the environment variable
representing that parameter will be ``\texttt{HG\_NODE}''.

A boolean parameter is represented as the string ``\texttt{1}'' for
``true'', ``\texttt{0}'' for ``false''.  If an environment variable is
named \envar{HG\_NODE}, \envar{HG\_PARENT1} or \envar{HG\_PARENT2}, it
contains a changeset ID represented as a hexadecimal string.  The
empty string is used to represent ``null changeset ID'' instead of a
string of zeroes.  If an environment variable is named
\envar{HG\_URL}, it will contain the URL of a remote repository, if
that can be determined.

If a hook exits with a status of zero, it is considered to have
succeeded.  If it exits with a non-zero status, it is considered to
have failed.

\section{Finding out where changesets come from}

A hook that involves the transfer of changesets between a local
repository and another may be able to find out information about the
``far side''.  Mercurial knows \emph{how} changes are being
transferred, and in many cases \emph{where} they are being transferred
to or from.

\subsubsection{Sources of changesets}
\label{sec:hook:sources}

Mercurial will tell a hook what means are, or were, used to transfer
changesets between repositories.  This is provided by Mercurial in a
Python parameter named \texttt{source}, or an environment variable named
\envar{HG\_SOURCE}.

\begin{itemize}
\item[\texttt{serve}] Changesets are transferred to or from a remote
  repository over http or ssh.
\item[\texttt{pull}] Changesets are being transferred via a pull from
  one repository into another.
\item[\texttt{push}] Changesets are being transferred via a push from
  one repository into another.
\item[\texttt{bundle}] Changesets are being transferred to or from a
  bundle.
\end{itemize}

\subsubsection{Where changes are going---remote repository URLs}
\label{sec:hook:url}

When possible, Mercurial will tell a hook the location of the ``far
side'' of an activity that transfers changeset data between
repositories.  This is provided by Mercurial in a Python parameter
named \texttt{url}, or an environment variable named \envar{HG\_URL}.

This information is not always known.  If a hook is invoked in a
repository that is being served via http or ssh, Mercurial cannot tell
where the remote repository is, but it may know where the client is
connecting from.  In such cases, the URL will take one of the
following forms:
\begin{itemize}
\item \texttt{remote:ssh:\emph{ip-address}}---remote ssh client, at
  the given IP address.
\item \texttt{remote:http:\emph{ip-address}}---remote http client, at
  the given IP address.  If the client is using SSL, this will be of
  the form \texttt{remote:https:\emph{ip-address}}.
\item Empty---no information could be discovered about the remote
  client.
\end{itemize}

\section{Hook reference}

\section{\hook{changegroup}---after remote changesets added}
\label{sec:hook:changegroup}

This hook is run after a group of pre-existing changesets has been
added to the repository, for example via a \hgcmd{pull} or
\hgcmd{unbundle}.  This hook is run once per operation that added one
or more changesets.  This is in contrast to the \hook{incoming} hook,
which is run once per changeset, regardless of whether the changesets
arrive in a group.

Some possible uses for this hook include kicking off an automated
build or test of the added changesets, updating a bug database, or
notifying subscribers that a repository contains new changes.

Parameters to this hook:
\begin{itemize}
\item[\texttt{node}] A changeset ID.  The changeset ID of the first
  changeset in the group that was added.  All changesets between this
  and \index{tags!\texttt{tip}}\texttt{tip}, inclusive, were added by
  a single \hgcmd{pull}, \hgcmd{push} or \hgcmd{unbundle}.
\item[\texttt{source}] A string.  The source of these changes.  See
  section~\ref{sec:hook:sources} for details.
\item[\texttt{url}] A URL.  The location of the remote repository, if
  known.  See section~\ref{sec:hook:url} for more information.
\end{itemize}

See also: \hook{incoming} (section~\ref{sec:hook:incoming}),
\hook{prechangegroup} (section~\ref{sec:hook:prechangegroup}),
\hook{pretxnchangegroup} (section~\ref{sec:hook:pretxnchangegroup})

\section{\hook{commit}---after a new changeset is created}
\label{sec:hook:commit}

This hook is run after a new changeset has been created.

Parameters to this hook:
\begin{itemize}
\item[\texttt{node}] A changeset ID.  The changeset ID of the newly
  committed changeset.
\item[\texttt{parent1}] A changeset ID.  The changeset ID of the first
  parent of the newly committed changeset.
\item[\texttt{parent2}] A changeset ID.  The changeset ID of the second
  parent of the newly committed changeset.
\end{itemize}

See also: \hook{precommit} (section~\ref{sec:hook:precommit}),
\hook{pretxncommit} (section~\ref{sec:hook:pretxncommit})

\section{\hook{incoming}---after one remote changeset is added}
\label{sec:hook:incoming}

This hook is run after a pre-existing changeset has been added to the
repository, for example via a \hgcmd{push}.  If a group of changesets
was added in a single operation, this hook is called once for each
added changeset.

You can use this hook for the same purposes as the \hook{changegroup}
hook (section~\ref{sec:hook:changegroup}); it's simply more convenient
sometimes to run a hook once per group of changesets, while other
times it's handier once per changeset.

Parameters to this hook:
\begin{itemize}
\item[\texttt{node}] A changeset ID.  The ID of the newly added
  changeset.
\item[\texttt{source}] A string.  The source of these changes.  See
  section~\ref{sec:hook:sources} for details.
\item[\texttt{url}] A URL.  The location of the remote repository, if
  known.  See section~\ref{sec:hook:url} for more information.
\end{itemize}

See also: \hook{changegroup} (section~\ref{sec:hook:changegroup}) \hook{prechangegroup} (section~\ref{sec:hook:prechangegroup}), \hook{pretxnchangegroup} (section~\ref{sec:hook:pretxnchangegroup})

\section{\hook{outgoing}---after changesets are propagated}
\label{sec:hook:outgoing}

This hook is run after a group of changesets has been propagated out
of this repository, for example by a \hgcmd{push} or \hgcmd{bundle}
command.

One possible use for this hook is to notify administrators that
changes have been pulled.

Parameters to this hook:
\begin{itemize}
\item[\texttt{node}] A changeset ID.  The changeset ID of the first
  changeset of the group that was sent.
\item[\texttt{source}] A string.  The source of the of the operation
  (see section~\ref{sec:hook:sources}).  If a remote client pulled
  changes from this repository, \texttt{source} will be
  \texttt{serve}.  If the client that obtained changes from this
  repository was local, \texttt{source} will be \texttt{bundle},
  \texttt{pull}, or \texttt{push}, depending on the operation the
  client performed.
\item[\texttt{url}] A URL.  The location of the remote repository, if
  known.  See section~\ref{sec:hook:url} for more information.
\end{itemize}

See also: \hook{preoutgoing} (section~\ref{sec:hook:preoutgoing})

\section{\hook{prechangegroup}---before starting to add remote changesets}
\label{sec:hook:prechangegroup}

This controlling hook is run before Mercurial begins to add a group of
changesets from another repository.

This hook does not have any information about the changesets to be
added, because it is run before transmission of those changesets is
allowed to begin.  If this hook fails, the changesets will not be
transmitted.

One use for this hook is to prevent external changes from being added
to a repository.  For example, you could use this to ``freeze'' a
server-hosted branch temporarily or permanently so that users cannot
push to it, while still allowing a local administrator to modify the
repository.

Parameters to this hook:
\begin{itemize}
\item[\texttt{source}] A string.  The source of these changes.  See
  section~\ref{sec:hook:sources} for details.
\item[\texttt{url}] A URL.  The location of the remote repository, if
  known.  See section~\ref{sec:hook:url} for more information.
\end{itemize}

See also: \hook{changegroup} (section~\ref{sec:hook:changegroup}),
\hook{incoming} (section~\ref{sec:hook:incoming}), ,
\hook{pretxnchangegroup} (section~\ref{sec:hook:pretxnchangegroup})

\section{\hook{precommit}---before starting to commit a changeset}
\label{sec:hook:precommit}

This hook is run before Mercurial begins to commit a new changeset.
It is run before Mercurial has any of the metadata for the commit,
such as the files to be committed, the commit message, or the commit
date.

One use for this hook is to disable the ability to commit new
changesets, while still allowing incoming changesets.  Another is to
run a build or test, and only allow the commit to begin if the build
or test succeeds.

Parameters to this hook:
\begin{itemize}
\item[\texttt{parent1}] A changeset ID.  The changeset ID of the first
  parent of the working directory.
\item[\texttt{parent2}] A changeset ID.  The changeset ID of the second
  parent of the working directory.
\end{itemize}
If the commit proceeds, the parents of the working directory will
become the parents of the new changeset.

See also: \hook{commit} (section~\ref{sec:hook:commit}),
\hook{pretxncommit} (section~\ref{sec:hook:pretxncommit})

\section{\hook{preoutgoing}---before starting to propagate changesets}
\label{sec:hook:preoutgoing}

This hook is invoked before Mercurial knows the identities of the
changesets to be transmitted.

One use for this hook is to prevent changes from being transmitted to
another repository.

Parameters to this hook:
\begin{itemize}
\item[\texttt{source}] A string.  The source of the operation that is
  attempting to obtain changes from this repository (see
  section~\ref{sec:hook:sources}).  See the documentation for the
  \texttt{source} parameter to the \hook{outgoing} hook, in
  section~\ref{sec:hook:outgoing}, for possible values of this
  parameter.
\item[\texttt{url}] A URL.  The location of the remote repository, if
  known.  See section~\ref{sec:hook:url} for more information.
\end{itemize}

See also: \hook{outgoing} (section~\ref{sec:hook:outgoing})

\section{\hook{pretag}---before tagging a changeset}
\label{sec:hook:pretag}

This controlling hook is run before a tag is created.  If the hook
succeeds, creation of the tag proceeds.  If the hook fails, the tag is
not created.

Parameters to this hook:
\begin{itemize}
\item[\texttt{local}] A boolean.  Whether the tag is local to this
  repository instance (i.e.~stored in \sfilename{.hg/localtags}) or
  managed by Mercurial (stored in \sfilename{.hgtags}).
\item[\texttt{node}] A changeset ID.  The ID of the changeset to be tagged.
\item[\texttt{tag}] A string.  The name of the tag to be created.
\end{itemize}

If the tag to be created is revision-controlled, the \hook{precommit}
and \hook{pretxncommit} hooks (sections~\ref{sec:hook:commit}
and~\ref{sec:hook:pretxncommit}) will also be run.

See also: \hook{tag} (section~\ref{sec:hook:tag})

\section{\hook{pretxnchangegroup}---before completing addition of
  remote changesets}
\label{sec:hook:pretxnchangegroup}

This controlling hook is run before a transaction---that manages the
addition of a group of new changesets from outside the
repository---completes.  If the hook succeeds, the transaction
completes, and all of the changesets become permanent within this
repository.  If the hook fails, the transaction is rolled back, and
the data for the changesets is erased.

This hook can access the metadata associated with the almost-added
changesets, but it should not do anything permanent with this data.
It must also not modify the working directory.

While this hook is running, if other Mercurial processes access this
repository, they will be able to see the almost-added changesets as if
they are permanent.  This may lead to race conditions if you do not
take steps to avoid them.

This hook can be used to automatically vet a group of changesets.  If
the hook fails, all of the changesets are ``rejected'' when the
transaction rolls back.

Parameters to this hook:
\begin{itemize}
\item[\texttt{node}] A changeset ID.  The changeset ID of the first
  changeset in the group that was added.  All changesets between this
  and \index{tags!\texttt{tip}}\texttt{tip}, inclusive, were added by
  a single \hgcmd{pull}, \hgcmd{push} or \hgcmd{unbundle}.
\item[\texttt{source}] A string.  The source of these changes.  See
  section~\ref{sec:hook:sources} for details.
\item[\texttt{url}] A URL.  The location of the remote repository, if
  known.  See section~\ref{sec:hook:url} for more information.
\end{itemize}

See also: \hook{changegroup} (section~\ref{sec:hook:changegroup}),
\hook{incoming} (section~\ref{sec:hook:incoming}),
\hook{prechangegroup} (section~\ref{sec:hook:prechangegroup})

\section{\hook{pretxncommit}---before completing commit of new changeset}
\label{sec:hook:pretxncommit}

This controlling hook is run before a transaction---that manages a new
commit---completes.  If the hook succeeds, the transaction completes
and the changeset becomes permanent within this repository.  If the
hook fails, the transaction is rolled back, and the commit data is
erased.

This hook can access the metadata associated with the almost-new
changeset, but it should not do anything permanent with this data.  It
must also not modify the working directory.

While this hook is running, if other Mercurial processes access this
repository, they will be able to see the almost-new changeset as if it
is permanent.  This may lead to race conditions if you do not take
steps to avoid them.

Parameters to this hook:
\begin{itemize}
\item[\texttt{node}] A changeset ID.  The changeset ID of the newly
  committed changeset.
\item[\texttt{parent1}] A changeset ID.  The changeset ID of the first
  parent of the newly committed changeset.
\item[\texttt{parent2}] A changeset ID.  The changeset ID of the second
  parent of the newly committed changeset.
\end{itemize}

See also: \hook{precommit} (section~\ref{sec:hook:precommit})

\section{\hook{preupdate}---before updating or merging working directory}
\label{sec:hook:preupdate}

This controlling hook is run before an update or merge of the working
directory begins.  It is run only if Mercurial's normal pre-update
checks determine that the update or merge can proceed.  If the hook
succeeds, the update or merge may proceed; if it fails, the update or
merge does not start.

Parameters to this hook:
\begin{itemize}
\item[\texttt{parent1}] A changeset ID.  The ID of the parent that the
  working directory is to be updated to.  If the working directory is
  being merged, it will not change this parent.
\item[\texttt{parent2}] A changeset ID.  Only set if the working
  directory is being merged.  The ID of the revision that the working
  directory is being merged with.
\end{itemize}

See also: \hook{update} (section~\ref{sec:hook:update})

\section{\hook{tag}---after tagging a changeset}
\label{sec:hook:tag}

This hook is run after a tag has been created.

Parameters to this hook:
\begin{itemize}
\item[\texttt{local}] A boolean.  Whether the new tag is local to this
  repository instance (i.e.~stored in \sfilename{.hg/localtags}) or
  managed by Mercurial (stored in \sfilename{.hgtags}).
\item[\texttt{node}] A changeset ID.  The ID of the changeset that was
  tagged.
\item[\texttt{tag}] A string.  The name of the tag that was created.
\end{itemize}

If the created tag is revision-controlled, the \hook{commit} hook
(section~\ref{sec:hook:commit}) is run before this hook.

See also: \hook{pretag} (section~\ref{sec:hook:pretag})

\section{\hook{update}---after updating or merging working directory}
\label{sec:hook:update}

This hook is run after an update or merge of the working directory
completes.  Since a merge can fail (if the external \command{hgmerge}
command fails to resolve conflicts in a file), this hook communicates
whether the update or merge completed cleanly.

\begin{itemize}
\item[\texttt{error}] A boolean.  Indicates whether the update or
  merge completed successfully.
\item[\texttt{parent1}] A changeset ID.  The ID of the parent that the
  working directory was updated to.  If the working directory was
  merged, it will not have changed this parent.
\item[\texttt{parent2}] A changeset ID.  Only set if the working
  directory was merged.  The ID of the revision that the working
  directory was merged with.
\end{itemize}

See also: \hook{preupdate} (section~\ref{sec:hook:preupdate})

%%% Local Variables: 
%%% mode: latex
%%% TeX-master: "00book"
%%% End: 
