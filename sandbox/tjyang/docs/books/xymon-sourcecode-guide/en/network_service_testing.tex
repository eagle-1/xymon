%%%%%%%%%%%%%%%%%%%%%%%%%%%%%%%%%%%%%%%%%%%%%%%%%%%%%%%%%%%%%%%%%%%%%%%%%%%%%%
%
%%%%%%%%%%%%%%%%%%%%%%%%%%%%%%%%%%%%%%%%%%%%%%%%%%%%%%%%%%%%%%%%%%%%%%%%%%%%%%
\chapter{Network Service Testing}

%%%%%%%%%%%%%%%%%%%%%%%%%%%%%%%%%%%%%%%%%%%%%%%%%%%%%%%%%%%%%%%%%%%%%%%%%%%%%%
%
%%%%%%%%%%%%%%%%%%%%%%%%%%%%%%%%%%%%%%%%%%%%%%%%%%%%%%%%%%%%%%%%%%%%%%%%%%%%%%
\section{BBTEST-NET}
 bbtest-net - Hobbit network test tool 

\subsection{SYNOPSIS}
\textbf{bbtest-net -?}
 
\textbf{bbtest-net --help}
 
\textbf{bbtest-net --version}
 
\textbf{bbtest-net [options]}
 
 (See the OPTIONS section for a description of the available commandline options). 

 
\subsection{DESCRIPTION}
\emph{bbtest-net(1)} handles the network tests of hosts defined in the
Hobbit configuration file, bb-hosts. It is normally run at regular
intervals by \emph{hobbitlaunch(8)} via an entry in the
\emph{hobbitlaunch.cfg(5)} file. 

 bbtest-net does all of the normal tests of TCP-based network services
 (telnet, ftp, ssh, smtp, pop, imap ....) - i.e. all of the services
 listed as BBNETSVCS in bbdef.sh. For these tests, a completely new
 and very speedy service- checker has been implemented. 



  bbtest-net has built-in support for testing SSL-enabled protocols,
  e.g. imaps, pop3s, nntps, telnets, if SSL-support was enabled when
  configuring bbgen. The full list of known tests is found in the
  \emph{bb-services(5)} file in \$BBHOME/etc/bb-services. 



  In addition, it implements the ``dns'' and ``dig'' tests for testing
  DNS servers. This is done in the same way as bb-network.sh does it. 


  bbtest-net also implements a check for NTP servers - this test is
  called ``ntp''. If you want to use it, you must define the NTPDATE
  environment variable to point at the location of your
  \emph{ntpdate(1)} program. 



  Note: bbtest-net performs the connectivity test (ping) based on the
  hostname, unless the host is tagged with ``testip'' or the
  ``--dns=ip'' option is used. So the target of the connectivity test
  can be determined by your /etc/hosts file or DNS. 



 
\subsection{GENERAL OPTIONS}


\begin{description}

\item[--timeout=N] Determines the timeout (in seconds) for each
  service that is tested. For TCP tests (those from BBNETSVCS), if the
  connection to the service does not succeed within N seconds, the
  service is reported as being down. For HTTP tests, this is the
  absolute limit for the entire request to the webserver (the time
  needed to connect to the server, plus the time it takes the server
  to respond to the request). Default: 10 seconds 



 

\item[--conntimeout=N] This option is deprecated, and will be
  ignored. Use the --timeout option instead. 


 

\item[--cmdtimeout=N] This option sets a timeout for the external
  commands used for testing of NTP and RPC services, and to perform
  traceroute. 


 

\item[--concurrency=N] Determines the number of network tests that run
  in parallel. Default is operating system dependent, but will usually
  be 256. If bbtest-net begins to complain about not being able to get
  a ``socket'', try running bbtest-net with a lower value like 50 or
  100. 


 

\item[--dns-timeout=N (default: 30 seconds)] bbtest-net will timeout
  all DNS lookups after N seconds. Any pending DNS lookups are
  regarded as failed, i.e. the network tests that depend on this DNS
  lookup will report an error.  

 Note: If you use the --no-ares option, timeout of DNS lookups cannot
 be controlled by bbtest-net. 


\item[--dns-max-all=N] Same as ``--dns-timeout=N''. The
  ``--dns-max-all'' option is deprecated and should not be used. 
 

\item[--dns=[ip|only|standard]] Determines how bbtest-net finds the IP
  adresses of the hosts to test. By default (the ``standard''),
  bbtest-net does a DNS lookup of the hostname to determine the IP
  address, unless the host has the ``testip'' tag, or the DNS lookup
  fails.  

 With ``--dns=only'' bbtest-net will ONLY do the DNS lookup; it it
 fails, then all services on that host will be reported as being down.  

 With ``--dns=ip'' bbtest-net will never do a DNS lookup; it will use
 the IP adresse specified in bb-hosts for the tests. Thus, this
 setting is equivalent to having the ``testip'' tag on all hosts. Note
 that http tests will ignore this setting and still perform a DNS
 lookup for the hostname given in the URL; see the ``bbtest-net tags
 for HTTP tests'' section in \emph{bb-hosts(5)}

\item[--no-ares] Disable the ARES resolver built into bbtest-net. This
  makes bbtest-net resolve hostnames using your system resolver
  function. You should only use this as a last resort if bbtest-net
  cannot resolve the hostnames you use in the normal way (via DNS or
  /etc/hosts). One reason for using this would be if you need to
  resolve hostnames via NIS/NIS+ (a.k.a. Yellow Pages).  

 The system resolver function does not provide a mechanism for
 controlling timeouts of the hostname lookups, so if your DNS or NIS
 server is down, bbtest-net can take a very long time to run. The
 --dns-timeout option is effectively disabled when using this option. 
 

\item[--dnslog=FILENAME] Log failed hostname lookups to the file
  FILENAME. FILENAME should be a full pathname. 

\item[--report[=COLUMNNAME]] With this option, bbtest-net will send a
  status message with details of how many hosts were processed, how
  many tests were generated, any errors that occurred during the run,
  and some timing statistics. The default columnname is ``bbtest''. 

\item[--test-untagged] When using the BBLOCATION environment variable
  to test only hosts on a particular network segment, bbtest-net will
  ignore hosts that do not have any ``NET:x'' tag. So only hosts that
  have a NET:\$BBLOCATION tag will be tested.  

 With this option, hosts with no NET: tag are included in the test, so
 that all hosts that either have a matching NET: tag, or no NET: tag
 at all are tested. 

 

\item[--frequenttestlimit=N] Used with the \emph{bbretest-net.sh(1)}
 bbgen extension. This option determines how long failed tests remain
 in the frequent-test queue. The default is 1800 seconds (30
 minutes). 


 

\item[--timelimit=N] Causes bbtest-net to generate a warning if the
  run-time of bbtest-net exceeds N seconds. By default N is set to the
  value of BBSLEEP, so a warning triggers if the network tests cannot
  complete in the time given for one cycle of the BBNET server. Apart
  from the warning, this option has no effect, i.e. it will not
  terminate bbtest-net prematurely. So to eliminate any such warnings,
  use this option with a very high value of N. 



\item[--huge=N] Warn if the response from a TCP test is more than N
  bytes. If you see from the bbtest status report that you are
  transferring large amounts of data for your tests, you can enable
  this option to see which tests have large replies.   Default: 0
  (disabled). 


 

\item[--validity=N] Make the test results valid for N minutes before
  they go purple. By default test results are valid for 30 minutes; if
  you run bbtest-net less often than that, the results will go purple
  before the next run of bbtest-net. This option lets you change how
  long the status is valid. 


\end{description}

\subsection{OPTIONS FOR TESTS OF THE SIMPLE TCP SERVICES}
\begin{description}
\item[--checkresponse[=COLOR]] When testing well-known services
  (e.g. FTP, SSH, SMTP, POP-2, POP-3, IMAP, NNTP and rsync),
  bbtest-net will look for a valid service-specific ``OK''
  response. If another reponse is seen, this will cause the test to
  report a warning (yellow) status. Without this option, the response
  from the service is ignored.  

 The optional color-name is used to select a color other than yellow
 for the status message when the response is
 wrong. E.g. ``--checkresponse=red'' will cause a ``red'' status
 message to be sent when the service does not respond as expected. 


 

\item[--no-flags] By default, bbtest-net sends some extra information
  in the status messages, called ``flags''. These are used by bbgen
  e.g. to pick different icons for reversed tests when generating the
  Hobbit webpages. This option makes bbtest-net omit these flags from
  the status messages. 



\end{description}

\subsection{OPTIONS FOR THE PING TEST}
 Note: bbtest-net uses the program defined by the FPING environment to
 execute ping-tests - by default, that is the \emph{hobbitping(1)}
 utility. See \emph{hobbitserver.cfg(5)} for a description of how to
 customize this, e.g. if you need to run it with ``sudo'' or a similar
 tool. 


 \begin{description}
\item[--ping] Enables bbtest-net's ping test. The column name used for
  ping test results is defined by the PINGCOLUMN environment variable
  in \emph{hobbitserver.cfg(5).}

 
 If not specifed, bbtest-net uses the CONNTEST environment variable to
 determine if it should perform the ping test or not. So if you prefer
 to use another tool to implement ping checks, either set the CONNTEST
 environment variable to false, or run bbtest-net with the
 ``--noping''. 


 

\item[--noping] Disable the connectivity test. 

 

\item[--trace]
\item[--notrace] Enable/disable the use of traceroute when a ping-test
  fails. Performing a traceroute for failed ping tests is a slow
  operation, so the default is not to do any traceroute, unless it is
  requested on a per-host basis via the ``trace'' tag in the
  \emph{bb-hosts(5) } entry for each host. The ``--trace'' option
  changes this, so the default becomes to run traceroute on all hosts
  where the ping test fails; you can then disable it on specific hosts
  by putting a ``notrace'' tag on the host-entry. 


 


\end{description}

\subsection{OPTIONS FOR HTTP (WEB) TESTS}
\begin{description}
\item[--content=CONTENTTESTNAME] Determines the name of the column
  Hobbit displays for content checks. The default is ``content''. If
  you have used the ``cont.sh'' or ``cont2.sh'' scripts earlier, you
  may want to use ``--content=cont'' to report content checks using
  the same test name as these scripts do. 


\end{description}

\subsection{OPTIONS FOR SSL CERTIFICATE TESTS}
\begin{description}
\item[--ssl=SSLCERTTESTNAME] Determines the name of the column Hobbit
  displays for the SSL certificate checks. The default is
  ``sslcert''. 

\item[--no-ssl] Disables reporting of the SSL certificate check. 

 

\item[--sslwarn=N]
\item[--sslalarm=N] Determines the number of days before an SSL
  certificate expires, where bbtest-net will generate a warning or
  alarm status for the SSL certificate column. 



\end{description}
\subsection{DEBUGGING OPTIONS}
\begin{description}
\item[--no-update] Don't send any status updates to the BBDISPLAY
  server. Instead, all messages are dumped to stdout. 



\item[--timing] Causes bbtest-net to collect information about the
  time spent in different parts of the program. The information is
  printed on stdout just before the program ends. Note that this
  information is also included in the status report sent with the
  ``--report'' option. 


\item[--debug] Dumps a bunch of status about the tests as they progress to stdout. 

 

\item[--dump[=before|=after|=both]] Dumps internal memory structures
  before and/or after the tests have executed. 


 
\end{description}
\subsection{INFORMATIONAL OPTIONS}
\begin{description}
\item[--help or -?] Provide a summary of available commandline options. 

 

\item[--version] Prints the version number of bbtest-net 

 

\item[--services] Dump the list of defined TCP services bbtest-net
  knows how to test. Do not run any tests. 



\end{description}
\subsection{USING COOKIES IN WEB TESTS}
 If the file \$BBHOME/etc/cookies exist, cookies will be read from
 this file and sent along with the HTTP requests when checking
 websites. This file is in the Netscape Cookie format, see
 \url{http://www.netscape.com/newsref/std/cookie}\_spec.html for
 details on this format. The \emph{curl(1)} utility can output a file
 in this format if run with the ``--cookie-jar FILENAME'' option. 


 


 
\subsection{ABOUT SSL CERTIFICATE CHECKS}
 When bbtest-net tests services that use SSL- or TLS-based protocols,
 it will check that the server certificate has not expired. This check
 happens automatically for https (secure web), pop3s, imaps, nntps and
 all other SSL-enabled services (except ldap, see LDAP TESTS below). 


  All certificates found for a host are reported in one status message. 


  Note: On most systems, the end-date of the certificate is limited to
  Jan 19th, 2038. If your certificate is valid after this date,
  bbtest-net will report it as valid only until Jan 19, 2038. This is
  due to limitations in your operating system C library. 


 
\subsection{LDAP TESTS}
 ldap testing can be done in two ways. If you just put an ``ldap'' or
 ``ldaps'' tag in bb-hosts, a simple test is performed that just
 verifies that it is possible to establish a connection to the port
 running the ldap service (389 for ldap, 636 for ldaps). 


  Instead you can put an LDAP URI in bb-hosts. This will cause
  bbtest-net to initiate a full-blown LDAP session with the server,
  and do an LDAP search for the objects defined by the URI. This
  requires that bbtest-net was built with LDAP support, and relies on
  an existing LDAP library to be installed. It has been tested with
  OpenLDAP 2.0.26 (from Red Hat 9) and 2.1.22. The Solaris 8 system
  ldap library has also been confirmed to work for un-encrypted (plain
  ldap) access. 


  The format of LDAP URI's is defined in RFC 2255. LDAP URLs look like this: \begin{verbatim}


  \textbf{ldap://}
\emph{hostport}
\textbf{/}
\emph{dn}
[\textbf{?}
\emph{attrs}
[\textbf{?}
\emph{scope}
[\textbf{?}
\emph{filter}
[\textbf{?}
\emph{exts}
]]]]

where:
  \emph{hostport}
 is a host name with an optional ":portnumber"
  \emph{dn}
 is the search base
  \emph{attrs}
 is a comma separated list of attributes to request
  \emph{scope}
 is one of these three strings:
    base one sub (default=base)
  \emph{filter}
 is filter
  \emph{exts}
 are recognized set of LDAP and/or API extensions.

Example:
  ldap://ldap.example.net/dc=example,dc=net?cn,sn?sub?(cn=*)

\end{verbatim}



  All ``bind'' operations to LDAP servers use simple
  authentication. Kerberos and SASL are not supported. If your LDAP
  server requires a username/password, use the ``ldaplogin'' tag to
  specify this, cf. \emph{bb-hosts(5) } If no username/password
  information is provided, an anonymous bind will be attempted. 



  SSL support requires both a client library and an LDAP server that
  support LDAPv3; it uses the LDAP ``STARTTLS'' protocol request after
  establishing a connection to the standard (non-encrypted) LDAP port
  (usually port 389). It has only been tested with OpenSSL 2.x, and
  probably will not work with any other LDAP library. 



  The older LDAPv2 experimental method of tunnelling normal LDAP
  traffic through an SSL connection - ldaps, running on port 636 - is
  not supported, unless someone can explain how to get the OpenLDAP
  library to support it. This method was never formally described in
  an RFC, and implementations of it are non-standard. 



  For a discussion of the various ways of running encrypted ldap, see  
\url{http://www.openldap.org/lists/openldap-software/200305/msg00079.html} 
\url{http://www.openldap.org/lists/openldap-software/200305/msg00084.html} 
\url{http://www.openldap.org/lists/openldap-software/200201/msg00042.html} 
\url{http://www.openldap.org/lists/openldap-software/200206/msg00387.html}


  When testing LDAP URI's, all of the communications are handled by
  the ldap library. Therefore, it is not possible to obtain the SSL
  certificate used by the LDAP server, and it will not show up in the
  ``sslcert'' column. 



 
\subsection{USING MULTIPLE NETWORK TEST SYSTEMS}
 If you have more than one system running network tests - e.g. if your
 network is separated by firewalls - then is is problematic to
 maintain multiple bb-hosts files for each of the systems. bbtest-net
 supports the NET:location tag in \emph{bb-hosts(5)} to distinguish
 between hosts that should be tested from different network
 locations. If you set the environment variable BBLOCATION e.g. to
 ``dmz'' before running bbtest-net, then it will only test hosts that
 have a ``NET:dmz'' tag in bb-hosts. This allows you to keep all of
 your hosts in the same bb-hosts file, but test different sets of
 hosts by different BBNET systems. 


 


 
\subsection{BBTEST-NET INTERNALS}
 bbtest-net first reads the bb-services file to see which network
 tests are defined. It then scans the bb-hosts file, and collects
 information about the TCP service tests that need to be tested. It
 picks out only the tests that were listed in the bb-services file,
 plus the ``dns'', ``dig'' and ``ntp'' tests - those tests that
 bb-network.sh would normally use the ``bbnet'' tool to test. 


  It then runs two tasks in parallel: First, a separate process is
  started to run the ``hobbitping'' tool for the connectivity
  tests. While hobbitping is busy doing the ``ping'' checks,
  bbtest-net runs all of the TCP-based network tests. 



  All of the TCP-based service checks are handled by a connection
  tester written specifically for this purpose. It uses only standard
  Unix-style network programming, but relies on the Unix ``select(2)''
  system-call to handle many simultaneous connections happening in
  parallel. Exactly how many parallel connections are being used
  depends on your operating system - the default is FD\_SETSIZE/4,
  which amounts to 256 on many Unix systems. 



  You can choose the number of concurrent connections with the
  ``--concurrency=N'' option to bbtest-net. 



  Connection attempts timeout after 10 seconds - this can be changed
  with the ``--timeout=N'' option. 



  Both of these settings play a part in deciding how long the testing
  takes. A conservative estimate for doing N TCP tests is: 



  
(1+(N/concurrency))*timeout 


  In real life it will probably be less, as the above formula is for
  every test to require a timeout. Since the most normal use of BB is
  to check for services that are active, you should have a lot less
  timeouts. 



  The ``ntp'' and ``rpcinfo'' checks rely on external programs to do
  each test. Thus, they perform only marginally better than the
  standard bb-network.sh script. 



 
\subsection{ENVIRONMENT VARIABLES}
\begin{description}
\item[BBLOCATION] Defines the network segment where bbtest-net is
  currently running. This is used to filter out only the entries in
  the \emph{bb-hosts(5)} file that have a matching ``NET:LOCATION''
  tag, and execute the tests for only those hosts. 


 

\item[BBMAXMSGSPERCOMBO] Defines the maximum number of status messages
  that can be sent in one combo message. Default is 0 - no limit.  

 In practice, the maximum size of a single Hobbit message sets a limit
 - the default value for the maximum message size is 32 KB, but that
 will easily accomodate 100 status messages per transmission. So if
 you want to experiment with this setting, I suggest starting with a
 value of 10. 



\item[BBSLEEPBETWEENMSGS] Defines a a delay (in microseconds) after
  each message is transmitted to the BBDISPLAY server. The default is
  0, i.e. send the messages as fast as possible. This gives your
  BBDISPLAY server some time to process the message before the next
  message comes in. Depending on the speed of your BBDISPLAY server,
  it may be necessary to set this value to half a second or even 1 or
  2 seconds. Note that the value is specified in MICROseconds, so to
  define a delay of half a second, this must be set to the value
  ``500000''; a delay of 1 second is achieved by setting this to
  ``1000000'' (one million). 


 

\item[FPING] Command used to run the \emph{hobbitping(1) }
  utility. Used by bbtest-net for connectivity (ping) testing. See
  \emph{hobbitserver.cfg(5)} for more information about how to
  customize the program that is executed to do ping tests. 


\item[TRACEROUTE] Location of the \emph{traceroute(8)} utility, or an
  equivalent tool e.g. \emph{mtr(8).} Optionally used when a
  connectivity test fails to pinpoint the network location that is
  causing the failure. 


 

\item[NTPDATE] Location of the \emph{ntpdate(1) }
 utility. Used by bbtest-net when checking the ``ntp'' service. 

 

\item[RPCINFO] Location of the \emph{rpcinfo(8) }
 utility. Used by bbtest-net for the ``rpc'' service checks. 

 


\end{description}
\subsection{FILES}
\begin{description}
\item[~/server/etc/bb-services (Hobbit)] This file contains
  definitions of TCP services that bbtest-net can test. Definitions
  for a default set of common services is built into bbtest-net, but
  these can be overridden or supplemented by defining services in the
  bb-services file. See \emph{bb-services(5)} for details on this
  file. 


 

\item[\$BBHOME/etc/netrc - authentication data for password-protected
  webs] If you have password-protected sites, you can put the
  usernames and passwords for these here. They will then get picked up
  automatically when running your network tests. This works for
  web-sites that use the ``Basic'' authentication scheme in HTTP. See
  \emph{ftp(1)} for details - a sample entry would look like this  

 
machinewww.acme.comloginfredpasswordWilma1  
 Note that the machine-name must be the name you use in the
 \url{http://machinename/} URL setting - it need not be the one you
 use for the system-name in Hobbit. 


 
\item[\$BBHOME/etc/cookies] This file may contain website cookies, in
  the Netscape HTTP Cookie format. If a website requires a static
  cookie to be present in order for the check to complete, then you
  can add this cookie to this file, and it will be sent along with the
  HTTP request. To get the cookies into this file, you can use the
  ``curl --cookie-jar FILE'' to request the URL that sets the cookie. 


 

\item[\$BBTMP/*.status - test status summary] Each time bbtest-net
  runs, if any tests fail (i.e. they result in a red status) then they
  will be listed in a file name TESTNAME.[LOCATION].status. The
  LOCATION part may be null. This file is used to determine how long
  the failure has lasted, which in turn decides if this test should be
  included in the tests done by \emph{bbretest-net.sh(1)}

 
 It is also used internally by bbtest-net when determining the color
 for tests that use the ``badconn'' or ``badTESTNAME'' tags. 


 

\item[\$BBTMP/frequenttests.[LOCATION]] This file contains the
  hostnames of those hosts that should be retested by the
  \emph{bbretest-net.sh(1)} test tool. It is updated only by
  bbtest-net during the normal runs, and read by bbretest-net.sh. 



\end{description}
\subsection{SEE ALSO}
bb-hosts(5), bb-services(5), hobbitserver.cfg(5), hobbitping(1), curl(1), ftp(1), fping(1), ntpdate(1), rpcinfo(8) 


%%%%%%%%%%%%%%%%%%%%%%%%%%%%%%%%%%%%%%%%%%%%%%%%%%%%%%%%%%%%%%%%%%%%%%%%%%%%%%
%
%%%%%%%%%%%%%%%%%%%%%%%%%%%%%%%%%%%%%%%%%%%%%%%%%%%%%%%%%%%%%%%%%%%%%%%%%%%%%%
\newpage
\section{BB-SERVICES}

 bb-services - Configuration of TCP network services 

 
\subsection{SYNOPSIS}
\textbf{\$BBHOME/etc/bb-services}


 
\subsection{DESCRIPTION}
\textbf{bb-services}
 contains definitions of how \emph{bbtest-net(1)}
 should test a TCP-based network service (i.e. all common network services except HTTP and DNS). For each service, a simple dialogue can be defined to check that the service is functioning normally, and optional flags determine if the service has e.g. a banner or requires SSL- or telnet-style handshaking to be tested. 

 
\subsection{FILE FORMAT}
 bb-services is a text file. A simple service definition for the SMTP service would be this:  


  
[smtp]  
 
send''mail$\backslash$r$\backslash$nquit$\backslash$r$\backslash$n''  
 
expect''220''  
 
optionsbanner  



  This defines a service called ``smtp''. When the connection is first established, bbtest-net will send the string ``mail$\backslash$r$\backslash$nquit$\backslash$r$\backslash$n'' to the service. It will then expect a response beginning with ``220''. Any data returned by the service (a so-called ``banner'') will be recorded and included in the status message. 


  The full set of commands available for the bb-services file are: 


 \begin{description}
\item[[NAME]] Define the name of the TCP service, which will also be the column-name in the resulting display on the test status. If multiple tests share a common definition (e.g. ssh, ssh1 and ssh2 are tested identically), you may list these in a single ``[ssh|ssh1|ssh2]'' definition, separating each service-name with a pipe-sign. 

 

\item[send STRING]
\item[expect STRING] Defines the strings to send to the service after a connection is established, and the response that is expected. Either of these may be omitted, in which case \emph{bbtest-net(1)}
 will simply not send any data, or match a response against anything. 

  The send- and expect-strings use standard escaping for non-printable characters. ``$\backslash$r'' represents a carriage-return (ASCII 13), ``$\backslash$n'' represents a line-feed (ASCII 10), ``$\backslash$t'' represents a TAB (ASCII 8). Binary data is input as ``$\backslash$xNN'' with NN being the hexadecimal value of the byte. 


 

\item[port NUMBER] Define the default TCP port-number for this service. If no portnumber is defined, \emph{bbtest-net(1)}
 will attempt to lookup the portnumber in the standard /etc/services file. 

 

\item[options option1[,option2][,option3]] Defines test options. The possible options are  
 
banner-includereceiveddatainthestatusmessage  
 
ssl-serviceusesSSLsoperformanSSLhandshake  
 
telnet-serviceistelnet,soexchangetelnetoptions 

 


 


\end{description}

\subsection{FILES}
\textbf{\$BBHOME/etc/bb-services}


 
\subsection{SEE ALSO}
bbtest-net(1) 
  


%%%%%%%%%%%%%%%%%%%%%%%%%%%%%%%%%%%%%%%%%%%%%%%%%%%%%%%%%%%%%%%%%%%%%%%%%%%%%%
%
%%%%%%%%%%%%%%%%%%%%%%%%%%%%%%%%%%%%%%%%%%%%%%%%%%%%%%%%%%%%%%%%%%%%%%%%%%%%%%
\newpage
\section{HOBBITFETCH}
\subsection{NAME}
 hobbitfetch - fetch client data from passive clients \
\subsection{SYNOPSIS}
\textbf{hobbitfetch [--server=HOBBIT.SERVER.IP] [options]}


 
\subsection{DESCRIPTION}
 This utility is used to collect data from Hobbit clients. 

  Normally, Hobbit clients will themselves take care of sending all of
  their data directly to the Hobbit server. In that case, you do not
  need this utility at all. However, in some network setups clients
  may be prohibited from establishing a connection to an external
  server such as the Hobbit server, due to firewall policies. In such
  a setup you can configure the client to store all of the client data
  locally by enabling the \emph{msgcache(8)} utility on the client,
  and using \textbf{hobbitfetch} on the Hobbit server to collect data
  from the clients. 



  hobbitfetch will only collect data from clients that have the
  \textbf{pulldata} tag listed in the \emph{bb-hosts(5)} file. The
  IP-address listed in the bb-hosts file must be correct, since this
  is the IP-address where hobbitfetch will attempt to contact the
  client. If the msgcache daemon is running on a non-standard
  IP-address or portnumber, you can specify the portnumber as in
  \textbf{pulldata=192.168.1.2:8084} for contacting the msgcache
  daemon using IP 192.168.1.2 port 8084. If the IP-address is omitted,
  the default IP in the bb-hosts file is used. If the port number is
  omitted, the portnumber from the BBPORT setting in
  \emph{hobbitserver.cfg(5)} is used (normally, this is port 1984). 



 
\subsection{OPTIONS}
\begin{description}
\item[--server=HOBBIT.SERVER.IP] Defines the IP address of the Hobbit
  server where the collected client messages are forwarded to. By
  default, messages are sent to the loopback address 127.0.0.1,
  i.e. to a Hobbit server running on the same host as hobbitfetch. 


 

\item[--interval=N] Sets the interval (in seconds) between polls of a client. Default: 60 seconds. 

 

\item[--id=N] Used when you have a setup with multiple Hobbit
 servers. In that case, you must run hobbitfetch on each of the Hobbit
 servers, with hobbitfetch instance using a different value of N. This
 allows several Hobbit servers to pick up data from the clients
 running msgcache, and msgcache can distinguish between which messages
 have already been forwarded to which server.  

 N is a number in the range 1-31. 

 

\item[--log-interval=N] Limit how often hobbitfetch will log problems
  with fetching data from a host, in seconds. Default: 900 seconds (15
  minutes). This is to prevent a host that is down or where msgcache
  has not been started from flooding the hobbitfetch logs. Note that
  this is ignored when debugging is enabled. 


 

\item[--debug] Enable debugging output. 

 


\end{description}
\subsection{SEE ALSO}
msgcache(8), hobbitd(8), hobbit(7) 

 
%%%%%%%%%%%%%%%%%%%%%%%%%%%%%%%%%%%%%%%%%%%%%%%%%%%%%%%%%%%%%%%%%%%%%%%%%%%%%%
%
%%%%%%%%%%%%%%%%%%%%%%%%%%%%%%%%%%%%%%%%%%%%%%%%%%%%%%%%%%%%%%%%%%%%%%%%%%%%%%
\newpage
\section{bbcmd}
 \motohbcmd{bbcmd} - Run a Hobbit command with environment set \subsection{SYNOPSIS}
\textbf{bbcmd --env=ENVFILE COMMAND}


 
\subsection{DESCRIPTION}
\emph{bbcmd(1)} is a utility that can setup the Hobbit environment
variables as defined in a \emph{hobbitlaunch(8)} compatible
environment definition file, and then execute a command with this
environment in place. It is mostly used for testing extension scripts
or in other situations where you need to run a single command with the
environment in place. 


The ``--env=ENVFILE'' option points bbcmd to the file where the
environment definitions are loaded from. 



COMMAND is the command to execute after setting up the environment. 

If you want to run multiple commands, it is often easiest to just use
``sh'' as the COMMAND - this gives you a sub-shell with the
environment defined globally. 

\subsection{SEE ALSO} hobbitlaunch(8), hobbit(7) 



%%%%%%%%%%%%%%%%%%%%%%%%%%%%%%%%%%%%%%%%%%%%%%%%%%%%%%%%%%%%%%%%%%%%%%%%%%%%%%
%
%%%%%%%%%%%%%%%%%%%%%%%%%%%%%%%%%%%%%%%%%%%%%%%%%%%%%%%%%%%%%%%%%%%%%%%%%%%%%%
\newpage
\section{CLIENTLAUNCH.CFG}
 clientlaunch.cfg - Task definitions for the hobbitlaunch utility 

 
\subsection{SYNOPSIS}
\textbf{~hobbit/client/etc/clientlaunch.cfg}


 
\subsection{DESCRIPTION}
 The clientlaunch.cfg file holds the list of tasks that hobbitlaunch
 runs on a Hobbit client. This is typically just the Hobbit client
 itself, but you can add custom test scripts here and have them
 executed regularly by the Hobbit scheduler. 


 
\subsection{FILE FORMAT}
 See the \emph{hobbitlaunch.cfg(5)}
 description. 

 
\subsection{SEE ALSO}
hobbitlaunch(8), hobbit(7) 

