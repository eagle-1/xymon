\chapter{File names and pattern matching}
\label{chap:names}

Mercurial provides mechanisms that let you work with file names in a
consistent and expressive way.

\section{Simple file naming}

Mercurial uses a unified piece of machinery ``under the hood'' to
handle file names.  Every command behaves uniformly with respect to
file names.  The way in which commands work with file names is as
follows.

If you explicitly name real files on the command line, Mercurial works
with exactly those files, as you would expect.
\interaction{filenames.files}

When you provide a directory name, Mercurial will interpret this as
``operate on every file in this directory and its subdirectories''.
Mercurial traverses the files and subdirectories in a directory in
alphabetical order.  When it encounters a subdirectory, it will
traverse that subdirectory before continuing with the current
directory.
\interaction{filenames.dirs}

\section{Running commands without any file names}

Mercurial's commands that work with file names have useful default
behaviours when you invoke them without providing any file names or
patterns.  What kind of behaviour you should expect depends on what
the command does.  Here are a few rules of thumb you can use to
predict what a command is likely to do if you don't give it any names
to work with.
\begin{itemize}
\item Most commands will operate on the entire working directory.
  This is what the \hgcmd{add} command does, for example.
\item If the command has effects that are difficult or impossible to
  reverse, it will force you to explicitly provide at least one name
  or pattern (see below).  This protects you from accidentally
  deleting files by running \hgcmd{remove} with no arguments, for
  example.
\end{itemize}

It's easy to work around these default behaviours if they don't suit
you.  If a command normally operates on the whole working directory,
you can invoke it on just the current directory and its subdirectories
by giving it the name ``\dirname{.}''.
\interaction{filenames.wdir-subdir}

Along the same lines, some commands normally print file names relative
to the root of the repository, even if you're invoking them from a
subdirectory.  Such a command will print file names relative to your
subdirectory if you give it explicit names.  Here, we're going to run
\hgcmd{status} from a subdirectory, and get it to operate on the
entire working directory while printing file names relative to our
subdirectory, by passing it the output of the \hgcmd{root} command.
\interaction{filenames.wdir-relname}

\section{Telling you what's going on}

The \hgcmd{add} example in the preceding section illustrates something
else that's helpful about Mercurial commands.  If a command operates
on a file that you didn't name explicitly on the command line, it will
usually print the name of the file, so that you will not be surprised
what's going on.

The principle here is of \emph{least surprise}.  If you've exactly
named a file on the command line, there's no point in repeating it
back at you.  If Mercurial is acting on a file \emph{implicitly},
because you provided no names, or a directory, or a pattern (see
below), it's safest to tell you what it's doing.

For commands that behave this way, you can silence them using the
\hggopt{-q} option.  You can also get them to print the name of every
file, even those you've named explicitly, using the \hggopt{-v}
option.

\section{Using patterns to identify files}

In addition to working with file and directory names, Mercurial lets
you use \emph{patterns} to identify files.  Mercurial's pattern
handling is expressive.

On Unix-like systems (Linux, MacOS, etc.), the job of matching file
names to patterns normally falls to the shell.  On these systems, you
must explicitly tell Mercurial that a name is a pattern.  On Windows,
the shell does not expand patterns, so Mercurial will automatically
identify names that are patterns, and expand them for you.

To provide a pattern in place of a regular name on the command line,
the mechanism is simple:
\begin{codesample2}
  syntax:patternbody
\end{codesample2}
That is, a pattern is identified by a short text string that says what
kind of pattern this is, followed by a colon, followed by the actual
pattern.

Mercurial supports two kinds of pattern syntax.  The most frequently
used is called \texttt{glob}; this is the same kind of pattern
matching used by the Unix shell, and should be familiar to Windows
command prompt users, too.  

When Mercurial does automatic pattern matching on Windows, it uses
\texttt{glob} syntax.  You can thus omit the ``\texttt{glob:}'' prefix
on Windows, but it's safe to use it, too.

The \texttt{re} syntax is more powerful; it lets you specify patterns
using regular expressions, also known as regexps.

By the way, in the examples that follow, notice that I'm careful to
wrap all of my patterns in quote characters, so that they won't get
expanded by the shell before Mercurial sees them.

\section{Shell-style \texttt{glob} patterns}

This is an overview of the kinds of patterns you can use when you're
matching on glob patterns.

The ``\texttt{*}'' character matches any string, within a single
directory.
\interaction{filenames.glob.star}

The ``\texttt{**}'' pattern matches any string, and crosses directory
boundaries.  It's not a standard Unix glob token, but it's accepted by
several popular Unix shells, and is very useful.
\interaction{filenames.glob.starstar}

The ``\texttt{?}'' pattern matches any single character.
\interaction{filenames.glob.question}

The ``\texttt{[}'' character begins a \emph{character class}.  This
matches any single character within the class.  The class ends with a
``\texttt{]}'' character.  A class may contain multiple \emph{range}s
of the form ``\texttt{a-f}'', which is shorthand for
``\texttt{abcdef}''.
\interaction{filenames.glob.range}
If the first character after the ``\texttt{[}'' in a character class
is a ``\texttt{!}'', it \emph{negates} the class, making it match any
single character not in the class.

A ``\texttt{\{}'' begins a group of subpatterns, where the whole group
matches if any subpattern in the group matches.  The ``\texttt{,}''
character separates subpatterns, and ``\texttt{\}}'' ends the group.
\interaction{filenames.glob.group}

\subsubsection{Watch out!}

Don't forget that if you want to match a pattern in any directory, you
should not be using the ``\texttt{*}'' match-any token, as this will
only match within one directory.  Instead, use the ``\texttt{**}''
token.  This small example illustrates the difference between the two.
\interaction{filenames.glob.star-starstar}

\section{Regular expression matching with \texttt{re} patterns}

Mercurial accepts the same regular expression syntax as the Python
programming language (it uses Python's regexp engine internally).
This is based on the Perl language's regexp syntax, which is the most
popular dialect in use (it's also used in Java, for example).

I won't discuss Mercurial's regexp dialect in any detail here, as
regexps are not often used.  Perl-style regexps are in any case
already exhaustively documented on a multitude of web sites, and in
many books.  Instead, I will focus here on a few things you should
know if you find yourself needing to use regexps with Mercurial.

A regexp is matched against an entire file name, relative to the root
of the repository.  In other words, even if you're already in
subbdirectory \dirname{foo}, if you want to match files under this
directory, your pattern must start with ``\texttt{foo/}''.

One thing to note, if you're familiar with Perl-style regexps, is that
Mercurial's are \emph{rooted}.  That is, a regexp starts matching
against the beginning of a string; it doesn't look for a match
anywhere within the string.  To match anywhere in a string, start
your pattern with ``\texttt{.*}''.

\section{Filtering files}

Not only does Mercurial give you a variety of ways to specify files;
it lets you further winnow those files using \emph{filters}.  Commands
that work with file names accept two filtering options.
\begin{itemize}
\item \hggopt{-I}, or \hggopt{--include}, lets you specify a pattern
  that file names must match in order to be processed.
\item \hggopt{-X}, or \hggopt{--exclude}, gives you a way to
  \emph{avoid} processing files, if they match this pattern.
\end{itemize}
You can provide multiple \hggopt{-I} and \hggopt{-X} options on the
command line, and intermix them as you please.  Mercurial interprets
the patterns you provide using glob syntax by default (but you can use
regexps if you need to).

You can read a \hggopt{-I} filter as ``process only the files that
match this filter''.
\interaction{filenames.filter.include}
The \hggopt{-X} filter is best read as ``process only the files that
don't match this pattern''.
\interaction{filenames.filter.exclude}

\section{Ignoring unwanted files and directories}

XXX.

\section{Case sensitivity}
\label{sec:names:case}

If you're working in a mixed development environment that contains
both Linux (or other Unix) systems and Macs or Windows systems, you
should keep in the back of your mind the knowledge that they treat the
case (``N'' versus ``n'') of file names in incompatible ways.  This is
not very likely to affect you, and it's easy to deal with if it does,
but it could surprise you if you don't know about it.

Operating systems and filesystems differ in the way they handle the
\emph{case} of characters in file and directory names.  There are
three common ways to handle case in names.
\begin{itemize}
\item Completely case insensitive.  Uppercase and lowercase versions
  of a letter are treated as identical, both when creating a file and
  during subsequent accesses.  This is common on older DOS-based
  systems.
\item Case preserving, but insensitive.  When a file or directory is
  created, the case of its name is stored, and can be retrieved and
  displayed by the operating system.  When an existing file is being
  looked up, its case is ignored.  This is the standard arrangement on
  Windows and MacOS.  The names \filename{foo} and \filename{FoO}
  identify the same file.  This treatment of uppercase and lowercase
  letters as interchangeable is also referred to as \emph{case
    folding}.
\item Case sensitive.  The case of a name is significant at all times.
  The names \filename{foo} and {FoO} identify different files.  This
  is the way Linux and Unix systems normally work.
\end{itemize}

On Unix-like systems, it is possible to have any or all of the above
ways of handling case in action at once.  For example, if you use a
USB thumb drive formatted with a FAT32 filesystem on a Linux system,
Linux will handle names on that filesystem in a case preserving, but
insensitive, way.

\section{Safe, portable repository storage}

Mercurial's repository storage mechanism is \emph{case safe}.  It
translates file names so that they can be safely stored on both case
sensitive and case insensitive filesystems.  This means that you can
use normal file copying tools to transfer a Mercurial repository onto,
for example, a USB thumb drive, and safely move that drive and
repository back and forth between a Mac, a PC running Windows, and a
Linux box.

\section{Detecting case conflicts}

When operating in the working directory, Mercurial honours the naming
policy of the filesystem where the working directory is located.  If
the filesystem is case preserving, but insensitive, Mercurial will
treat names that differ only in case as the same.

An important aspect of this approach is that it is possible to commit
a changeset on a case sensitive (typically Linux or Unix) filesystem
that will cause trouble for users on case insensitive (usually Windows
and MacOS) users.  If a Linux user commits changes to two files, one
named \filename{myfile.c} and the other named \filename{MyFile.C},
they will be stored correctly in the repository.  And in the working
directories of other Linux users, they will be correctly represented
as separate files.

If a Windows or Mac user pulls this change, they will not initially
have a problem, because Mercurial's repository storage mechanism is
case safe.  However, once they try to \hgcmd{update} the working
directory to that changeset, or \hgcmd{merge} with that changeset,
Mercurial will spot the conflict between the two file names that the
filesystem would treat as the same, and forbid the update or merge
from occurring.

\section{Fixing a case conflict}

If you are using Windows or a Mac in a mixed environment where some of
your collaborators are using Linux or Unix, and Mercurial reports a
case folding conflict when you try to \hgcmd{update} or \hgcmd{merge},
the procedure to fix the problem is simple.

Just find a nearby Linux or Unix box, clone the problem repository
onto it, and use Mercurial's \hgcmd{rename} command to change the
names of any offending files or directories so that they will no
longer cause case folding conflicts.  Commit this change, \hgcmd{pull}
or \hgcmd{push} it across to your Windows or MacOS system, and
\hgcmd{update} to the revision with the non-conflicting names.

The changeset with case-conflicting names will remain in your
project's history, and you still won't be able to \hgcmd{update} your
working directory to that changeset on a Windows or MacOS system, but
you can continue development unimpeded.

\begin{note}
  Prior to version~0.9.3, Mercurial did not use a case safe repository
  storage mechanism, and did not detect case folding conflicts.  If
  you are using an older version of Mercurial on Windows or MacOS, I
  strongly recommend that you upgrade.
\end{note}

%%% Local Variables: 
%%% mode: latex
%%% TeX-master: "00book"
%%% End: 
