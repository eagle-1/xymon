%%%%%%%%%%%%%%%%%%%%%%%%%%%%%%%%%%%%%%%%%%%%%%%%%%%%%%%%%%%%%%%%%%%%%%%%%%%%%%
%
%%%%%%%%%%%%%%%%%%%%%%%%%%%%%%%%%%%%%%%%%%%%%%%%%%%%%%%%%%%%%%%%%%%%%%%%%%%%%%
\chapter{Hobbit Overview}
%%%%%%%%%%%%%%%%%%%%%%%%%%%%%%%%%%%%%%%%%%%%%%%%%%%%%%%%%%%%%%%%%%%%%%%%%%%%%%
%
%%%%%%%%%%%%%%%%%%%%%%%%%%%%%%%%%%%%%%%%%%%%%%%%%%%%%%%%%%%%%%%%%%%%%%%%%%%%%%
\section{Hobbit - Introduction to the Hobbit Monitor}

~\cite{web:patchutils} Hobbit is a tool for monitoring the health of your networked servers
 and the applications running on them. It provides a simple, intuitive
 way of checking the health of your systems from a webbrowser, and can
 also alert you to any problems that arise through alarms sent as
 e-mail, SMS messages, via a pager or by other means. 


 Hobbit is Open Source software, licensed under the GNU GPL. This
 means that you are free to use Hobbit as much as you like, and you
 are free to re-distribute it and change it to suit your specific
 needs. However, if you change it then you must make your changes
 available to others on the same terms that you received Hobbit
 originally. See the file COPYING in the Hobbit source-archive for
 details. 

 Hobbit initially began life as an enhancement to Big Brother called
 ``bbgen''. Over a period of 5 years, Hobbit has evolved from a small
 add-on to a full-fledged monitoring system with capabilities far
 exceeding what was in the original Big Brother package. Hobbit does
 still maintain some compatibility with Big Brother, so it is
 possible to migrate from Big Brother to Hobbit without too much
 trouble. 

 Migrating to Hobbit will give you a significant performance boost,
 and provide you with much more advanced monitoring. The Hobbit tools
 are designed for installations that need to monitor a large number
 of hosts, with very little overhead on the monitoring
 server. Monitoring of thousands of hosts with a single Hobbit server
 is possible - it was developed to handle just this task. 

\section{FEATURES}
 These are some of the core features in Hobbit: 

\begin{description}

\item[Monitoring of hosts and networks] Hobbit collects information
  about your systems in two ways: From querying network services (Web,
  LDAP, DNS, Mail etc.), or from scripts that run either on the Hobbit
  server or on the systems you monitor. The Hobbit package includes a
  Hobbit client which you can install on the servers you monitor; it
  collects data about the CPU-load, disk- and memory-utilisation,
  logfiles, network ports in use, file- and directory-information and
  more. All of the information is stored inside Hobbit, and you can
  define conditions that result in alerts, e.g. if a network service
  stops responding, or a disk fills up. 

\item[Centralized configuration] All configuration of Hobbit is done
  on the Hobbit server. Even when monitoring hundreds or thousands of
  hosts, you can control their configuration centrally on the Hobbit
  server - so there is no need for you to login to a system just to
  change e.g. which processes are monitored. 

\item[Works on all major platforms] The Hobbit server works on all
  Unix-like systems, including Linux, Solaris, FreeBSD, AIX, HP-UX and
  others. The Hobbit client supports all major Unix platforms, and
  there are other Open Source projects - e.g. BBWin, see
  \url{http://bbwin.sourceforge.net/} - providing support for
  Microsoft Windows based systems. 

\item[A simple, intuitive web-based front-end] ``Green is good, red is
  bad''. Using the Hobbit webpages is as simple as that. The hosts you
  monitor can be grouped together in a way that makes sense in your
  organisation and presented in a tree-structure. The webpages use
  many techniques to convey information about the monitored systems,
  e.g. different icons can be used for recently changed statuses;
  links to subpages can be listed in multiple columns; different icons
  can be used for dialup-tests or reverse-tests; selected columns can
  be dropped or unconditionally included on the webpages to eliminate
  unwanted information, or always include certain information;
  user-friendly names can be shown for hosts regardless of their true
  hostname. You can also have automatic links to on-line
  documentation, so information about your critical systems is just a
  click away. 

\item[Integrated trend analysis, historical data and SLA reporting]
  Hobbit stores trend- and availability-information about everything
  it monitors. So if you need to look at how your systems behave over
  time, Hobbit has all of the information you need: Whether is is
  response times of your webpages during peak hours, the CPU
  utilisation over the past 4 weeks, or what the availability of a
  site was compared to the SLA - it's all there inside Hobbit. All
  measurements are tracked and made available in time-based graphs. 


  When you need to drill down into events that have occurred, Hobbit
  provides a powerful tool for viewing the event history for each
  statuslog, with overviews of when problems have occurred during the
  past and easy-to-use zoom-in on the event. 

  For SLA reporting, You can configure planned downtime, agreed
  service availability level, service availability time and have
  Hobbit generate availability reports directly showing the actual
  availability measured against the agreed SLA. Such reports of
  service availability can be generated on-the-fly, or pre-generated
  e.g. for monthly reporting. 

\item[Role-based views] You can have multiple different views of the
  same hosts for different parts of the organisation, e.g. one view
  for the hardware group, and another view for the webmasters - all of
  them fed by the same test tools. 


  If you have a dedicated Network Operations Centre, you can configure
  precisely which alerts will appear on their monitors - e.g. a simple
  anomaly in the system logfile need not trigger a call to 3rd-level
  support at 2 AM, but if the on-line shop goes down you do want
  someone to respond immediately. So you put the webcheck for the
  on-line shop on the NOC monitor page, and leave out the log-file
  check. 
 

\item[Also for the techies] The Hobbit user-interface is simple, but
  engineers will also find lots of relevant information. E.g. the data
  that clients report to Hobbit contain the raw output from a number
  of system commands. That information is available directly in
  Hobbit, so an administrator no longer needs to login to a server to
  get an overview of how it is behaving - the very commands they would
  normally run have alredy been performed, and the results are on-line
  in Hobbit. 


\item[Easy to adapt to your needs] Hobbit includes a lot of tests in
  the core package, but there will always be something specific to
  your setup that you would like to watch. Hobbit allows you to write
  test scripts in your favourite scripting language and have the
  results show up as regular status columns in Hobbit. You can trigger
  alerts from these, and even track trends in graphs just by a simple
  configuration setting. 


\item[Real network service tests] The network test tool knows how to
  test most commonly used protocols, including HTTP, SMTP (e-mail),
  DNS, LDAP (directory services), and many more. When checking
  websites, it is possible to not only check that the webserver is
  responding, but also that the response looks correct by matching the
  response against a pre-defined pattern or a checksum. So you can
  test that a network service is really working and supplying the data
  you expect - not just that the service is running. 

  Protocols that use SSL encryption such as https-websites are fully
  supported, and while checking such services the network tester will
  automatically run a check of the validity of the SSL server
  certificate, and warn about certificates that are about to expire. 

\item[Highly configurable alerts] You want to know when something
  breaks. But you don't want to get flooded with alerts all the
  time. Hobbit lets you define several criteria for when to send out
  an alert, so you only get alerts when there is really something that
  needs your attention right away. While you are handling an incident,
  you can tell Hobbit about it so it stops sending more alerts, and so
  that everyone else can check with Hobbit and know that the problem
  is being taken care of. 


\item[Combined super-tests and test interdependencies] If a single
  test is not enough, combination tests can be defined that combine
  the result of several tests to a single status-report. So if you
  need to monitor that at least 3 out of 5 servers are running at any
  time, Hobbit can do that for you and generate the necessary
  availability report. 

  Tests can also be configured to depend on each other, so that when a
  critical router goes down you will get alerts only for the router -
  and not from the 200 hosts behind the router. 

 
\end{description}

\section{SECURITY}
 All of the Hobbit server tools run under an unprivileged user
 account. A single program - the \emph{hobbitping(1)} network
 connectivity tester - must be installed setuid-root, but has been
 written so that it drops all root privileges immediately after
 performing the operation that requires root privileges. 

 It is recommended that you setup a dedicated account for Hobbit. 

 Communications between the Hobbit server and Hobbit clients use the
 Big Brother TCP port 1984. If the Hobbit server is located behind a
 firewall, it must allow for inbound connections to the Hobbit server
 on tcp port 1984. Normally, Hobbit clients - i.e. the servers you
 are monitoring - must be permitted to connect to the Hobbit server
 on this port. However, if that is not possible due to firewall
 policies, then Hobbit includes the \emph{hobbitfetch(8)} and
 \emph{msgcache(8)} tools to allows for a pull-style way of
 collecting data, where it is the Hobbit server that initiates
 connections to the clients. 


 The Hobbit webpages are dynamically generated through CGI programs. 


 Access to the Hobbit webpages is controlled through your webserver
 access controls, e.g. you can require a login through some form of
 HTTP authentication. 

 
\section{DEMONSTRATION SITE}
 A site running this software can be seen at \url{http://www.hswn.dk/hobbit/}

 
\section{PREREQUISITES}

 You will need a Unix-like system (Linux, Solaris, HP-UX, AIX,
 FreeBSD, Mac OS X or similar) with a webserver installed. You will
 also need a C compiler and some additional libraries, but many
 systems come with the required development tools and libraries
 pre-installed. The required libraries are: 


\begin{enumerate}

 \item \textbf{RRDtool}
 This library is used to store and present trend-data. It is required. 

 \item \textbf{libpcre}
 This library is used for advanced pattern-matching of text strings in
 configuration files. This library is required. 

 \item \textbf{OpenSSL}
 This library is used for communication with SSL-enabled network
 services. Although optional, it is recommended that you install this
 for Hobbit since many network tests do use SSL. 

 \item \textbf{OpenLDAP}
 This library is used for testing LDAP servers. Use of this is optional. 

\end{enumerate}

\section{INSTALLATION}
  For more detailed information about Hobbit system requirements and
  how to install Hobbit, refer to the online documentation
  ``Installing Hobbit'' available from the Hobbit webserver (via the
  ``Help'' menu), or from the ``docs/install.html'' file in the Hobbit
  source archive. 


\section{SUPPORT and MAILING LISTS}
hobbit@hswn.dk is an open mailing list for discussions about
Hobbit. If you would like to participate, send an e-mail to
\textbf{hobbit-subscribe@hswn.dk} to join the list. 


An archive of the mailing list is available at \url{http://www.hswn.dk/hobbiton/}


If you just want to be notified of new releases of Hobbit, please
subscribe to the hobbit-announce mailing list. This is a moderated
list, used only for announcing new Hobbit releases. To be added to
the list, send an e-mail to \textbf{hobbit-announce-subscribe@hswn.dk}. 

\section{Hobbit Wiki Book}

~\cite{wikibook:wikibooktjyang} System Monitoring with Hobbit wiki book.

It is a community effort for hobbit documentation.

\begin{enumerate}

\item User Guide  \url{http://en.wikibooks.org/wiki/System_Monitoring_with_Hobbit/User_Guide}

\item User Guide Administration Guide
  \url{http://en.wikibooks.org/wiki/System_Monitoring_with_Hobbit/Administration_Guide}

\item Developer Guide \url{http://en.wikibooks.org/wiki/System_Monitoring_with_Hobbit/Developer_Guide}

\item Other Docs
\url{http://en.wikibooks.org/wiki/System_Monitoring_with_Hobbit/Other_Docs}

\end{enumerate}

\section{HOBBIT SERVER DAEMONS}
 These daemons implement the core functionality of the Hobbit server: 

\begin{enumerate}

 \item \emph{hobbitd(8)} is the core daemon that collects all reports about
 the status of your hosts. It uses a number of helper modules to
 implement certain tasks such as updating logfiles and sending out
 alerts: hobbitd\_client, hobbitd\_history, hobbitd\_alert and
 hobbitd\_rrd. There is also a hobbitd\_filestore module for
 compatibility with Big Brother. 



 \item \emph{hobbitd\_channel(8)}
 Implements the communication between the Hobbit daemon and the other Hobbit server modules. 


 \item \emph{hobbitd\_history(8)}
 Stores historical data about the things that Hobbit monitors. 


 \item \emph{hobbitd\_rrd(8)}
 Stores trend data, which is used to generate graphs of the data monitored by Hobbit. 


 \item \emph{hobbitd\_alert(8)}
 handles alerts. When a status changes to a critical state, this module decides if an alert should be sent out, and to whom. 


 \item \emph{hobbitd\_client(8)}
 handles data collected by the Hobbit clients, analyzes the data and feeds back several status updates to Hobbit to build the view of the client status. 


 \item \emph{hobbitd\_hostdata(8)}
 stores historical client data when something breaks. E.g. when a
 webpage stops responding hobbitd\_hostdata will save the latest
 client data, so that you can use this to view a snapshot of how the
 system state was just prior to it failing. 


\end{enumerate}
 
\section{HOBBIT NETWORK TEST TOOLS}

These tools are used on servers that execute tests of network
services. 

\begin{enumerate}

 \item \emph{hobbitping(1)}
 performs network connectivity (ping) tests. 


 \item \emph{bbtest-net(1)}
 runs the network service tests. 


 \item \emph{bbretest-net.sh(1)} is an extension script for re-doing failed
 network tests with a higher frequency than the normal network
 tests. This allows Hobbit to pick up the recovery of a network
 service as soon as it happens, resulting in less downtime being
 recorded. 

\end{enumerate}



 
\section{HOBBIT TOOLS HANDLING THE WEB USER-INTERFACE}
 These tools take care of generating and updating the various Hobbit web-pages. 

\begin{enumerate}
 \item \emph{bbgen(1)}
 takes care of updating the Hobbit webpages. 


 \item \emph{hobbitsvc.cgi(1)}
 This CGI program generates an HTML view of a single status log. It is used to present the Hobbit status-logs. 


 \item \emph{hobbitgraph.cgi(1)}
 This CGI program generates graphs of the trend-data collected by Hobbit. 


 \item \emph{hobbit-hostgraphs.cgi(1)}
 When you want to combine multiple graphs into one, this CGI lets you
 combine graphs so you can e.g. compare the load on all of the nodes
 in your server farm. 



 \item \emph{hobbit-nkview.cgi(1)}
 Generates the Critical Systems view, based on the currently critical
 systems and the configuration of what systems and services you want
 to monitor when. 



 \item \emph{bb-hist.cgi(1)}
 This CGI program generates a webpage with the most recent history of
 a particular host+service combination. 



 \item \emph{bb-eventlog.cgi(1)}
 This CGI lets you view a log of events that have happened over a
 period of time, for a single host or test, or for multiple systems. 



 \item \emph{bb-ack.cgi(1)}
 This CGI program allows a user to acknowledge an alert he received
 from Hobbit about a host that is in a critical state. Acknowledging
 an alert serves two purposes: First, it stops more alerts from being
 sent so the technicians are not bothered wit more alerts, and
 secondly it provides feedback to those looking at the Hobbit webpages
 that the problem is being handled. 



 \item \emph{hobbit-mailack(8)}
 is a tool for processing acknowledgements sent via e-mail, e.g. as a response to an e-mail alert. 


 \item \emph{hobbit-enadis.cgi(8)}
 is a CGI program to disable or re-enable hosts or individual
 tests. When disabling a host or test, you stop alarms from being sent
 and also any outages do not affect the SLA calculations. So this tool
 is useful when systems are being brought down for maintenance. 



 \item \emph{bb-findhost.cgi(1)}
 is a CGI program that finds a given host in the Hobbit webpages. As
 your Hobbit installation grows, it can become difficult to remember
 exactly which page a host is on; this CGI script lets you find hosts
 easily. 



 \item \emph{bb-rep.cgi(1)} This CGI program triggers the generation of Hobbit availability reports, using \emph{bbgen(1)}
 as the reporting back-end engine. 


 \item \emph{bb-replog.cgi(1)} This CGI program generates the detailed availability report for a particular host+service combination. 


 \item \emph{bb-snapshot.cgi(1)} is a CGI program to build the Hobbit
 webpages in a ``snapshot'' mode, showing the look of the webpages at
 a particular point in time. It uses \emph{bbgen(1)} as the back-end
 engine. 



 \item \emph{hobbit-statusreport.cgi(1)} is a CGI program reporting test
 results for a single status but for several hosts. It is used to
 e.g. see which SSL certificates are about to expire, across all of
 the Hobbit webpages. 



 \item \emph{bb-csvinfo.cgi(1)}
 is a CGI program to present information about a host. The information
 is pulled from a CSV (Comma Separated Values) file, which is easily
 exported from any spreadsheet or database program. 

\end{enumerate}

 
 
\section{CLIENT-SIDE TOOLS}
\begin{enumerate}

 \item \emph{logfetch(1)}
 is a utility used by the Hobbit Unix client to collect information
 from logfiles on the client. It can also monitor various other
 file-related data, e.g. file metadata or directory sizes. 

 \item \emph{clientupdate(1)}
 Is used on Hobbit clients, to automatically update the client
 software with new versions. Through this tool, updates of the client
 software can happen without an administrator having to logon to the
 server. 

 \item \emph{msgcache(8)}
 This tool acts as a mini Hobbit server to the client. It stores
 client data internally, so that the \emph{hobbitfetch(8)} utility can
 pick it up later and send it to the Hobbit server. It is typically
 used on hosts that cannot contact the Hobbit server directly due to
 network- or firewall-restrictions. 

\end{enumerate}

\section{HOBBIT COMMUNICATION TOOLS}

 These tools are used for communications between the Hobbit server and
 the Hobbit clients. If there are no firewalls then they are not
 needed, but it may be necessary due to network or firewall issues to
 make use of them. 

\begin{enumerate}
\item \emph{bbproxy(8)}
 is a proxy-server that forwards Hobbit messages between clients and
 the Hobbit server. The clients must be able to talk to the proxy, and
 the proxy must be able to talk to the Hobbit server. 



\item \emph{hobbitfetch(8)}
 is used when the client is not able to make outbound connections to
 neither bbproxy nor the Hobbit server (typically, for clients located
 in a DMZ network zone). Together with the \emph{msgcache(8)} utility
 running on the client, the Hobbit server can contact the clients and
 pick up their data. 

\end{enumerate}
 
\section{OTHER TOOLS}

\begin{enumerate}

\item \emph{hobbitlaunch(8)}
 is a program scheduler for Hobbit. It acts as a master program for
 running all of the Hobbit tools on a system. On the Hobbit server, it
 controls running all of the server tasks. On a Hobbit client, it
 periodically launches the client to collect data and send them to the
 Hobbit server. 


\item  \emph{bb(1)}
 is the tool used to communicate with the Hobbit server. It is used to
 send status reports to the Hobbit server, through the custom
 Hobbit/BB protocol, or via HTTP. It can be used to query the state of
 tests on the central Hobbit server and retrieve Hobbit configuration
 files. The server-side script \emph{bbmessage.cgi(1) } used to
 receive messages sent via HTTP is also included. 


\item  \emph{bbcmd(1)}
 is a wrapper for the other Hobbit tools which sets up all of the
 environment variables used by Hobbit tools. 



\item  \emph{bbhostgrep(1)}
 is a utility for use by Hobbit extension scripts. It allows an
 extension script to easily pick out the hosts that are relevant to a
 script, so it need not parse a huge bb-hosts file with lots of
 unwanted test-specifications. 



\item  \emph{bbhostshow(1)}
 is a utility to dump the full \emph{bb-hosts(5)}
 file following any ``include'' statements. 


\item  \emph{bbdigest(1)}
 is a utility to compute message digest values for use in content checks that use digests. 


\item  \emph{bbcombotest(1)}
 is an extension script for the Hobbit server, allowing you to build
 complicated tests from simpler Hobbit test results. E.g. you can
 define a test that uses the results from testing your webserver,
 database server and router to have a single test showing the
 availability of your enterprise web application. 



\item  \emph{trimhistory(8)}
 is a tool to trim the Hobbit history logs. It will remove all log
 entries and optionally also the individual status-logs for events
 that happened before a given time. 

\end{enumerate}

\section{VERSIONS}

\begin{enumerate}

 \item \emph{Version 1} of bbgen was relased in November 2002, and optimized the
 webpage generation on Big Brother servers. 

 \item \emph{Version 2} of bbgen was released in April 2003, and added a tool for performing network tests. 

 \item \emph{Version 3} of bbgen was released in September 2004, and eliminated
 the use of several external libraries for network tests, resulting
 in a significant performance improvement. 

 \item \emph{Version 4.0 }released on March 30 2005, the project was
 de-coupled from Big Brother, and the name changed to Hobbit. This
 version was the first full implementation of the Hobbit server, but
 it still used the data collected by Big Brother clients for
 monitoring host metrics. 

 \item \emph{Version 4.1} was released in July 2005 included a simple Hobbit client for Unix. Logfile monitoring was not implemented. 


 \item \emph{Version 4.2} was released in July 2006, and includes a fully functional Hobbit client for Unix. 

\end{enumerate}
 
 
\section{COPYRIGHT}

Hobbit is Copyright(C)2002-2007,HenrikStorner$<$henrik@storner.dk$>$   
Parts of the Hobbit sources are from public-domain or other freely available
sources. These are the the Red-Black tree implementation, and the
MD5-, SHA1-, SHA2- and RIPEMD160-implementations. Details of the
license for these is in the README file included with the Hobbit
sources. All other files are released under the GNU General Public
License version 2, with the additional exemption that compiling,
linking, and/or using OpenSSL is allowed. See the file COPYING for
details. 


 

\section{SEE ALSO}

\begin{enumerate}
\item hobbitd(8) 
\item hobbitd\_channel(8) 
\item hobbitd\_history(8) 
\item hobbitd\_rrd(8)
\item hobbitd\_alert(8)
\item hobbitd\_client(8)
\item hobbitd\_hostdata(8)
\item hobbitping(1)
\item bbtest-net(1)
\item bbretest-net.sh(1)
\item bbgen(1)
\item hobbitsvc.cgi(1)
\item hobbitgraph.cgi(1)
\item hobbit-hostgraphs.cgi(1)
\item hobbit-nkview.cgi(1)
\item bb-hist.cgi(1)
\item bb-eventlog.cgi(1)
\item bb-ack.cgi(1)
\item hobbit-mailack(8)
\item hobbit-enadis.cgi(8)
\item bb-findhost.cgi(1)
\item bb-rep.cgi(1)
\item bb-replog.cgi(1)
\item bb-snapshot.cgi(1)
\item hobbit-statusreport.cgi(1)
\item bb-csvinfo.cgi(1)
\item logfetch(1)
\item clientupdate(1)
\item msgcache(8)
\item bbproxy(8)
\item hobbitfetch(8)
\item hobbitlaunch(8)
\item bb(1)
\item bbmessage.cgi(1)
\item bbcmd(1)
\item bbhostgrep(1)
\item bbhostshow(1)
\item bbdigest(1)
\item bbcombotest(1)
\item trimhistory(8)
\item bb-hosts(5)
\item hobbitlaunch.cfg(5)
\item hobbitserver.cfg(5)
\item hobbit-alerts.cfg(5)
\item hobbit-clients.cfg(5)
\item client-local.cfg(5) 
\end{enumerate}
