



\chapter{BBPROXY}

\section{BBPROXY}
 Section: Maintenance Commands (8) 
Updated: Version Exp: 11 Jan 2008 
 
\section{NAME}
 bbproxy - Hobbit message proxy \section{SYNOPSIS}
\textbf{bbproxy [options] --servers=IP}

 
\section{DESCRIPTION}
\emph{bbproxy(8)}
 is a proxy for forwarding Hobbit messages from one server to another. It will typically be needed if you have clients behind a firewall, so they cannot send status messages to the Hobbit server directly. 

 ~\cite{web:patchutils}hgcmd{bbproxy} serves three purposes. First, it acts as a regular proxy server, allowing clients that cannot connect directly to the Hobbit server to send messages to the Hobbit servers. Although bbproxy is optimized for handling status messages, it will forward all types of messages.  



  Second, it acts as a buffer, smoothing out peak loads if many clients try to send status messages simultaneously. bbproxy can absorb messages very quickly, but will queue them up internally and forward them to the Hobbit server at a reasonable pace. This helps even out the load on your Hobbit server.  



  Third, bbproxy merges small ``status'' messages into larger ``combo'' messages. This can dramatically decrease the number of connections that need to go from bbproxy to the Hobbit server, and is a slightly more efficient way of transmitting data to the Hobbit server. The merging of messages causes ``status'' messages to be delayed for up to 0.25 seconds before being sent off to the Hobbit server. 


 
\section{OPTIONS}
\begin{description}
\item[--servers=SERVERIP[:PORT][,SERVER2IP[:PORT]]] Specifies the IP-address and optional portnumber where incoming messages are forwarded to. The default portnumber is 1984, the standard Hobbit port number. Up to 3 servers can be specified; incoming messages are sent to all of them (except ``config'', ``query'' and ``download'' messages, which go to the LAST server only). If you have Hobbit clients sending their data via this proxy, note that the clients will receive their configuration data from the LAST of the servers listed here. This option is required. 

 

\item[--bbdisplay=SERVERIP[:PORT][,SERVER2IP[:PORT]]] Obsolete. Use ``--servers'' instead. 

 

\item[--listen=LOCALIP[:PORT]] Specifies the IP-adress where bbproxy listens for incoming connections. By default, bbproxy listens on all IP-adresses assigned to the host. If no portnumber is given, port 1984 will be used. 

 

\item[--timeout=N] Specifies the number of seconds after which a connection is aborted due to a timeout. Default: 10 seconds. 

 

\item[--report=[PROXYHOSTNAME.]PROXYSERVICE] If given, this option causes bbproxy to send a status report every 5 minutes to the Hobbit server about itself. If you have set the standard Hobbit environment, you can use ``--report=bbproxy'' to have bbproxy report its status to a ``bbproxy'' column in Hobbit. The default for PROXYHOSTNAME is the \$MACHINE environment variable, i.e. the hostname of the server running bbproxy. See REPORT OUTPUT below for an explanation of the report contents. 

 

\item[--lqueue=N] Size of the listen-queue where incoming connections can queue up before being processed. This should be large to accomodate bursts of activity from clients. Default: 512. 

 

\item[--daemon] Run in daemon mode, i.e. detach and run as a background proces. This is the default. 

 

\item[--no-daemon] Runs bbproxy as a foreground proces. 

 

\item[--pidfile=FILENAME] Specifies the location of a file containing the proces-ID of the bbproxy daemon proces. Default: /var/run/bbproxy.pid. 

 

\item[--logfile=FILENAME] Sends all logging output to the specified file instead of stderr. 

 

\item[--log-details] Log details (IP-address, message type and hostname) to the logfile. This can also be enabled and disabled at run-time by sending the bbproxy proces a SIGUSR1 signal. 

 

\item[--debug] Enable debugging output. 

 


\end{description}
\section{REPORT OUTPUT}
 If enabled via the ``--report'' option, bbproxy will send a status message about itself to the Hobbit server once every 5 minutes. 

  The status message includes the following information: 


 \begin{description}
\item[Incoming messages] The total number of connections accepted from clients since the proxy started. The ``(N msgs/second)'' is the average number of messages per second over the past 5 minutes. 

 

\item[Outbound messages] The total number of messages sent to the Hobbit servers. Note that this is probably smaller than the number of incoming messages, since bbproxy merges messages before sending them. 

 

\item[Incoming - Combo messages] The number of ``combo'' messages received from a client. 

 

\item[Incoming - Status messages] The number of ``status'' messages received from a client. bbproxy attempts to merge these into ``combo'' messages. The ``Messages merged'' is the number of ``status'' messages that were merged into a combo message, the ``Resulting combos'' is the number of ``combo'' messages that resulted from the merging. 

 

\item[Incoming - Page messages] The number of ``page'' messages received from a client. These are discarded, they are generated by the old Big Brother clients, but have no meaning in Hobbit. 

 

\item[Incoming - Other messages] The number of other messages (data, notes, ack, query, ...) messages received from a client. 

 

\item[Proxy ressources - Connection table size] This is the number of connection table slots in the proxy. This measures the number of simultaneously active requests that the proxy has handled, and so gives an idea about the peak number of clients that the proxy has handled simultaneously. 

 

\item[Proxy ressources - Buffer space] This is the number of KB memory allocated for network buffers. 

 

\item[Timeout details - reading from client] The number of messages dropped because reading the message from the client timed out. 

 

\item[Timeout details - connecting to server] The number of messages dropped, because a connection to the Hobbit server could not be established. 

 

\item[Timeout details - sending to server] The number of messages dropped because the communication to the Hobbit server timed out after a connection was established. 

 

\item[Timeout details - recovered] When a timeout happens while sending the status message to the server, bbproxy will attempt to recover the message and retry sending it to the server after waiting a few seconds. This number is the number of messages that were recovered, and so were not lost. 

 

\item[Timeout details - reading from server] The number of response messages that timed out while attempting to read them from the server. Note that this applies to the ``config'' and ``query'' messages only, since all other message types do not get any response from the servers. 

 

\item[Timeout details - sending to client] The number of response messages that timed out while attempting to send them to the client. Note that this applies to the ``config'' and ``query'' messages only, since all other message types do not get any response from the servers. 

 

\item[Average queue time] The average time it took the proxy to process a message, calculated from the messages that have passed through the proxy during the past 5 minutes. This number is computed from the messages that actually end up establishing a connection to the Hobbit server, i.e. status messages that were combined into combo-messages do not go into the calculation - if they did, it would reduce the average time, since it is faster to merge messages than send them out over the network. 

 


\end{description}

\section{}
 If you think the numbers do not add up, here is how they relate. 

  The ``Incoming messages'' should be equal to the sum of the ``Incoming Combo/Status/Page/Other messages'', or slightly more because messages in transit are not included in the per-type message counts. 


  The ``Outbound messages'' should be equal to sum of the ``Incoming Combo/Page/Other messages'', plus the ``Resulting combos'' count, plus ``Incoming Status messages'' minus ``Messages merged'' (this latter number is the number of status messages that were NOT merged into combos, but sent directly). The ``Outbound messages'' may be slightly lower than that, because messages in transit are not included in the ``Outbound messages'' count until they have been fully sent. 


 
\section{SIGNALS}
\begin{description}
\item[SIGHUP] Re-opens the logfile, e.g. after it has been rotated. 

 

\item[SIGTERM] Shut down the proxy. 

 

\item[SIGUSR1] Toggles logging of individual messages. 

 


\end{description}
\section{SEE ALSO}
bb(1), hobbitd(8), hobbit(7) 


\section{Index}
\begin{description}
\item[NAME]
\item[SYNOPSIS]
\item[DESCRIPTION]
\item[OPTIONS]
\item[REPORT OUTPUT]
\item[]
\item[SIGNALS]
\item[SEE ALSO]

\end{description}


