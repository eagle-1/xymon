\chapter{BBGEN}
\label{chap:bbgen}
\section{BBGEN}
 Section: User Commands (1) 
Updated: Version Exp: 11 Jan 2008 

~\cite{web:patchutils}

\section{NAME}
 bbgen - Hobbit webpage generator \section{SYNOPSIS}
\textbf{bbgen -?}
 
\textbf{bbgen --help}
 
\textbf{bbgen --version}
 
\textbf{bbgen [options] [output-directory]}
 
 (See the OPTIONS section for a description of the available commandline options). 

 
\section{DESCRIPTION}
\textbf{bbgen}
 generates the overview webpages for the Hobbit monitor. These are the
 webpages that show the overall status of your hosts, not the detailed
 status pages for each test. 


 
\section{OPTIONS}
 bbgen has a large number of commandline options. The options can be
 used to change the behaviour of bbgen and affect the web pages
 generated by it. 


 
\section{GENERAL OPTIONS}


 \begin{description}
\item[--help or -?] Provide a summary of available commandline options. 

 

\item[--version] Prints the version number of bbgen 

 

\item[--docurl=URL] Make hostnames be hyperlinks to documentation,
  accessed via a common web page (typically a CGI script or a
  PHP-driven dynamic page). The URL parameter is a formatting string
  with the name of the web page - you can put a ``\%s'' in it which
  will be replaced by the hostname being accessed. E.g. if you use the
  bb-notes extension from www.deadcat.net, you would enable this with
  ``--docurl=/hobbit/admin/notes.php?host=\%s''. For the host
  www.storner.dk this will result in a link to
  ``/hobbit/admin/notes.php?host=www.storner.dk``. 


 

\item[--doccgi=URL] This option is deprecated; please use --docurl instead. 

 

\item[--no-doc-window] By default, links to documentation for hosts
  and services cause a new window to appear with the information. With
  this option, the documentation will appear in the same window as the
  Hobbit status. 


 

\item[--htmlextension=.EXTENSION] Sets the filename extension used for
  the webpages generated by bbgen. By default, an extension of
  ``.html'' is used. Note that you need to specify the ``dot''. 


 

\item[--report[=COLUMNNAME]] With this option, bbgen will send a
  status message with details of how many hosts were processed, how
  many pages were generated, any errors that occurred during the run,
  and some timing statistics. The default columnname is ``bbgen''. 


 

\item[--htaccess[=htaccess-filename]] Create .htaccess files when new
  web page directories are created. The content of the .htaccess files
  are determined by the BBHTACCESS environment variable (for the
  top-level directory with bb.html and bb2.html); by the
  BBPAGEHTACCESS variable (for the page-level directories); and by the
  BBSUBPAGEHTACCESS variable for subpage- and subparent-level
  directories. The filename of the .htaccess files default to
  ``.htaccess'' if no filename is given with this option. The
  BBHTACCESS variable is copied verbatim into the top-level .htaccess
  file. The BBPAGEHTACCESS variable may contain a ``\%s'' where the
  name of the page is inserted. The BBSUBPAGEHTACCESS variable may
  contain two ``\%s'' instances: The first is replaced with the
  pagename, the second with the subpagename. 


 

\item[--max-eventcount=N] Limit the eventlog on the BB2 page to only N
  events. Default: 100. 


 

\item[--max-eventtime=N] Limit the eventlog on the BB2 page to events
  that happened within the past N minutes. Default: 240. 


 

\item[--no-eventlog] Disable the eventlog normally displayed on the BB2 page 

 

\item[--max-ackcount=N] Limit the acknowledgment log on the BB2 page to only N events. Default: 25. 

 

\item[--max-acktime=N] Limit the acknowledgment log on the BB2 page to acks that happened within the past N minutes. Default: 240. 

 

\item[--no-acklog] Disable the acknowledgement log normally displayed on the BB2 page. 

 

\item[--nklog[=NK log column]] This generates a text-based log of what
  is shown on the bbnk.html status page, and sends a status message
  for the BBDISPLAY server itself reflecting the color of the NK
  status page. This allows you to track when problems have appeared on
  the bbnk status page. The logfile is stored in \$BBHOME/nkstatus.log 



 


\end{description}

\section{PAGE LAYOUT OPTIONS}
 These options affect how the webpages generated by bbgen appear in the browser. 

 \begin{description}
\item[--pages-last] Put page- and subpage-links after hosts. 
\item[--pages-first] Put page- and subpage-links before hosts (default). 

  These two options decide whether a page with links to subpages and hosts have the hosts or the subpages first. 


 

\item[--subpagecolumns=N] Determines the number of columns used for links to pages and subpages. The default is N=1. 

 

\item[--maxrows=N] Column headings on a page are by default only shown
  at the beginning of a page, subpage or group of hosts. This options
  causes the column headings to repeat for every N hosts shown. 


 

\item[--pagetitle-links] Normally, only the colored ``dots'' next to a
  page or subpage act as links to the page itself. With this option,
  the page title will link to the page also. 


 

\item[--pagetext-headings] Use the description text from the ``page''
  or ``subpage'' tags as a heading for the page, instead of the
  ``Pages hosted locally'' or other standard heading. 


 

\item[--no-underline-headings] Normally, page headings are underlined
  using an HTML ``horizontal ruler'' tag. This option disables the
  underlining of headings. 


 

\item[--recentgifs[=MINUTES]] Use images named COLOR-recent.gif for
  tests, where the test status has changed within the past 24
  hours. These GIF files need to be installed in the
  \$BBHOME/www/gifs/ directory. By default, the threshold is set to 24
  hours - if you want it differently, you can specify the time limit
  also. E.g. ``--recentgifs=3h'' will show the recent GIFs for only 3
  hours after a status change. 


 

\item[--sort-group-only-items] In a normal ``group-only'' directive,
  you can specify the order in which the tests are displayed, from
  left to right. If you prefer to have the tests listed in
  alphabetical order, use this option - the page will then generate
  ``group-only'' groups like it generates normal groups, and sort the
  tests alphabetically. 


 

\item[--dialupskin=URL] If you want to visually show that a test is a
  dialup-test, you can use an alternate set of icons for the
  green/red/yellow$>$/etc. images by specifying this option. The URL
  parameter specified here overrides the normal setting from the
  BBSKIN environment variable, but only for dialup tests. 


 

\item[--reverseskin=URL] Same as ``--dialupskin'', but for reverse tests (tests with '!' in front). 

 

\item[--tooltips=[always,never,main]] Determines which pages use
  tooltips to show the description of the host (from the COMMENT entry
  in the \emph{bb-hosts(5)}

 file). If set to \textbf{always}
, tooltips are used on all pages. If set to \textbf{never}
, tooltips are never used. If set to \textbf{main}
, tooltips are used on the main pages, but not on the BB2 (all non-green) or NK (critical systems) pages. 

 


\end{description}

\section{COLUMN SELECTION OPTIONS}
 These options affect which columns (tests) are included in the webpages generated by bbgen. 

 \begin{description}
\item[--ignorecolumns=test[,test]] The given columns will be
  completely ignored by bbgen when generating webpages. Can be used to
  generate reports where you eliminate some of the more noisy tests,
  like ``msgs''. 


 

\item[--nk-reds-only] Only red status columns will be included on the
  NK page. By default, the NK page will contain hosts with red, yellow
  and clear status. 


 

\item[--bb2colors=COLOR[,COLOR]] Defines which colors cause a test to
  appear on the ``All non-green'' status page (a.k.a. the BB2
  page). COLOR is red, yellow or purple. The default is to include all
  three. 


 

\item[--bb2-ignorecolumns=test[,test]] Same as the --ignorecolumns, but applies to hosts on the BB2 page only. 

 

\item[--bb2-ignorepurples] Deprecated, use ``--bb2colors'' instead. 

 

\item[--bb2-ignoredialups] Ignore all dialup hosts on the BB2 page, including the BB2 eventlog. 

 

\item[--includecolumns=test[,test]] Always include these columns on
  bb2 page Will include certain columns on the bb2.html page,
  regardless of its color. Normally, bb2.html drops a test-column, if
  all tests are green. This can be used e.g. to always have a link to
  the trends column (with the RRD graphs) from your bb2.html page. 


 

\item[--eventignore=test[,test]] Ignore these tests in the BB2 event log display. 

 


\end{description}

\section{STATUS PROPAGATION OPTIONS}
 These options suppress the normal propagation of a status upwards in
 the page hierarchy. Thus, you can have a test with status yellow or
 red, but still have the entire page green. It is useful for tests
 that need not cause an alarm, but where you still want to know the
 actual status. These options set global defaults for all hosts; you
 can use the NOPROPRED and NOPROPYELLOW tags in the \emph{bb-hosts(5)}
 file to apply similar limits on a per-host basis. 


\begin{description}

\item[--nopropyellow=test[,test] or --noprop=test[,test]] Disable
  upwards status propagation when YELLOW. The ``--noprop'' option is
  deprecated and should not be used. 
 

\item[--noproppurple=test[,test]] Disable upwards status propagation when PURPLE. 

 

\item[--nopropred=test[,test]] Disable upwards status propagation when RED or YELLOW. 

 

\item[--nopropack=test[,test]] Disable upwards status propagation when
  status has been acknowledged. If you want to disable all acked tests
  from being propageted, use ``--nopropack=*''. 


 


\end{description}

\section{PURPLE STATUS OPTIONS}
 Purple statuses occur when reporting of a test status stops. A test
 status is valid for a limited amount of time - normally 30 minutes -
 and after this time, the test becomes purple. 


 \begin{description}
\item[--purplelog=FILENAME] Generate a logfile of all purple status messages. 

 


\end{description}

\section{ALTERNATE PAGESET OPTIONS}


 \begin{description}
\item[--pageset=PAGESETNAME] Build webpages for an alternate pageset than the default. See the PAGESETS section below. 

 

\item[--template=TEMPLATE] Use an alternate template for header and
  footer files. Typically used together the the ``--pageset'' option;
  see the PAGESETS section below. 

\end{description}

\section{ALTERNATE OUTPUT FORMATS}


 \begin{description}

\item[--wml[=test1,test2,...]] This option causes bbgen to generate a
  set of WML ``card'' files that can be accessed by a WAP device (cell
  phone, PDA etc.) The generated files contain the hosts that have a
  RED or YELLOW status on tests specified. This option can define the
  default tests to include - the defaults can be overridden or amended
  using the ``WML:'' or ``NK:'' tags in the \emph{bb-hosts(5)}
  file. If no tests are specified, all tests will be included. 


 

\item[--nstab=FILENAME] Generate an HTML file suitable for a Netscape
  6/Mozilla sidebar entry. To actually enable your users to obtain
  such a sidebar entry, you need this Javascript code in a webpage
  (e.g. you can include it in the \$BBHOME/web/bb\_header file): 


%  $<$SCRIPT TYPE=''text/javascript''$>$  
% $<$!--  
% function addNetscapePanel() \{  
% 
%if((typeofwindow.sidebar==''object'')\&\&  
%(typeofwindow.sidebar.addPanel==''function''))  
% 
%window.sidebar.addPanel(``Hobbit'',  
% 
%''\url{http://your.server.com/nstab.html}``,''``);  
% 
%else  
% 
%alert(``SidebaronlyforMozillaorNetscape6+'');  
% \}  
% //--$>$  
% $<$/SCRIPT$>$ 
%

  and then you can include a ``Add this to sidebar'' link using this as a template: 


% $<$AHREF=''javascript:addNetscapePanel();''$>$AddtoSidebar$<$/A$>$ 


  or if you prefer to have the standard Netscape ``Add tab'' button, you would do it with 

%
%  
%$<$AHREF=''javascript:addNetscapePanel();''$>$  
% 
%$<$IMGSRC=''/gifs/add-button.gif''HEIGHT=45WIDTH=100  
% 
%ALT=''[AddSidebar]''STYLE=''border:0''$>$  
% 
%$<$/A$>$ 


  The ``add-button.gif'' is available from Netscape at
  \url{http://developer.netscape.com/docs/manuals/browser/sidebar/add-button.gif.}



  If FILENAME does not begin with a slash, the Netscape sidebar file is placed in the \$BBHOME/www/ directory. 


 

\item[--nslimit=COLOR] The minimum color to include in the Netscape
  Sidebar - default is ``red'', meaning only critical alerts are
  included. If you want to include warnings also, use
  ``--nslimit=yellow''. 


 

\item[--rss] Generate RSS/RDF content delivery stream of your Hobbit
  alerts. This output format can be dynamically embedded in other web
  pages, much like the live newsfeeds often seen on web sites. Two RSS
  files will be generated, one reflects the BB2 page, the other
  reflects the BBNK page. They will be in the ``bb2.rss'' and
  ``bbnk.rss'' files, respectively. In addition, an RSS file will be
  generated for each page and/or subpage listing the hosts present on
  that page or subpage.  

 The FILENAME parameter previously allowed on the --rss option is now obsolete.  
 For more information about RSS/RDF content feeds, please see \url{http://www.syndic8.com/.}

 

\item[--rssextension=.EXTENSION] Sets the filename extension used for
  the RSS files generated by bbgen. By default, an extension of
  ``.rss'' is used. Note that you need to specify the ``dot''. 


 

\item[--rssversion=\\{0.91|0.92|1.0|2.0\\}] The desired output format of
  the RSS/RDF feed. Version 0.91 appears to be the most commonly used
  format, and is the default if this option is omitted. 


 

\item[--rsslimit=COLOR] The minimum color to include in the RSS feed -
  default is ``red'', meaning only critical alerts are included. If
  you want to include warnings also, use ``--rsslimit=yellow''. 

\end{description}

\section{OPTIONS USED BY CGI FRONT-ENDS}
\begin{description}
\item[--reportopts=START:END:DYNAMIC:STYLE] Invoke bbgen in
  report-generation mode. This is normally used by the
  \emph{bb-rep.cgi(1)}

 CGI script, but may also be used directly when pre-generating
 reports. The START parameter is the start-time for the report in Unix
 time\_t format (seconds since Jan 1st 1970 00:00 UTC); END is the
 end-time for the report; DYNAMIC is 0 for a pre-built report and 1
 for a dynamic (on-line) report; STYLE is ``crit'' to include only
 critical (red) events, ``non-crit'' to include all non-green events,
 and ``all'' to include all events. 


 

\item[--csv=FILENAME] Used together with --reportopts, this causes
  bbgen to generate an availability report in the form of a
  comma-separated values (CSV) file. This format is commonly used for
  importing into spreadsheets for further processing.  

 The CSV file includes Unix timestamps. To display these as human
 readable times in Excel, the formula
 \textbf{=C2/86400+DATEVALUE(1-jan-1970)}

 (if you have the Unix timestamp in the cell C2) can be used. The
 result cell should be formatted as a date/time field. Note that the
 timestamps are in UTC, so you may also need to handle local timezone
 and DST issues yourself. 


 

\item[--csvdelim=DELIMITER] By default, a comma is used to delimit
  fields in the CSV output. Some non-english spreadsheets use a
  different delimiter, typically semi-colon. To generate a CSV file
  with the proper delimiter, you can use this option to set the
  character used as delimiter. E.g. ``--csvdelim=;'' - note that this
  normally should be in double quotes, to prevent the Unix shell from
  interpreting the delimiter character as a commandline delimiter. 



\item[--snapshot=TIME] Generate a snapshot of the Hobbit pages, as
  they appeared at TIME. TIME is given as seconds since Jan 1st 1970
  00:00 UTC. Normally used via the \emph{bb-snapshot.cgi(1)} CGI script. 

 


\end{description}
\section{DEBUGGING OPTIONS}


 \begin{description}
\item[--debug] Causes bbgen to dump large amounts of debugging output
  to stdout, if it was compiled with the -DDEBUG enabled. When
  reporting a problem with bbgen, please try to reproduce the problem
  and provide the output from running bbgen with this option. 


 

\item[--timing] Dump information about the time spent by various parts
  of bbgen to stdout. This is useful to see what part of the
  processing is responsible for the run-time of bbgen.  

 Note: This information is also provided in the output sent to the Hobbit display when using the ``--report'' option. 

 


 


\end{description}

\section{BUILDING ALTERNATE PAGESETS}
 With version 1.4 of bbgen comes the possibility to generate multiple
 sets of pages from the same data.  

 Suppose you have two groups of people looking at the BB
 webpages. Group A wants to have the hosts grouped by the client, they
 belong to. This is how you have Hobbit set up - the default
 pageset. Now group B wants to have the hosts grouped by operating
 system - let us call it the ``os'' set. Then you would add the page
 layout to bb-hosts like this: 


 ospage win Microsoft Windows  
 ossubpage win-nt4 MS Windows NT 4  
 osgroup NT4 File servers  
 osgroup NT4 Mail servers  
 ossubpage win-xp MS Windows XP  
 ospage unix Unix  
 ossubpage unix-sun Solaris  
 ossubpage unix-linux Linux 


  This defines a set of pages with one top-level page (the bb.html
  page), two pages linked from bb.html (win.html and unix.html), and
  from e.g. the win.html page there are subpages win-nt4.html and
  win-xp.html  

 The syntax is identical to the normal ``page'' and ``subpage''
 directives in bb-hosts, but the directive is prefixed with the
 pageset name. Dont put any hosts in-between the page and subpage
 directives - just add all the directives at the top of the bb-hosts
 file.  

 How do you add hosts to the pages, then ? Simple - just put a tag
 ``OS:win-xp'' on the host definition line. The ``OS'' must be the
 same as prefix used for the pageset names, but in uppercase. The
 ``win-xp'' must match one of the pages or subpages defined within
 this pageset. E.g. 



  207.46.249.190 www.microsoft.com \# OS:win-xp \url{http://www.microsoft.com/} 
 64.124.140.181 www.sun.com \# OS:unix-sun \url{http://www.sun.com/}


  If you want the host to appear inside a group defined on that page,
  you must identify the group by number, starting at 1. E.g. to put a
  host inside the ``NT4 Mail servers'' group in the example above, use
  ``OS:win-nt4,2'' (the second group on the ``win-nt4'' page).  

 If you want the host to show up on the frontpage instead of a subpage, use ``OS:*'' . 


  All of this just defines the layout of the new pageset. To generate
  it, you must run bbgen once for each pageset you define -
  i.e. create an extension script like this: \begin{description}

\item[]\begin{verbatim}

#!/bin/sh

BBWEB="/hobbit/os" $BBHOME/bin/bbgen \
        --pageset=os --template=os \
        $BBHOME/www/os/

\end{verbatim}


\end{description}



  Save this to \$BBHOME/ext/os-display.sh, and set this up to run as a
  Hobbit extension; this means addng an extra section to
  hobbitlaunch.cfg to run it. 



  This generates the pages. There are some important options used here:  
 * BBWEB=''/hobbit/os'' environment variable, and the  
''\$BBHOME/www/os/''optionworktogether,andplacesthe  
newpagesetHTMLfilesinasubdirectoryoffthenormal  
Hobbitwebroot.IfyounormallyaccesstheHobbitpagesas  
''\url{http://hobbit.acme.com/hobbit/}``,youwillthenaccess  
thenewpagesetas''\url{http://hobbit.acme.com/hobbit/os/}``  
NB:ThedirectorygivenasBBWEBmustcontainasymbolic  
linktothe\$BBHOME/www/html/directory,orlinksto  
individualstatusmessageswillnotwork.Similarlinks  
shouldbemadeforthegifs/,help/andnotes/  
directories.  
 * ``--pageset=os'' tells bbgen to structure the webpages  
usingthe''os''layout,insteadofthedefaultlayout.  
 * ``--template=os'' tells bbgen to use a different set of  
header-andfooter-templates.Normallybbgenusesthe  
standardtemplatein\$BBHOME/web/bb\_headerand  
.../bb\_footer-withthisoption,itwillinsteaduse  
thefiles''os\_header''and''os\_footer''fromthe  
\$BBHOME/web/directory.Thisallowsyoutocustomize  
headersandfootersforeachpageset.Ifyoujustwant  
tousethenormaltemplate,youcanomitthisoption. 


 
\section{USING BBGEN FOR REPORTS}
 bbgen reporting is implemented via drop-in replacements for the standard Hobbit reporting scripts (bb-rep.sh and bb-replog.sh) installed in your webservers cgi-bin directory. 

  These two shell script have been replaced with two very small
  shell-scripts, that merely setup the Hobbit environment variables,
  and invoke the \emph{bb-rep.cgi(1)} or \emph{bb-replog.cgi(1)}
  scripts in \$BBHOME/bin/ 



  You can use bbgen commandline options when generating reports,
  e.g. to exclude certain types of tests
  (e.g. ``--ignorecolumns=msgs'') from the reports, to specify the
  name of the trends- and info- columns that should not be in the
  report, or to format the report differently
  (e.g. ``--subpagecolumns=2''). If you want certain options to be
  used when a report is generated from the web interface, put these
  options into your \$BBHOME/etc/hobbitserver.cfg file in the
  BBGENREPOPTS environment variable. 



  The report files generated by bbgen are stored in individual
  directories (one per report) below the \$BBHOME/www/rep/
  directory. These should be automatically cleaned up - as new reports
  are generated, the old ones get removed. 



  After installing, try generating a report. You will probably see
  that the links in the upper left corner (to bb-ack.html, bb2.html
  etc.) no longer works. To fix these, change your
  \$BBHOME/web/bbrep\_header file so these links do not refer to
  ``\&BBWEB'' but to the normal URL prefix for your Hobbit pages. 



 
\section{SLA REPORTING}
 bbgen reporting allows for the generation of true SLA (Service Level
 Agreement) reports, also for service periods that are not 24x7. This
 is enabled by defining a ``REPORTTIME:timespec'' tag for the hosts to
 define the service period, and optionally a ``WARNPCT:level'' tag to
 define the agreed availability. 


  Note: See \emph{bb-hosts(5)}
 for the exact syntax of these options. 


  ``REPORTTIME:timespec'' specifies the time of day when the service
  is expected to be up and running. By default this is 24 hours a day,
  all days of the week. If your SLA only covers Mon-Fri 7am - 8pm, you
  define this as ``REPORTTIME=W:0700:2000'', and the report generator
  will then compute both the normal 24x7 availability but also a ``SLA
  availability'' which only takes the status of the host during the
  SLA period into account. 



  The DOWNTIME:timespec parameter affects the SLA availability
  calculation. If an outage occurs during the time defined as possible
  ``DOWNTIME'', then the failure is reported with a status of
  ``blue''. (The same color is used if you ``disable'' then host using
  the Hobbit ``disable'' function). The time when the test status is
  ``blue'' is not included in the SLA calculation, neither in the
  amount of time where the host is considered down, nor in the total
  amount of time that the report covers. So ``blue'' time is
  effectively ignored by the SLA availability calculation, allowing
  you to have planned downtime without affecting the reported SLA
  availability. 



  Example: A host has ``DOWNTIME:*:0700:0730 REPORTTIME=W:0600:2200''
  because it is rebooted every day between 7am and 7.30am, but the
  service must be available from 6am to 10pm. For the day of the
  report, it was down from 7:10am to 7:15am (the planned reboot), but
  also from 9:53pm to 10:15pm. So the events for the day are: 



  
0700:greenfor10minutes(600seconds)  
0710:bluefor5minutes(300seconds)  
0715:greenfor14hours38minutes(52680seconds)  
2153:redfor22minutes(1320seconds)  
2215:green 


  The service is available for 600+52680 = 53280 seconds. It is down
  (red) for 420 seconds (the time from 21:53 until 22:00 when the SLA
  period ends). The total time included in the report is 15 hours (7am
  - 10pm) except the 5 minutes blue = 53700 seconds. So the SLA
  availability is 53280/53700 = 99,22\% 



  The ``WARNPCT:level'' tag is supported in the bb-hosts file, to set
  the availability threshold on a host-by-host basis. This threshold
  determines whether a test is reported as green, yellow or red in the
  reports. A default value can be set for all hosts with the via the
  BBREPWARN environment variable, but overridden by this tag. The
  level is given as a percentage, e.g. ``WARNPCT:98.5'' 

 
\section{PRE-GENERATED REPORTS}
 Normally, bbgen produce reports that link to dynamically generated webpages with the detailed status of a test (via the bb-replog.sh CGI script). 

  It is possible to have bbgen produce a report without these dynamic
  links, so the report can be exported to another server. It may also
  be useful to pre-generate the reports, to lower the load by having
  multiple users generate the same reports. 



  To do this, you must run bbgen with the ``--reportopts'' option to
  select the time interval that the report covers, the reporting style
  (critical, non-green, or all events), and to request that no dynamic
  pages are to be generated. 



  The syntax is: 


  
bbgen--reportopts=starttime:endtime:nodynamic:style 


  ``starttime'' and ``endtime'' are specified as Unix time\_t values,
  i.e. seconds since Jan 1st 1970 00:00 GMT. Fortunately, this can
  easily be computed with the GNU date utility if you use the ``+\%s''
  output option. If you don't have the GNU date utility, either pick
  that up from www.gnu.org; or you can use the ``etime'' utility for
  the same purpose, which is available from the archive at
  www.deadcat.net. 



  ``nodynamic'' is either 0 (for dynamic pages, the default) or 1 (for no dynamic, i.e. pre-generated, pages). 


  ``style'' is either ``crit'' (include critical i.e. red events only), ``nongr'' (include all non-green events), or ``all'' (include all events). 


  Other bbgen options can be used, e.g. ``--ignorecolumns'' if you want to exclude certain tests from the report. 


  You will normally also need to specify the BBWEB environment
  variable (it must match the base URL for where the report will be
  made accessible from), and an output directory where the report
  files are saved. If you specify BBWEB, you should probably also
  define the BBHELPSKIN and BBNOTESSKIN environment variables. These
  should point to the URL where your Hobbit help- and notes-files are
  located; if they are not defined, the links to help- and notes-files
  will point inside the report directory and will probably not work. 



  So a typical invocation of bbgen for a static report would be: 


  
START=`date+\%s--date=''22Jun200300:00:00''`  
END=`date+\%s--date=''22Jun200323:59:59''`  
BBWEB=/reports/bigbrother/daily/2003/06/22$\backslash$  
BBHELPSKIN=/hobbit/help$\backslash$  
BBNOTESSKIN=/hobbit/notes$\backslash$  
bbgen--reportopts=\$START:\$END:1:crit$\backslash$  
--subpagecolumns=2$\backslash$  
/var/www/docroot/reports/hobbit/daily/2003/06/22 


  The ``BBWEB'' setting means that the report will be available with a
  URL of
  ``\url{http://www.server.com/reports/hobbit/daily/2003/06/22}``. The
  report contains internal links that use this URL, so it cannot be
  easily moved to another location. 



  The last parameter is the corresponding physical directory on your
  webserver matching the BBWEB URL. You can of course create the
  report files anywhere you like - perhaps on another machine - and
  then move them to the webserver later on. 



  Note how the \emph{date(1)}
 utility is used to calculate the start- and end-time parameters. 


 
\section{SEE ALSO}
bb-hosts(5), hobbitserver.cfg(5), hobbitlaunch.cfg(5), bb-rep.cgi(1), bb-snapshot.cgi(1), hobbit(7) 

 



\section{Index}
\begin{description}
\item[NAME]
\item[SYNOPSIS]
\item[DESCRIPTION]
\item[OPTIONS]
\item[GENERAL OPTIONS]
\item[PAGE LAYOUT OPTIONS]
\item[COLUMN SELECTION OPTIONS]
\item[STATUS PROPAGATION OPTIONS]
\item[PURPLE STATUS OPTIONS]
\item[ALTERNATE PAGESET OPTIONS]
\item[ALTERNATE OUTPUT FORMATS]
\item[OPTIONS USED BY CGI FRONT-ENDS]
\item[DEBUGGING OPTIONS]
\item[BUILDING ALTERNATE PAGESETS]
\item[USING BBGEN FOR REPORTS]
\item[SLA REPORTING]
\item[PRE-GENERATED REPORTS]
\item[SEE ALSO]

\end{description}
