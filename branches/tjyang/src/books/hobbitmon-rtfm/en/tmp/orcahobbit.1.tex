



\chapter{ORCAHOBBIT}

\section{ORCAHOBBIT}
 Section: User Commands (1) 
Updated: Version Exp: 11 Jan 2008 
Index Return to Main Contents 

�\section{NAME}
hgcmd{orcahobbit} - Hobbit client utility to grab data from ORCA �\section{SYNOPSIS}
\textbf{orcahobbit --orca=PREFIX [options]}


 �
\section{NOTICE}
 This utility is included in the client distribution for Hobbit 4.2. However, the backend module to parse the data it sends it \textbf{NOT}
 included in Hobbit 4.2. It is possible to use the generic Hobbit NCV data handler in \emph{hobbitd\_rrd(8)}
 to process ORCA data, if you have an urgent need to do so. 

 �
\section{DESCRIPTION}
\textbf{orcahobbit}
 is an add-on tool for the Hobbit client. It is used to grab data collected by the ORCA data collection tool (orcallator.se), and send it to the Hobbit server in NCV format. 

  orcahobbit should run from the client \emph{hobbitlaunch(8)}
 utility, i.e. there must be an entry in the \emph{clientlaunch.cfg(5)}
 file for orcahobbit. 


 �
\section{OPTIONS}
\begin{description}
\item[--orca=PREFIX] The filename prefix for the ORCA data log. Typically this is the directory for the ORCA logs, followed by ``orcallator''. The actual filename for the ORCA logs include a timestamp and sequence number, e.g. ``orcallator-2006-06-20-000''. This option is required. 

 

\item[--debug] Enable debugging output. 

 


\end{description}
�\section{SEE ALSO}
hobbit(7), clientlaunch.cfg(5) 

 


  
�
\section{Index}
\begin{description}
\item[NAME]
\item[SYNOPSIS]
\item[NOTICE]
\item[DESCRIPTION]
\item[OPTIONS]
\item[SEE ALSO]

\end{description}
 
 This document was created by man2html, using the manual pages. 
 Time: 16:21:46 GMT, January 11, 2008 

