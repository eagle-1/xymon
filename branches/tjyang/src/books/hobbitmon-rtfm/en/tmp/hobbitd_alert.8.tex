



\chapter{HOBBITD\_ALERT}

\section{HOBBITD\_ALERT}
 Section: Maintenance Commands (8) 
Updated: Version Exp: 11 Jan 2008 
Index Return to Main Contents 
�\section{NAME}
 hobbitd\_alert - hobbitd worker module for sending out alerts �\section{SYNOPSIS}
\textbf{hobbitd\_channel --channel=page hobbitd\_alert [options]}


 �
\section{DESCRIPTION}
 hobbitd\_alert is a worker module for hobbitd, and as such it is normally run via the \emph{hobbitd\_channel(8)}
 program. It receives hobbitd page- and ack-messages from the ``page'' channel via stdin, and uses these to send out alerts about failed and recovered hosts and services. 

  The operation of this module is controlled by the \emph{hobbit-alerts.cfg(5)}
 file. This file holds the definition of rules and recipients, that determine who gets alerts, how often, for what servers etc. 


 �
\section{OPTIONS}
\begin{description}
\item[--config=FILENAME] Sets the filename for the \emph{hobbit-alerts.cfg}
 file. The default value is ``etc/hobbit-alerts.cfg'' below the Hobbit server directory. 

 

\item[--dump-config] Dumps the configuration after parsing it. May be useful to track down problems with configuration file errors. 

 

\item[--checkpoint-file=FILENAME] File where the current state of the hobbitd\_alert module is saved. When starting up, hobbitd\_alert will also read this file to restore the previous state. 

 

\item[--checkpoint-interval=N] Defines how often (in seconds) the checkpoint-file is saved. 

 

\item[--cfid] If this option is present, alert messages will include a line with ``cfid:N'' where N is the linenumber in the hobbit-alerts.cfg file that caused this message to be sent. This can be useful to track down problems with duplicate alerts. 

 

\item[--test HOST SERVICE [options]] Shows which alert rules matches the given HOST/SERVICE combination. Useful to debug configuration problems, and see what rules are used for an alert. 

  The possible options are:  
\textbf{--color=COLORNAME}
 The COLORNAME parameter is the color of the alert: red, yellow or purple.  
\textbf{--duration=SECONDS}
 The SECONDS parameter is the duration of the alert in seconds.  
\textbf{--group=GROUPNAME}
 The GROUPNAME paramater is a groupid string from the hobbit-clients.cfg file.  
\textbf{--time=TIMESTRING}
 The TIMESTRING parameter is the time-of-day for the alert, expressed as an absolute time in the epoch format (seconds since Jan 1 1970). This is easily obtained with the GNU date utility using the ``+\%s'' output format. 


 

\item[--debug] Enable debugging output. 

 


\end{description}
�\section{HOW HOBBIT DECIDES WHEN TO SEND ALERTS}
 The hobbitd\_alert module is responsible for sending out all alerts. When a status first goes to one of the ALERTCOLORS, hobbitd\_alert is notified of this change. It notes that the status is now in an alert state, and records the \textbf{timestamp}
 when this event started, and adds the alert to the list statuses that may potentially trigger one or more alert messages. 

  This list is then matched against the hobbit-alerts.cfg configuration. This happens at least once a minute, but may happen more often. E.g. when status first goes into an alert state, this will always trigger the matching to happen. 


  When scanning the configuration, hobbitd\_alert looks at all of the configuration rules. It also checks the DURATION setting against how long time has elapsed since the event started - i.e. against the timestamp logged when hobbitd\_alert first heard of this event. 


  When an alert recipient is found, the alert is sent and it is recorded when this recipient is due for his next alert message, based on the REPEAT setting defined for this recipient. The next time hobbitd\_alert scans the configuration for what alerts to send, it will still find this recipient because all of the configuration rules are fulfilled, but an alert message will not be generated until the repeat interval has elapsed. 


  It can happen that a status first goes yellow and triggers an alert, and later it goes red - e.g. a disk filling up. In that case, hobbitd\_alert clears the internal timer for when the next (repeat) alert is due for all recipients. You generally want to be told when something that has been in a warning state becomes critical, so in that case the REPEAT setting is ignored and the alert is sent. This only happens the first time such a change occurs - if the status switches between yellow and red multiple times, only the first transition from yellow-$>$red causes this override. 


  When an status recovers, a recovery message may be sent - depending on the configuration - and then hobbitd\_alert forgets everything about this status. So the next time it goes into an alert state, the entire process starts all over again. 


 


 �
\section{ENVIRONMENT}
\begin{description}
\item[MAIL] The first part of a command line used to send out an e-mail with a subject, typically set to ``/usr/bin/mail -s'' . hobbitd\_alert will add the subject and the mail recipients to form the command line used for sending out email alerts. 

 

\item[MAILC] The first part of a command line used to send out an e-mail without a subject. Typically this will be ``/usr/bin/mail''. hobbitd\_alert will add the mail recipients to form the command line used for sending out email alerts. 

 


\end{description}
�\section{FILES}
\begin{description}
\item[~hobbit/server/etc/hobbit-alerts.cfg]

 


\end{description}
�\section{SEE ALSO}
hobbit-alerts.cfg(5), hobbitd(8), hobbitd\_channel(8), hobbit(7) 

 


  
�
\section{Index}
\begin{description}
\item[NAME]
\item[SYNOPSIS]
\item[DESCRIPTION]
\item[OPTIONS]
\item[HOW HOBBIT DECIDES WHEN TO SEND ALERTS]
\item[ENVIRONMENT]
\item[FILES]
\item[SEE ALSO]

\end{description}
 
 This document was created by man2html, using the manual pages. 
 Time: 16:21:46 GMT, January 11, 2008 

