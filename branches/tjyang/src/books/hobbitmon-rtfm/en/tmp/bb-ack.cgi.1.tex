\chapter{BB-ACK.CGI}

section{NAME}
 bb-ack.cgi - Hobbit CGI script to acknowledge alerts �\section{SYNOPSIS}
\textbf{bb-ack.cgi?ACTION=action\&NUMBER=acknum\&DELAY=validity\&MESSAGE=text}


 �
\section{DESCRIPTION}
\textbf{bb-ack.cgi}
 is invoked as a CGI script via the bb-ack.sh CGI wrapper. 

  bb-ack.cgi is passed a QUERY\_STRING environment variable with the ACTION, NUMBER, DELAY and MESSAGE parameters. 


 �
\section{PARAMETERS}
 ACTION is the action to perform. The only supported action is ``Ack'' to acknowledge an alert. 

  NUMBER is the number identifying the host/service to be acknowledged. It is included in all alert-messages sent out by Hobbit. 


  DELAY is the time (in minutes) that the acknowledge is valid. 


  MESSAGE is an optional text which will be shown on the status page while the acknowledgment is active. You can use it to e.g. tell users not to contact you about the problem, or inform them when the problem is expected to be resolved. 


 �
\section{OPTIONS}
\begin{description}
\item[--no-pin] bb-ack.cgi normally requires the user to enter the acknowledgment code received in an alert message. If you run it with this option, the user will instead get a list of the current non-green statuses, and he may send an acknowledge without knowing the code. 

 

\item[--no-cookies] Normally, bb-ack.cgi uses a cookie sent by the browser to initially filter the list of hosts presented. If this is not desired, you can turn off this behaviour by calling bb-ack.cgi with the --no-cookies option. This would normally be placed in the CGI\_ACK\_OPTS setting in \emph{hobbitcgi.cfg(5)}


 

\item[--env=FILENAME] Loads the environment defined in FILENAME before executing the CGI script. 

 

\item[--debug] Enables debugging output. 

 


\end{description}
�\section{FILES}
\begin{description}
\item[\$BBHOME/web/acknowledge\_header] HTML header file for the generated web page 

 

\item[\$BBHOME/web/acknowledge\_footer] HTML footer file for the generated web page 

 

\item[\$BBHOME/web/acknowledge\_form] Query form displayed when bb-ack.cgi is called with no parameters. 

 


\end{description}
�\section{ENVIRONMENT VARIABLES}
\begin{description}
\item[BBHOME] Used to locate the template files for the generated web pages. 

 

\item[QUERY\_STRING] Contains the parameters for the CGI script. 

 


\end{description}
�\section{BUGS}
 When using alternate pagesets, hosts will only show up on the acknowledgment page if this is accessed from the primary page in which they are defined. So if you have hosts on multiple pages, they will only be visible for acknowledging from their main page which is not what you would expect. 

 �
\section{SEE ALSO}
bbgen(1), bb-hosts(5), hobbitserver.cfg(5) 

 
