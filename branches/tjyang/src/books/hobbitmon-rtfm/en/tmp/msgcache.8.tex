



\chapter{MSGCACHE}

\section{MSGCACHE}
 Section: Maintenance Commands (8) 
Updated: Version Exp: 11 Jan 2008 
Index Return to Main Contents 
�\section{NAME}
 msgcache - Cache client messages for later pickup by hobbitfetch 

 �
\section{SYNOPSIS}
\textbf{msgcache [options]}


 �
\section{DESCRIPTION}
\textbf{msgcache}
 implements a Hobbit message cache. It is intended for use with clients which cannot deliver their data to the Hobbit server in the normal way. Instead of having the client tools connect to the Hobbit server, msgcache runs locally and the client tools then deliver their data to the msgcache daemon. The msgcache daemon is then polled regularly by the \emph{hobbitfetch(8)}
 utility, which collects the client messages stored by msgcache and forwards them to the Hobbit server. 

 \textbf{NOTE:}
 When using msgcache, the \textbf{BBDISP}
 setting for the clients should be \textbf{BBDISP=127.0.0.1}
 instead of pointing at the real Hobbit server. 


 �
\section{RESTRICTIONS}
 Clients delivering their data to msgcache instead of the real Hobbit server will in general not notice this. Specifically, the client configuration data provided by the Hobbit server when a client delivers its data is forwarded through the hobbitfetch / msgcache chain, so the normal centralized client configuration works. 

  However, other commands which rely on clients communicating directly with the Hobbit server will not work. This includes the \textbf{config}
 and \textbf{query}
 commands which clients may use to fetch configuration files and query the Hobbit server for a current status. 


  The \textbf{download}
 command also does not work with msgcache. This means that the automatic client update facility will not work for clients communicating via msgcache. 


 �
\section{OPTIONS}
\begin{description}
\item[--listen=IPADDRESS[:PORT]] Defines the IP-address and portnumber where msgcache listens for incoming connections. By default, msgcache listens for connections on all network interfaces, port 1984. 

 

\item[--server=IPADDRESS[,IPADDRESS]] Restricts which servers are allowed to pick up the cached messages. By default anyone can contact the msgcache utility and request all of the cached messages. This option allows only the listed servers to request the cached messages. 

 

\item[--max-age=N] Defines how long cached messages are kept. If the message has not been picked up with N seconds after being delivered to msgcache, it is silently discarded. Default: N=600 seconds (10 minutes). 

 

\item[--daemon] Run as a daemon, i.e. msgcache will detach from the terminal and run as a background task 

 

\item[--no-daemon] Run as a foreground task. This option must be used when msgcache is started by \emph{hobbitlaunch(8)}
 which is the normal way of running msgcache. 

 

\item[--pidfile=FILENAME] Store the process ID of the msgcache task in FILENAME. 

 

\item[--logfile=FILENAME] Log msgcache output to FILENAME. 

 

\item[--debug] Enable debugging output. 

 


\end{description}
�\section{SEE ALSO}
hobbitfetch(8), hobbit(7) 

 


  
�
\section{Index}
\begin{description}
\item[NAME]
\item[SYNOPSIS]
\item[DESCRIPTION]
\item[RESTRICTIONS]
\item[OPTIONS]
\item[SEE ALSO]

\end{description}
 
 This document was created by man2html, using the manual pages. 
 Time: 16:21:46 GMT, January 11, 2008 

