



\chapter{HOBBIT-MAILACK}

\section{HOBBIT-MAILACK}
 Section: Maintenance Commands (8) 
Updated: Version Exp: 11 Jan 2008 
Index Return to Main Contents 
�\section{NAME}
 hobbit-mailack - permit acknowledging alerts via e-mail �\section{SYNOPSIS}
\textbf{hobbit-mailack --env=FILENAME [--debug]}


 �
\section{DESCRIPTION}
 hobbit-mailack normally runs as an input mail-filter for the hobbit user, e.g. by being called from the hobbit users' \emph{procmailrc(5)}
 file. hobbit-mailack recognizes e-mails that are replies to \emph{hobbitd\_alert(8)}
 mail alerts, and converts the reply mail into an acknowledge message that is sent to the Hobbit system. This permits an administrator to acknowledge an alert via e-mail. 

 �
\section{ADDING INFORMATION TO THE REPLY MAIL}
 By default, an acknowledgment is valid for 1 hour. If you know in advance that solving the problem is going to take longer, you can change this by adding \textbf{delay=DURATION}
 to the subject of your mail reply or on a line in the reply message. Duration is in minutes, unless you add a trailing 'h' (for 'hours'), 'd' (for 'days') or 'w' (for 'weeks'). 

  You can also include a message that will show up on the status-page together with the acknowledgment, e.g. to provide an explanation for the issue or some other information to the users. You can either put it at the end of the subject line as \textbf{msg=Some random text}
, or you can just enter it in the e-mail as the first non-blank line of text in the mail (a ``delay=N'' line is ignored when looking for the message text). 


 �
\section{USE WITH PROCMAIL}
 To setup hobbit-mailack, create a \textbf{.procmailrc}
 file in the hobbit-users home-directory with the following contents: \begin{description}
\item[]\begin{verbatim}

DEFAULT=$HOME/Mailbox
LOGFILE=$HOME/procmail.log
:0
| $HOME/server/bin/hobbit-mailack --env=$HOME/server/etc/hobbitserver.cfg

\end{verbatim}


 


\end{description}
�\section{USE WITH QMAIL}
 If you are using Qmail to deliver mail locally, you can run hobbit-mailack directly from a \textbf{.qmail}
 file. Setup the hobbit-users .qmail file like this: \begin{description}
\item[]\begin{verbatim}

| $HOME/server/bin/hobbit-mailack --env=$HOME/server/etc/hobbitserver.cfg

\end{verbatim}


 


\end{description}
�\section{OPTIONS}
\begin{description}
\item[--env=FILENAME] Load environment from FILENAME, usually hobbitserver.cfg. 

 

\item[--debug] Dont send a message to hobbitd, but dump the message to stdout. 

 


\end{description}
�\section{SEE ALSO}
hobbitd\_alert(8), hobbitd(8), hobbit(7) 

 


  
�
\section{Index}
\begin{description}
\item[NAME]
\item[SYNOPSIS]
\item[DESCRIPTION]
\item[ADDING INFORMATION TO THE REPLY MAIL]
\item[USE WITH PROCMAIL]
\item[USE WITH QMAIL]
\item[OPTIONS]
\item[SEE ALSO]

\end{description}
 
 This document was created by man2html, using the manual pages. 
 Time: 16:21:46 GMT, January 11, 2008 

