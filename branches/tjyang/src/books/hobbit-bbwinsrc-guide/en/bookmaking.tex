\chapter{Making of this book}
\label{chap:bookmaking}

\section{Where to download the book source}
\label{sec:srcinstall:unixlike}

If you are using a Unix-like system that has a sufficiently recent
version of Python (2.3~or newer) available, it is easy to install
Mercurial from source.
\begin{enumerate}
\item Using svn to check out source code from sourceforge:
  \begin{codesample4}
      svn co https://bbwin.svn.sourceforge.net/svnroot/bbwin bbwin
  \end{codesample4}
\item Go into the source directory and run the installer script.  This
  will build Mercurial and install it in your home directory.
  \begin{codesample4}
    cd mercurial-\emph{version}
    python setup.py install --force --home=\$HOME
  \end{codesample4}
\end{enumerate}
Once the install finishes, Mercurial will be in the \texttt{bin}
subdirectory of your home directory.  Don't forget to make sure that
this directory is present in your shell's search path.

You will probably need to set the \envar{PYTHONPATH} environment
variable so that the Mercurial executable can find the rest of the
Mercurial packages.  For example, on my laptop, I have set it to
\texttt{/home/bos/lib/python}.  The exact path that you will need to
use depends on how Python was built for your system, but should be
easy to figure out.  If you're uncertain, look through the output of
the installer script above, and see where the contents of the
\texttt{mercurial} directory were installed to.

\section{On Windows}

Building and installing Mercurial on Windows requires a variety of
tools, a fair amount of technical knowledge, and considerable
patience.  I very much \emph{do not recommend} this route if you are a
``casual user''.  Unless you intend to hack on Mercurial, I strongly
suggest that you use a binary package instead.

If you are intent on building Mercurial from source on Windows, follow
the ``hard way'' directions on the Mercurial wiki at
\url{http://www.selenic.com/mercurial/wiki/index.cgi/WindowsInstall},
and expect the process to involve a lot of fiddly work.

%%% Local Variables: 
%%% mode: latex
%%% TeX-master: "00book"
%%% End: 
