%%%%%%%%%%%%%%%%%%%%%%%%%%%%%%%%%%%%%%%%%%%%%%%%%%%%%%%%%%%%%%%%%%%%%%%%%%%%%%
%
%%%%%%%%%%%%%%%%%%%%%%%%%%%%%%%%%%%%%%%%%%%%%%%%%%%%%%%%%%%%%%%%%%%%%%%%%%%%%%
\chapter{Hobbit Server Administration}

%%%%%%%%%%%%%%%%%%%%%%%%%%%%%%%%%%%%%%%%%%%%%%%%%%%%%%%%%%%%%%%%%%%%%%%%%%%%%%
%
%%%%%%%%%%%%%%%%%%%%%%%%%%%%%%%%%%%%%%%%%%%%%%%%%%%%%%%%%%%%%%%%%%%%%%%%%%%%%%

\newpage
\section{HOBBIT-ENADIS.CGI}
 hobbit-enadis.cgi - CGI program to enable/disable Hobbit tests \
\subsection{SYNOPSIS}
\textbf{hobbit-enadis.cgi (invoked via CGI from webserver)}


 
\subsection{DESCRIPTION}
\textbf{hobbit-enadis.cgi} is a CGI tool for disabling and enabling
hosts and tests monitored by Hobbit. You can disable monitoring of a
single test, all tests for a host, or multiple hosts - immediately or
at a future point in time. 


  hobbit-enadis.cgi runs as a CGI program, invoked by your
  webserver. It is normally run via a wrapper shell-script in the
  secured CGI directory for Hobbit. 



  hobbit-enadis.cgi is the back-end script for the enable/disable form
  present on the ``info'' status-pages. It can also run in
  ``stand-alone'' mode, in which case it displays a web form allowing
  users to select what to enable or disable. 



 


 
\subsection{OPTIONS}
\begin{description}
\item[--no-cookies] Normally, hobbit-enadis.cgi uses a cookie sent by
  the browser to initially filter the list of hosts presented. If this
  is not desired, you can turn off this behaviour by calling
  bb-ack.cgi with the --no-cookies option. This would normally be
  placed in the CGI\_ENADIS\_OPTS setting in \emph{hobbitcgi.cfg(5)}



 

\item[--env=FILENAME] Load the environment from FILENAME before
  executing the CGI. 


 

\item[--area=NAME] Load environment variables for a specific area. NB:
  if used, this option must appear before any --env=FILENAME option. 

\end{description}
\

subsection{FILES}
\begin{description}
\item [\$BBHOME/web/maint\_{header,form,footer}] HTML template header 

 


\end{description}
\subsection{BUGS}
 When using alternate pagesets, hosts will only show up on the
 Enable/Disable page if this is accessed from the primary page in
 which they are defined. So if you have hosts on multiple pages, they
 will only be visible for disabling from their main page which is not
 what you would expect. 


 
\subsection{SEE ALSO}
hobbit(7) 

 
%%%%%%%%%%%%%%%%%%%%%%%%%%%%%%%%%%%%%%%%%%%%%%%%%%%%%%%%%%%%%%%%%%%%%%%%%%%%%%
%
%%%%%%%%%%%%%%%%%%%%%%%%%%%%%%%%%%%%%%%%%%%%%%%%%%%%%%%%%%%%%%%%%%%%%%%%%%%%%%
\newpage
\section{BB-ACK.CGI}
 bb-ack.cgi - Hobbit CGI script to acknowledge alerts \

\subsection{SYNOPSIS}
\textbf{bb-ack.cgi?ACTION=action\&NUMBER=acknum\&DELAY=validity\&MESSAGE=text}


 
\subsection{DESCRIPTION}
\textbf{bb-ack.cgi}
 is invoked as a CGI script via the bb-ack.sh CGI wrapper. 

  bb-ack.cgi is passed a QUERY\_STRING environment variable with the
  ACTION, NUMBER, DELAY and MESSAGE parameters. 



 
\subsection{PARAMETERS}
 ACTION is the action to perform. The only supported action is ``Ack'' to acknowledge an alert. 

  NUMBER is the number identifying the host/service to be
  acknowledged. It is included in all alert-messages sent out by
  Hobbit. 



  DELAY is the time (in minutes) that the acknowledge is valid. 


  MESSAGE is an optional text which will be shown on the status page
  while the acknowledgment is active. You can use it to e.g. tell
  users not to contact you about the problem, or inform them when the
  problem is expected to be resolved. 



 
\subsection{OPTIONS}
\begin{description}
\item[--no-pin] bb-ack.cgi normally requires the user to enter the
  acknowledgment code received in an alert message. If you run it with
  this option, the user will instead get a list of the current
  non-green statuses, and he may send an acknowledge without knowing
  the code. 


 

\item[--no-cookies] Normally, bb-ack.cgi uses a cookie sent by the
  browser to initially filter the list of hosts presented. If this is
  not desired, you can turn off this behaviour by calling bb-ack.cgi
  with the --no-cookies option. This would normally be placed in the
  CGI\_ACK\_OPTS setting in \emph{hobbitcgi.cfg(5)}



 
\item[--env=FILENAME] Loads the environment defined in FILENAME before executing the CGI script. 

 

\item[--debug] Enables debugging output. 

 


\end{description}
\subsection{FILES}
\begin{description}
\item[\$BBHOME/web/acknowledge\_header] HTML header file for the generated web page 

 

\item[\$BBHOME/web/acknowledge\_footer] HTML footer file for the generated web page 

 

\item[\$BBHOME/web/acknowledge\_form] Query form displayed when bb-ack.cgi is called with no parameters. 

 
\end{description}
\subsection{ENVIRONMENT VARIABLES}
\begin{description}
\item[BBHOME] Used to locate the template files for the generated web pages. 

 

\item[QUERY\_STRING] Contains the parameters for the CGI script. 

 


\end{description}
\subsection{BUGS}
 When using alternate pagesets, hosts will only show up on the
 acknowledgment page if this is accessed from the primary page in
 which they are defined. So if you have hosts on multiple pages, they
 will only be visible for acknowledging from their main page which is
 not what you would expect. 

 
\subsection{SEE ALSO}
bbgen(1), bb-hosts(5), hobbitserver.cfg(5) 


%%%%%%%%%%%%%%%%%%%%%%%%%%%%%%%%%%%%%%%%%%%%%%%%%%%%%%%%%%%%%%%%%%%%%%%%%%%%%%
%
%%%%%%%%%%%%%%%%%%%%%%%%%%%%%%%%%%%%%%%%%%%%%%%%%%%%%%%%%%%%%%%%%%%%%%%%%%%%%%
\newpage
\section{HOBBIT-GHOSTS.CGI}
 hobbit-ghosts.cgi - CGI program to view ghost clients
 \subsection{SYNOPSIS}
\textbf{hobbit-ghosts.cgi}


 
\subsection{DESCRIPTION}
\textbf{hobbit-ghosts.cgi} is invoked as a CGI script via the hobbit-ghosts.sh CGI wrapper. 

  It generates a listing of the Hobbit clients that have reported data
  to the Hobbit server, but are not listed in the \emph{bb-hosts(5)}
  file. Data from these clients - called ``ghosts'' - are ignored,
  since Hobbit does not know which webpage to present the data on. 



  The listing includes the hostname that the client reports with, the
  IP-address where the report came from, and how long ago the report
  arrived. 



  By far the most common reason for hosts showing up here is that the
  client uses a hostname without a DNS domain, but the bb-hosts file
  uses the hostname with the DNS domain. Or vice versa. You can then
  use a \textbf{CLIENT} setting in the bb-hosts file to match the two
  hostnames together. 



 
\subsection{OPTIONS}
\begin{description}
\item[--env=FILENAME] Loads the environment defined in FILENAME before executing the CGI script. 


\end{description}
\subsection{SEE ALSO}
bb-hosts(5), hobbitserver.cfg(5) 

 

%%%%%%%%%%%%%%%%%%%%%%%%%%%%%%%%%%%%%%%%%%%%%%%%%%%%%%%%%%%%%%%%%%%%%%%%%%%%%%
%
%%%%%%%%%%%%%%%%%%%%%%%%%%%%%%%%%%%%%%%%%%%%%%%%%%%%%%%%%%%%%%%%%%%%%%%%%%%%%%
\newpage
\section{TRIMHISTORY}
\subsection{NAME}
 \motohbcmd{trimhistory} - Remove old Hobbit history-log entries 

subsection{SYNOPSIS}
\textbf{trimhistory --cutoff=TIME [options]}


 
\subsection{DESCRIPTION}
 The \textbf{trimhistory}
 tool is used to purge old entries from the Hobbit history logs. These
 logfiles accumulate information about all status changes that have
 occurred for any given service, host, or the entire Hobbit system,
 and is used to generate the event- and history-log webpages. 


  Purging old entries can be done while Hobbit is running, since the
  tool takes care not to commit updates to a file if it changes
  mid-way through the operation. In that case, the update is aborted
  and the existing logfile is left untouched. 



  Optionally, this tool will also remove logfiles from hosts that are
  no longer defined in the Hobbit \emph{bb-hosts(5)} file. As an
  extension, even logfiles from services can be removed, if the
  service no longer has a valid status-report logged in the current
  Hobbit status. 



 
\subsection{OPTIONS}
\begin{description}
\item[--cutoff=TIME] This defines the cutoff-time when processing the
  history logs. Entries dated before this time are discarded. TIME is
  specified as the number of seconds since the beginning of the
  Epoch. This is easily generated by the GNU \emph{date(1)} utility,
  e.g. the following command will trim history logs of all entries
  prior to Oct. 1st 2004:  



  
trimhistory--cutoff=`date+\%s--date=''1Oct2004''` 


 

\item[--outdir=DIRECTORY] Normally, files in the BBHIST directory are
  replaced. This option causes trimhistory to save the shortened
  history logfiles to another directory, so you can verify that the
  operation works as intended. The output directory must exist. 


 

\item[--drop] Causes trimhistory to delete files from hosts that are
  not listed in the \emph{bb-hosts(5)} file. 

 

\item[--dropsvcs] Causes trimhistory to delete files from services
  that are not currently tracked by Hobbit. Normally these files would
  be left untouched if only the host exists. 

 

\item[--droplogs] Process the BBHISTLOGS directory also, and delete
  status-logs from events prior to the cut-off time. Note that this
  can dramatically increase the processing time, since there are often
  lots and lots of files to process. 


 

\item[--progress[=N]] This will cause trimhistory to output a status
  line for every N history logs or status-log collections it
  processes, to indicate how far it has progressed. The default
  setting for N is 100. 


 

\item[--env=FILENAME] Loads the environment from FILENAME before executing trimhistory. 

 

\item[--debug] Enable debugging output. 

 
\end{description}
\subsection{FILES}
\begin{description}
\item[\$BBHIST/allevents] The eventlog of all events that have happened in Hobbit. 

 

\item[\$BBHIST/HOSTNAME] The per-host eventlogs. 

 

\item[\$BBHIST/HOSTNAME.SERVICE] The per-service eventlogs. 

 

\item[\$BBHISTLOGS/*/*] The historical status-logs. 

 


\end{description}
\subsection{ENVIRONMENT VARIABLES}
\begin{description}
\item[BBHIST] The directory holding all history logs. 

 

\item[BBHISTLOGS] The top-level directory for the historical status-log collections. 

 

\item[BBHOSTS] The location of the bb-hosts file, holding the list of currently known hosts in Hobbit. 


\end{description}
\subsection{SEE ALSO}
hobbit(7), bb-hosts(5) 


%%%%%%%%%%%%%%%%%%%%%%%%%%%%%%%%%%%%%%%%%%%%%%%%%%%%%%%%%%%%%%%%%%%%%%%%%%%%%%
%
%%%%%%%%%%%%%%%%%%%%%%%%%%%%%%%%%%%%%%%%%%%%%%%%%%%%%%%%%%%%%%%%%%%%%%%%%%%%%%
\newpage
\section{HOBBIT-ACKINFO.CGI}
 hobbit-ackinfo.cgi - Hobbit CGI script to acknowledge alerts \subsection{SYNOPSIS}
\textbf{hobbit-ackinfo.cgi}


 
\subsection{DESCRIPTION}
\textbf{hobbit-ackinfo.cgi}
 is invoked as a CGI script via the hobbit-ackinfo.sh CGI wrapper. 

  hobbit-ackinfo.cgi is used to acknowledge an alert on the Hobbit
  ``Critical Systems'' view, generated by the
  \emph{hobbit-nkview.cgi(1)} utility. This allows the staff viewing
  the Critical Systems view to acknowledge alerts with a ``Level 1''
  alert, thereby removing the alert from the Critical Systems view. 



  Note that the Level 1 alert generated by the hobbit-ackinfo.cgi
  utility does \textbf{NOT} stop alerts from being sent. 



  In a future version of Hobbit (after Hobbit 4.2), this utility will
  also be used for acknowledging alerts at other levels. 



 
\subsection{OPTIONS}
\begin{description}
\item[--level=NUMBER] Sets the acknowledgment level. This is typically
  used to force a specific level of the acknowledgment, e.g. a level 1
  acknowledge when called from the Critical Systems view. 


 

\item[--validity=TIME] Sets the validity of the acknowledgment. By
  default this is taken from the CGI parameters supplied by the user. 


 

\item[--sender=STRING] Logs STRING as the sender of the
  acknowledgment. By default, this is taken from the loginname of the
  webuser sending the acknowledgment. 


 

\item[--env=FILENAME] Loads the environment defined in FILENAME before executing the CGI script. 

 

\item[--area=NAME] Load environment variables for a specific area. NB:
  if used, this option must appear before any --env=FILENAME option. 


 

\item[--debug] Enables debugging output. 


\end{description}
\subsection{ENVIRONMENT VARIABLES}
\begin{description}
\item[BBHOME] Used to locate the template files for the generated web pages. 


\item[QUERY\_STRING] Contains the parameters for the CGI script. 

\end{description}
\subsection{SEE ALSO}
hobbit-nkview.cgi(1), hobbit(7) 


